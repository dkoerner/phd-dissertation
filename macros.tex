
\newcommand{\TD}[1]{\todo[inline]{#1}}
\newcommand{\ud}{\,\mathrm{d}} % for integrals
\newcommand{\SHBR}{Y_{\mathbb{R}}} % for integrals
\newcommand{\SHBC}{Y_{\mathbb{C}}} % for integrals
\newcommand{\iu}{{i\mkern1mu}}
\newcommand{\icaption}[1]{\caption{\emph{#1}}} % italic caption 
\newcommand{\colvec}[3]{\ensuremath{\begin{pmatrix}#1\\#2\\#3\end{pmatrix}}}
\newcommand{\norm}[1]{\left\lVert#1\right\rVert}
\renewcommand{\vec}[1]{\boldsymbol{\mathbf{#1}}}
\newcommand{\phase}{\rho}
\newcommand{\mydash}{ \textendash\ }

%mathcolor for colored equations in appendix
\makeatletter
\def\mathcolor#1#{\@mathcolor{#1}}
\def\@mathcolor#1#2#3{%
  \protect\leavevmode
  \begingroup\color#1{#2}#3\endgroup
}
\makeatother

% adds a parallel for section to algorithm --------
% declaration of the new block
\algblock{ParFor}{EndParFor}
% customising the new block
\algnewcommand\algorithmicparfor{\textbf{parallel for}}
\algnewcommand\algorithmicpardo{\textbf{do}}
\algnewcommand\algorithmicendparfor{\textbf{end\ parallel for}}
\algrenewtext{ParFor}[1]{\algorithmicparfor\ #1\ \algorithmicpardo}
\algrenewtext{EndParFor}{\algorithmicendparfor}
