\chapter{Flux-limited Diffusion for Multiple Scattering in Participating Media}
\label{sec:fld}

The previous chapter highlighted that the first order truncation of the $P_N$-equations can be collapsed into a simple diffusion equation, which has a low memory footprint and can be solved very efficiently using a multigrid solver. The diffusion approximation therefore offers significant advantages over the $P_N$-method in terms of required computational resources. At the same time, it gives only a poor approximation to the ground truth result.

In this chapter, flux-limited diffusion is introduced to the problem of rendering participating media in computer graphics. The theory has been invented in the astrophysics domain and is primarily attributed to Levermore et al.~\cite{Levermore81}. Their work introduces the theory and discusses its properties for certain canonical problems, such as a single point source within a homogeneous participating medium. In this chapter, a novel numerical method is devised, based on flux-limited diffusion theory. It allows to solve for the fluence on a finite difference grid with higher accuracy than the diffusion approximation, at a significantly lower cost than $P_N$.

The chapter continues with revisiting the moment closure problem from section~\ref{sec:moment_closure} and introduces the flux-limit constraint, which is violated by diffusion and causes energy loss. The streaming transport regime is introduced, for which an advection equation is derived in section~\ref{sec:fld_streaming_limit_approximation}. Flux-limited diffusion is derived as a means to blend between the two extreme modes of transport using local information. This blending is driven by the variable Eddington factor, which is introduced in section~\ref{sec:fld_vef}. Various concrete realizations of the Eddington factor are given and compared in section~\ref{sec:fld_vef_factors}. Finally, an iterative solver is presented in section~\ref{sec:fld_solver}, for which results are presented in section~\ref{sec:fld_results}. Finally the chapter closes with a discussion in section~\ref{sec:fld_discussion}.


\section{Moment Closure Problem Revisited}
\label{sec:moment_problem_revisited}

In section~\ref{sec:moment_closure}, it was established that a choice for the second moment of the radiance distribution was required to close the system of equations, which was found by collapsing the moment expansion of the radiative transfer equation, truncated after the first moment. The classical diffusion approximation was derived by assuming an isotropic radiance distribution for the second moment. In this section, a closer look is taken at the moments and an important feature is carved out, which explains the inaccuracy of classical diffusion approximation in certain scenarios.

The distribution of power $\phi$ over solid angle was given by the normalized radiance $\widehat{L}$. Since it is a probability distribution, it has to integrate to one, which is verified by
\begin{align*}
\int_{\Omega}{\hat{L}(\vec{x}, \omega)\ud\omega}=
\int_{\Omega}{\frac{L\left(\vec{x}, \omega\right)}{\phi\left(\vec{x}\right)}\ud\omega}=
\frac{1}{\phi\left(\vec{x}\right)}
\int_{\Omega}{L\left(\vec{x}, \omega\right)\ud\omega}
=1
=\hat{L}_0
\end{align*}
Further, the normalized radiance $\widehat{L}$ has to be non-negative to be a probability distribution:
\begin{align*}
\hat{L}(\vec{x}, \omega)\ge 0 \qquad \text{for all unit directions } \omega
\end{align*}
which is verified by the fact, that the radiance field is non-negative by definition:
\begin{align*}
L\left(\vec{x}, \omega\right) \ge 0
\implies
\int_{\Omega}{L(\vec{x}, \omega)\ud\omega}=\phi\left(\vec{x}\right)\ge 0
\implies
\frac{L\left(\vec{x}, \omega\right)}{\phi\left(\vec{x}\right)}=\widehat{L}(\vec{x}, \omega) \ge 0
\end{align*}

Akhiezer~\cite{Akhiezer65} established, that the probability distribution contraints on a function results in a set of inequalities between moments of this function. In particular, the non-negativity constraint on the radiance field $L$ and its normalized distribution impose a constraint between the zero moment and the first moment. In case of the radiance field, this constraint can be derived by considering the dot product between direction vectors $\omega'$ and $\omega$, both of unit length. This results in the equation:
\begin{align*}
-1 \le \omega'\cdot\omega \le 1
\implies
0 \le 1 - \omega'\cdot\omega \le 1
\implies
\int_{\Omega}{\left(1-\omega'\cdot\omega\right)L\left(\vec{x}, \omega\right)\ud\omega}\ge 0
\end{align*}
which can be rearranged into:
\begin{align*}
\int_{\Omega}{\left(1-\omega'\cdot\omega\right)L\left(\vec{x}, \omega\right)\ud\omega} &\ge 0
\\
\int_{\Omega}{L\left(\vec{x}, \omega\right)\ud\omega}
-\int_{\Omega}{\omega'\cdot\omega L\left(\vec{x}, \omega\right)\ud\omega}
&\ge 0
\\
\int_{\Omega}{L\left(\vec{x}, \omega\right)\ud\omega}
-\omega'\cdot\int_{\Omega}{L\left(\vec{x}, \omega\right)\omega\ud\omega}
&\ge 0
\\
\phi\left(\vec{x}\right)
-\norm{\vec{E}\left(\vec{x}\right)}
&\ge 0
\\
\phi\left(\vec{x}\right)
&\ge \norm{\vec{E}\left(\vec{x}\right)}
\end{align*}
This constraint states, that the total power at position $\vec{x}$ must never exceed the length of the flux-vector. Considering that the flux-vector is found by integrating the radiance field over solid angle multiplied by a weighting factor between $\left[-1, 1\right]$ (the direction vector component). This must never be larger than the fluence, which is the integral over the radiance field itsself.

A very similar constraint can be derived for the normalized radiance $\widehat{L}$, by following the steps above. This results in:
\begin{align}
\label{eq:fld_flux_limit}
1
\ge
\norm{
\frac{\vec{E}\left(\vec{x}\right)}{\phi\left(\vec{x}\right)}
}
\end{align}
In the case of the classical diffusion approximation $\vec{E}\approx-D\nabla\phi$ (equation~\ref{eq:diffusion_ficks_law}). Inserting this into equation~\ref{eq:fld_flux_limit} gives:
\begin{align}
\label{eq:fld_flux_limit}
1
\ge
\norm{
\frac{\nabla\phi\left(\vec{x}\right)}{3\sigma_t\left(\vec{x}\right)\phi\left(\vec{x}\right)}
}
\end{align}
It shows, that with the classical diffusion approximation, this constraint can be violated when the extinction coeffient $\sigma_t$ becomes very small, meaning in the presence of very thin medium. It breaks down completely in case of vacuum $\sigma_t=0$. The constraint can also be violated if the the fluence gradient becomes very large in relation to $\sigma_t\phi$. This happens very close to point light sources and near strong density gradients in the medium.

Violation of the constraint above means, that for the classical diffusion approximation, the flux vector is much larger than what is physically possible. Since the flux vector governs light transport with diffusion, the literature often refers to classical diffusion as to suffer from unphysical fluxes or unphysical light transport.


\subsubsection*{A Transport Regime Measure}

The right hand side of equation~\ref{eq:fld_flux_limit} is an important measure, which allows to detect at every point within the domain, where and to what extend the diffusion approximation fails. For this only local information about the moments and extinction coefficient is needed. It is an important building block for the technique presented in this chapter, which seeks to improve the accuracy of the diffusion approximation.

The flux-limit in equation~\ref{eq:fld_flux_limit} can be factorized to show a clearer intuition
\begin{align}
\label{eq:flux_limit_factors}
\norm{
\frac{\nabla\phi\left(\vec{x}\right)}{3\sigma_t\left(\vec{x}\right)\phi\left(\vec{x}\right)}
}
=
\frac{1}{3}
l\left(\vec{x}\right)
\frac{\norm{\nabla\phi\left(\vec{x}\right)}}{\phi\left(\vec{x}\right)}
\end{align}
It consists of two factors, of which the first is the mean free path $l$, which is the inverse of the extinction coefficient and parameterizes the medium. If the mean free path is small and approaches zero, photons travel only very short distances in average before they encounter another interaction with the medium. In this case the medium is very dense. If the mean free path is large, the medium is very thin and photons travel long distances before they encounter another interaction with the medium. In case of vacuum, where no medium is present, photons travel unhindered and never interact. In this case the mean free path is infinite.
\begin{figure}[h]
\centering
\includegraphics[width=0.9\textwidth]{06_fld/figures/fig_transport_regimes_mfp.pdf}
\caption{Intuition behind paramterizing the transport regime according to mean free path (average distance between two consecutive scattering events). Small mean free paths imply higher medium density, while larger mean free paths imply lower density or vacuum in case of infinity.}
\label{fig:fld_transport_regimes_mfp}
\end{figure}



The second factor in equation~\ref{eq:flux_limit_factors} is the ratio between the length of the flux-vector (according to diffusion) and the zero moment. If the magnitude of the flux-vector is small, while the total power $\phi$ is large, the incoming radiance is distributed equally over solid angle, which means that light is coming uniformly from all directions. If the flux-vector magnitude is large in comparison to the total power, the energy is concentrating in certain directions. In the extreme case, when the total power is equal to the flux-vector magnitude, the light comes only from a single direction.
\begin{figure}[h]
\centering
\includegraphics[width=0.9\textwidth]{06_fld/figures/fig_transport_regimes_moment_ratio.pdf}
\caption{Intuition behind paramterizing the transport regime according to the ratio between the total amount of light arriving at a point and the directionality given by the magnitude of the flux vector (red in the figure). A small ratio results from small directionality in relation to large amounts of light arriving. This implies a uniform distribution of light coming from all directions, which is the case for diffusive transport. The ballistic transport regime is characterized by increased directionality in relation to the amount of light. In free streaming transport, the magnitude of the flux vector equals the amount of light arriving. This is the case when light comes from a single direction.}
\label{fig:fld_transport_regimes_mfp}
\end{figure}

The two factors in equation~\ref{eq:flux_limit_factors} parameterize the transport regime according to the medium and the distribution of incoming light over solid angle respectively. Combined, they are a local measure for the transport regime in general (as perceived by diffusion). If the medium is very dense (small mean free path) and light arrives equally distributed from all directions (small moment ratio), diffusive transport is dominating and the flux-limit is not violated. The diffusion approximation is a good approximation in these scenarios. If the medium is very thin (large mean free path) and the moment ratio approaches one, we have ballistic transport. In the case of vacuum, when the moment ration is exactly one, we have the free streaming limit or vacuum, where light travels infinte distances unhindered.

The $1/3$ factor in equation~\ref{eq:flux_limit_factors} is ignored and $R$ is introduced, a transport measure according to the diffusion approximation:
\begin{align}
\label{eq:fld_transport_measure_R}
R\left(\vec{x}\right)
= 
\frac{\norm{\nabla\phi\left(\vec{x}\right)}}{\sigma_t\left(\vec{x}\right)\phi\left(\vec{x}\right)}
\qquad
\qquad
\begin{array}{cc}
R\left(\vec{x}\right)\rightarrow 0 : \text{diffussive transport}\\
R\left(\vec{x}\right)\rightarrow 1 : \text{streaming transport}
\end{array}
\end{align}
In section~\ref{sec:moment_closure}, the diffusion approximation was derived by assuming isotropic distribution of radiance and therefore the same requirement in the fux-limit contraint is seen where a small moment ratio is required and implies that light is arriving equally from all directions.

The idea of flux-limited diffusion is to avoid violation of the fux-limit and consquently prevent unphysical transport. The key idea behind flux-limited diffusion is to not only assume isotropic distribution, but also allow streaming transport and use the measure $R$, to locally realize the right mix between diffusive and streaming transport. The next important building block is therefore the question of how the diffusion equation looks like in the presence of streaming transport. This is outlined in the next section. Section~\ref{sec:fld_vef} will then develop flux-limited diffusion as a combination with classical diffusion.

\section{Streaming Limit Approximation}
\label{sec:fld_streaming_limit_approximation}

In the previous section, we introduced the transport measure $R$, which contains the factor $\norm{\nabla\phi}/\phi$. This factor approaches one as transport becomes less diffusive and enters the streaming regime. In the case of $\norm{\nabla\phi}/\phi=1$, the length of the flux-vector matches the amount of total power. In this case, light is coming from a single direction. The radiance distribution of this configuration is given as
\begin{align}
\hat{L}\left(\omega\right)=\delta_{\Omega}\left(\omega,\vec{n}\right)
\end{align}
where $\delta_{\Omega}$ is the angular Dirac delta distribution into direction $\vec{n}$. Spatial dependency is omitted and not relevant for this discussion. The moments of this distribution are:
\begin{align}
\label{eq:zero_moment_Lhat}
\hat{L}_0\left(\omega\right)&=\int_{\Omega}{\delta_{\Omega}\left(\omega,\vec{n}\right)\ud\omega} = 1\\
\label{eq:first_moment_Lhat}
\hat{L}_1\left(\omega\right)&=\int_{\Omega}{\delta_{\Omega}\left(\omega,\vec{n}\right)\omega\ud\omega} = \vec{n}\\
\label{eq:second_moment_Lhat}
\hat{L}_2\left(\omega\right)&=\int_{\Omega}{\delta_{\Omega}\left(\omega,\vec{n}\right)\omega_i\omega_j\ud\omega} = \vec{n}_i\vec{n}_j
\end{align}
The zero moment expresses that all power given by $\phi$ comes from a single direction. The second equation shows, that the first moment of a delta distribution is identical to the vector which defines that distribution ($\vec{n}$ in our case). Also important is, that we can conclude that the length of the flux-vector equals the zero moment and its direction is $\vec{n}$ (all under the assumption of a delta radiance distribution):
\begin{align}
\label{eq:iso_delta_normE}
\vec{E} = \hat{L}_1\phi = \vec{n}\phi  &\implies \norm{\vec{E}} = \phi\\
&\implies \vec{E} \parallel \vec{n} \qquad \text{($\vec{E}$ and $\vec{n}$ are parallel)}
\end{align}
We see from the second moment equation, that the second moment is found by the outer product of the defining vector. Therefore, assuming a delta distribution results in the following Eddington tensor (see definition in equation~\ref{eq:eddington_tensor}):
\begin{align}
T_{ij} = \vec{n}_i\vec{n}_j
\label{eq:iso_delta_T}
\end{align}
Now we know the form of the Eddington tensor for a delta distribution of radiance in angular domain. In the end, this tensor is to be used to approximate the second moment of the radiance field $P$ in equation~\ref{eq:general_diffusion_equation} by $T\phi\approx P$. However, this way we introduce the direction $\vec{n}$ as another unknown. The next step in the derivation therefore is, to reformulate $T$ and eliminate of the unknown $\vec{n}$, which is possible under certain assumptions.
%An important consideration is the fact, that the vector $\vec{n}$ is scaled to infinite length under the integral sign.
% \emph{This means our delta distribution is represented by a single vector of infinite length with direction $\vec{n}$}.

Inserting the approximation $T\phi$ into the flux-vector definition (equation~\ref{eq:me_first_resolved_E}) and assuming an isotropic emission $Q$ gives:
\begin{align*}
\vec{E}&= -\frac{1}{\sigma_t'}\operatorname{div}\left (T\phi\right )
\end{align*}
The key assumption used is that the spatial variations of $T$ can be neglected. Fundamentaly, the approach therefore is to assume, that the radiance field $L$ can be seperated into a product of two functions. One depending on angle and another function depending on position. This allows us to express the divergence of the second moment as a matrix product with a gradient vector:
\begin{align}
\vec{E}&= -T\left(\frac{1}{\sigma_t'}\nabla\phi\right )
\label{eq:second_moment_iso2}
\end{align}
For the derivation we look at the normalized flux, which we find by scaling the flux-vector by $1/\phi$:
\begin{align}
\widehat{\vec{E}} = \frac{\vec{E}}{\phi}= -T\underbrace{\left(\frac{\nabla\phi}{\sigma_t'\phi}\right )}_{=\vec{R}}
\label{eq:second_moment_iso3}
\end{align}
We know from equation~(\ref{eq:first_moment_Lhat}), that $\vec{n}$ points into the same direction as the flux-vector $\vec{E}$. This means, that the result of the transformation $T$ will be a vector which is parallel to $\vec{n}$. Since we construct the Eddington tensor with $T_{ij}=\vec{n}_i\vec{n}_j$, we know that $\vec{n}$ is the only eigenvector of $T$, with an eigenvalue greater than zero. Therefore we can conclude, that (under the assumption of negligible spatial variation of $T$) the transformation $T$ is applying a scaling operation and the dimensionless gradient $\vec{R}$ is parallel to $E$. We therefore can express the product of tensor $T$ with $\vec{R}$ as a scaling operation and equation~(\ref{eq:second_moment_iso3}) then becomes:
\begin{align}
\widehat{\vec{E}} = -\lambda\left(\frac{\nabla\phi}{\sigma_t'\phi}\right )
\label{eq:second_moment_iso4}
\end{align}
So by making an assumption about the spatial variation of $T$, we were able to express our normalized flux-vector $\hat{L}_1$ with respect to the zero moment $\phi$ and its spatial derivatives. What remains to be done is to find an expression for the eigenvalue $\lambda$. Then we would have found an expression for $\vec{E}$ (by using $\vec{E}=\hat{L}_1\phi$), which does not depend on itsself in anyway and therefore can be used to substitute $\vec{E}$ into the zero moment equation.

Apparently, finding $\lambda$ is easy in the case of a delta distribution of radiance. In that case we know from equation~(\ref{eq:iso_delta_normE}), that $\norm{\vec{E}}=\phi$ and therefore $\norm{\vec{E}/\phi}=1$. This requires that:
\begin{align*}
%\norm[\big]{\vec{f}}=\norm[\Bigg]
\norm{\vec{E}}
=
\norm{-\lambda\left(\frac{\nabla\phi}{\sigma_t'\phi}\right )}=1
\implies
\lambda=\frac{\sigma_t'\phi}{\norm{\nabla\phi}}
\end{align*}
Note that lambda is under the vector norm, which would make the sign of lambda ambiguous (it could be positive or negative: in both cases the length of the normalized flux-vector would be one). However, as mentioned earlier, the way we define $T=\vec{n}_i\vec{n}_j$ allows us to conclude that $\lambda$ is the only Eigenvalue of $T$ and must greater one. This gives our final expression for the flux-vector $\vec{E}$:
\begin{align*}
\vec{E}=\widehat{\vec{E}}\phi= -\lambda\left(\frac{\nabla\phi}{\sigma_t'\phi}\right )\phi= -\frac{\nabla\phi}{\norm{\nabla\phi}}\phi
\end{align*}
which states that the flux-vector is the unit vector pointing into the direction of the gradient of $\phi$ scaled by $\phi$ itsself. Inserting this into the collapsed $P_1$-equation (equation~\ref{eq:general_diffusion_equation}) gives:
\begin{align}
\label{eq:iso_delta_advection_equation}
\nabla\cdot\left(\frac{\nabla\phi}{\norm{\nabla\phi}}\phi\right) &= \sigma_a\phi - Q_0
\end{align}
We see that in case of a delta distribution of radiance in angular domain and with the assumption of negligible spatial variation of $T$, the moment equation turns into an advection equation, where the zero moment quantity $\phi$ is moved around by its normalized gradient.

We have seen in this section, that the streaming transport is best represented by advection in case of the two term expansion of the radiative transfer equations. The core idea behind flux-limited diffusion is, to mix diffusive transport and advective transport, depending on local properties at $\vec{x}$. This is developed in the next section.
\section{The Variable Eddington Factor}
\label{sec:fld_vef}

In the two previous sections, diffusion theories have been derived for the two extreme limits of transport. That is, purely isotropic radiance which happens in the limit of thick, highly scattering media or a delta distribution which happens in the limit of very transparent, low order scattering media. This section will introduce the Variable Eddington Factor formalism, which is a theoretical framework, in which flux-limited diffusion is embedded. While diffusive transport has been related to diffussive transport, the streaming transport has been shown to relate to advection and the Variable Eddington Factor formalism is a theory which allows to realize and mix both types of transport within a domain.

\TD{show advection and diffusion transport}

Classical diffusion assumed an isotropic distribution of radiance for the second moment, while pure streaming transport assumed a Delta distribution. The Variable Eddington Factor theory assumes a radiance distribution, which is rotationally symmetric around a dominant vector $\vec{n}$. This assumtion allows to to derive a form for $T$, which allows to represent isotropic distribution, as well as a pure Delta distribution and forms inbetween, which are rotationally symmetric around a dominant direction $\vec{x}$.

\TD{show image with isotropic distribution, streaming limit distribution and inbetween forms}

We start deriving the Variable Eddington Factor form of $T$ by considering a principal direction of transport, given by the vector $\vec{n}$ of unit length. Now an important assumption, which is being made, is that the radiance distribution will be radially symmetric about this direction. This means, that the value of $\hat{L}$ will be invariant to a rotation about axis $\vec{n}$. It follows, that its first and second moment $\hat{L}_1$ and $\hat{L}_2$ will also be invariant to rotation about $\vec{n}$. If we approximate $\hat{L}_2$ using the Eddington tensor $T$ with $\hat{L}_2\approx T\phi$, we can conclude that $\vec{n}$ will be an eigenvector of $T$, with some eigenvalue $\chi$:
\begin{align*}
T\vec{n} = \chi\vec{n}
\end{align*}
\TD{explain/give reference why eigenvectors need to sum up to one}
The plane perpendicular to $\vec{n}$ is an eigenspace of $T$. By requiring that all eigenvalues sum up to one, we can express the eigenvalues of the two eigenvectors spanning that plane by distributing the remaining eigenvalue $1-\chi$ evenly between both:
\begin{align}
\frac{1}{2}\left(1-\chi\right)\mathbf{I}
\label{eq:iso_var_T_isoterm}
\end{align}
From the free streaming limit case (section~\ref{sec:fld_streaming_limit_approximation}), we know that $\chi=1$ when the radiance distribution becomes a delta distribution (see equation~\ref{eq:zero_moment_Lhat} and~\ref{eq:first_moment_Lhat}). In that case, the eigenvalues associated with the eigenvectors perpendicular to $\vec{n}$ become zero and the term above vanishes.

Tensor diagonalization allows us to explicitly add the eigenvector $\vec{n}$ by adding the matrix $\vec{n}_i\vec{n}_j$ (with $\vec{n}$ being of unit length). The scaling of its coefficients is found by considering that the term in equation~\ref{eq:iso_var_T_isoterm} introduces three eigenvectors. The sum of their eigenvalues will be $3/2(1-\chi)$. Since eigenvalues of our final tensor $T$ need to add up to one, we know that the eigenvalue associated with the matrix $\vec{n}_i\vec{n}_j$ will be $\left(1- \frac{3}{2}\left(1 - \chi\right)\right)$. This results in the following form for $T$:
\begin{align}
T &= \frac{1}{2}\left(1-\chi\right)\mathbf{I} + \left(1- \frac{3}{2}\left(1 - \chi\right)\right) \vec{n}\otimes\vec{n}
\nonumber
\\
&= \frac{1}{2}\left(1-\chi\right)\mathbf{I} + \frac{1}{2}\left(3\chi-1\right) \vec{n}\otimes\vec{n}
\label{eq:iso_var_T}
\end{align}
This form of $T$ is called the Variable Eddington Tensor (VET). It can be understood as an \emph{interpolation} between an isotropic distribution tensor ($1/3\mathbf{I}$) and a delta distribution tensor ($\vec{n}_i\vec{n}_j$). The variable $1/3 \le \chi \le 1$ is the interpolation variable and it is called the Eddington factor (VEF). Theories, which respect this structure of the eddington tensor are referred to as theories under the variable eddington factor formalism (VEF-formalism).

If $\chi=1/3$, then equation~\ref{eq:iso_var_T} will result in the isotropic distribution tensor, which is the base assumption for classical diffusion:
\begin{align}
\frac{1}{2}\frac{2}{3}\mathbf{I} + \frac{1}{2}\left(\frac{3}{3}-1\right) \vec{n}\otimes\vec{n}
=\frac{1}{3}\mathbf{I} + 0\vec{n}\otimes\vec{n} = \frac{1}{3}\mathbf{I}
\end{align}
For $\chi=1$, equation~\ref{eq:iso_var_T} produces the Eddington tensor, which was derived for the pure streaming limit distribution, where all light comes from a singular direction:
\begin{align}
\frac{1}{2}0\mathbf{I} + \vec{n}\otimes\vec{n}
= \vec{n}\otimes\vec{n}
\end{align}
The Eddington factor $\chi$ can also be interpreted as a measure of anisotropy of the radiance distribution $\widehat{L}$ with respect to direction $\vec{n}$. It can be expressed in terms of the radiance distribution by the squared mean cosine (given by Levermore~\cite{Levermore84}):
\begin{align}
\chi &= \int_{S^2}{ \left(\omega\cdot\vec{n}\right)^2\hat{L}\left(\vec{x}, \omega\right)\ud\omega}
\label{eq:iso_var_chi}
\end{align}
With the variable Eddington factor formalism, the only thing we now need to find is a function for the interpolation variable $\chi$ (the Eddington factor). This function should have $1/3$ and $1$ as its limits. With such an interpolation function, the limit transport cases (diffusion and free streaming) will be a subset of any theory, which adheres to the variable Eddington factor formalism.

The Eddington factor formalism provides a framework, by setting up the Eddington factor and the limits of its parameterization, $\chi$. The specific expression for this factor is not defined and it is clear that there are many options for the function $\chi$, which respect the given function limits. This is why there is a rich variety of theories, which all propose their own version of that interpolation function. Some of them are ad-hoc schemes which are derived from heuristics, while others have a clear connection to transport theory or are derived from entropy theory.
\TD{insert references to different theories}
The general strategy for all these different variable Eddington factor theories is:
\begin{enumerate}
\item Find a model or theory, from which a certain radially symmetric form of the radiance distribution $\hat{L}$ about the normalized flux-vector$\vec{E}/\phi$ can be found or justified.
\item Then derive an expression for $\chi(\vec{E}/\phi)$, from the model for $\hat{L}$, which can be used to construct $T$. Further assumptions are applied, in order to be able to express $\vec{E}$ in terms of $\nabla\phi$. This is required in order to not have $T$ depend on the flux-vector directly. This is because it still needs to be possible to resolve equation~\ref{eq:me_first_resolved_E} for the flux-vector.
\end{enumerate}

In this thesis, we implemented and present results for the most popular theories, but discuss in detail only flux-limited diffusion from Levermore et. al~\cite{Levermore81}, since it is the most popular and also has the strongest connection to transport theory. Discussing all other theories is beyond the scope of this thesis and also not really necessary, as we will see in section~\ref{sec:fld_results}, that the particular choice of theory is not of significant importance for applications in computer graphics.
\section{Flux-limiters}
\label{sec:fld_vef_factors}

Levermore et. al~\cite{Levermore81} construct their theory by starting from the time-dependent form of the radiative transfer equation. They further assume a constant phase function $f_p=1/(4\pi)$:
\begin{align}
\label{eq:iso_var_fld_rte}
\frac{1}{c}\frac{\partial (\phi\hat{L})}{\partial t} + \left(\omega\cdot\nabla\right)(\phi\hat{L})&=-\sigma_t\phi\hat{L} + \frac{1}{4\pi}\sigma_s\phi + Q
\end{align}
After applying the moment expansion, we have for the first moment equations:
\begin{align}
\label{eq:iso_var_fld_rte_zero}
\frac{1}{c}\frac{\partial \phi}{\partial t} + \nabla\vec{E} &= -\sigma_a\phi + Q_0
\end{align}
As a next step, equation~\ref{eq:iso_var_fld_rte_zero} is resolved for $\partial \phi/\partial t$ on the left hand side and used to substitute $\partial\phi/\partial t$ in equation~\ref{eq:iso_var_fld_rte} (after applying the product rule to the time derivative term). By further assuming the absence of self-emission ($Q=0$) and that the space and time derivatives of the radiance distribution $\hat{L}$ can be neglected, we arrive at:
\begin{align}
\label{eq:iso_var_fld_eliminated_time}
\left( \left(\omega\cdot\nabla\right)\phi -\vec{f}\cdot\nabla\phi + \sigma_t'\phi\right)\hat{L} = \frac{1}{4\pi}\sigma_t'\phi
\end{align}
Solving Eq.~\ref{eq:iso_var_fld_eliminated_time} for $\hat{L}$ gives:
\begin{align}
\label{eq:iso_var_fld_Lhat}
\hat{L} = \frac{1}{4\pi}\frac{1}{1+\widehat{\vec{E}}\cdot\vec{R}-\omega\cdot\vec{R}}
\end{align}
with
\begin{align}
\label{eq:iso_var_fld_R}
\vec{R} = -\frac{\nabla\phi}{\sigma_t'\phi}
\end{align}
We use the definition of the flux-vector in equation~\ref{eq:second_moment_iso3}, which was derived in section~\ref{sec:fld_streaming_limit_approximation}:
\begin{align}
\widehat{\vec{E}} = \frac{\vec{E}}{\phi}= T\vec{R}
\label{eq:iso_var_fld_normalized_flux}
\end{align}

Following the same argument about the radial symmetry of $\hat{L}$ and its relation to the flux-vector and eigenvectors of $T$, we replace $T$ with a proportionality function $\lambda(R)$, where $R=\norm{\vec{R}}$:
\begin{align}
\widehat{\vec{E}} = \lambda(R)\vec{R}
\label{eq:iso_var_fld_relation_normalized_flux_R}
\end{align}

Using this in Eq.~\ref{eq:iso_var_fld_Lhat} gives:
\begin{align}
\label{eq:iso_var_fld_Lhat_R}
\hat{L} = \frac{1}{4\pi}\frac{1}{1+\lambda(R)R^2-\omega\cdot\vec{R}}
\end{align}

By enforcing that $\hat{L}$ integrates to one over the solid angle of the unit sphere, we can derive the following expression for $\lambda(R)$:
\begin{align}
\label{eq:iso_var_fld_lambdaR}
\lambda(R) = \frac{1}{R}\left(\mathbf{\operatorname{coth}}R - \frac{1}{R}\right)
\end{align}

Using this result and equation~\ref{eq:iso_var_fld_relation_normalized_flux_R}, gives the following expression for the flux vector:
\begin{align}
\label{eq:iso_var_fld_fluxvector}
\vec{E} = \widehat{\vec{E}}\phi=-\frac{1}{\sigma_t'}\lambda(R)\nabla\phi
\end{align}

Inserting this into the zero moment equation (equation~\ref{eq:me_zero}) gives the flux-limited diffusion equation, a diffusion-type equation of the following form:
\begin{align}
\label{eq:iso_var_fld_final}
\nabla\left( \underbrace{-\frac{1}{\sigma_t'}\lambda(R)}_{D_F}\nabla\phi\right) &= -\sigma_a\phi + Q_0
\end{align}

The flux-limited diffusion coefficient $D_F$ is non-linear and therefore turns the zero moment equation into a non-linear diffusion equation. $\lambda(R)$ is called the flux-limiter. In the diffusion limit, $R$ approaches zero and the flux-limiter approaches $\lambda(R)=1/3$, which will turn equation~\ref{eq:iso_var_fld_final} into the classical diffusion equation for isotropic media.
\begin{align}
\lim_{R\rightarrow 0 }D_F =-\frac{1}{3\sigma_t'}
\implies
\nabla
\left(
-\frac{1}{3\sigma_t'\left(\vec{x}\right)}
\nabla \phi\left(\vec{x}\right)
\right)
=
-\phi(\vec{x})\sigma_a(\vec{x})
+Q_0\left(\vec{x}\right)
\end{align}
In the transport limit, $R$ will approach infinity and the diffusion coefficient will cause the equation to become an advection equation as seen in the delta radiance distribution case (equation~\ref{eq:iso_delta_advection_equation}):
\begin{align}
\lim_{R\rightarrow\infty }D_F =-\frac{\phi}{\norm{\nabla\phi}}
\implies
\nabla
\left(
-\phi\frac{\nabla\phi\left(\vec{x}\right)}{\norm{\nabla\phi}}
\right)
=
-\phi(\vec{x})\sigma_a(\vec{x})
+Q_0\left(\vec{x}\right)
\end{align}
It can be seen that the flux-limited diffusion coefficient will normalize the fluence gradient $\nabla\phi$ to unit length and scale it with the total power $\phi$. Flux-limited diffusion not only suppresses the flux in the free streaming transport regime, it also saturates it at the appropriate value to ensure correct free propagation (at the level of the approximation).

The flux-limiter introduced by Levermore et al.~\cite{Levermore81} was shown to relate to the variable Eddington factor formalism in a seperate study (\cite{Whalen82, Levermore84}) to be:
\begin{align}
\label{eq:iso_var_fld_vef}
\chi = \lambda(R) + \lambda(R)^2R^2
\end{align}
As mentioned in the previous section, the theory sets up the limits, which flux-limiters have to respect. This allows different models and theories to find and justify particular choices of flux-limiters. Table~\ref{tbl:flux-limiters} presents the most prominent flux-limiters.
\begin{table}[h]
\center
\caption{Various prominent flux-limiters which all respect the Eddington factor limits and represent different flux-limited diffusion theories.}
\begin{tabular}{ l l }
\hline\hline
 Flux-limiter & $\lambda\left(R\right)$ \\ 
\hline
 sum~\cite{Bowers82} & $(3+R)^{-1}$ \\
 max~\cite{Bowers82} & $\mbox{max}(3, R)^{-1}$ \\
 Kershaw~\cite{Kershaw76} & $2(3+\sqrt{9 + 4R^2}\,)^{-1}$ \\
 Larsen-$n$~\cite{Larsen74} & $(3^n + R^n)^{-\frac{1}{n}}$ \\
 Levermore-Pomraning~\cite{Levermore81} & $\frac{1}{R} \left(\coth(R)-\frac{1}{R}\right)$    
\end{tabular}
\label{tbl:flux-limiters}
\end{table}
\missingfigure{flux-limiters graphs}

In this section we derived a non-linear form of the diffusion equation which respects the flux-limit constraint. The following section will present a novel solver, which solves flux-limited diffusion on a finite difference grid over a given domain.

\section{Non-Linear Gauss-Seidel Solver with Successive Overrelaxation}
\label{sec:fld_solver}

We now introduce a new method for solving the modified diffusion equation, provided by flux-limited diffusion theory. Most prominent with the flux-limited diffusion equation is the non-linear diffusion coefficient. Unfortunately, this non-linearity prevents the application of any components of the solver-framework that was developed in chapter~\ref{sec:pnmethod} and application of the multigrid solver developed in chapter~\ref{sec:diffusion_approximation}, as both are exclusively geared towards coupled systems of linear partial differential equations.

The equation, which has to be solved is the flux-limited diffusion equation (equation~\ref{eq:iso_var_fld_final}). The flux-limiter $\lambda$ by Levermore et al.~\cite{Levermore81} will be used (equation~\ref{eq:iso_var_fld_lambdaR}), which in term depends on the transport measure $R$ (equation~\ref{eq:fld_transport_measure_R}). In summary, the solver developed in this section tries to find a solution for $\phi$, which satisfies the following set of equations:
\begin{align}
\nabla\left(D_F\left(\vec{x}\right)\nabla\phi\right) &= -\sigma_a\phi + Q_0
\quad \text{with}\quad
D_F\left(R\left(\vec{x}\right)\right) = -\frac{1}{\sigma_t'}\lambda\left(R\left(\vec{x}\right)\right)
\label{eq:fld_solver_diffusion_equation}
\\
\lambda(R\left(\vec{x}\right)) &= \frac{1}{R\left(\vec{x}\right)}\left(\mathbf{\operatorname{coth}}R\left(\vec{x}\right) - \frac{1}{R\left(\vec{x}\right)}\right)
\label{eq:fld_solver_flux_limiter}
\\
R\left(\vec{x}\right) &= \frac{\norm{\nabla\phi\left(\vec{x}\right)}}{\sigma_t\left(\vec{x}\right)\phi\left(\vec{x}\right)}
\label{eq:fld_solver_transport_measure_R}
\end{align}
This is a diffusion equation with a non-linear diffusion coefficient $D_F$ depending on the flux-limiter $\lambda$. The non-linearity arises from the fact that the flux-limiter depends on the solution, which after substitution will create non-linear terms in the partial differential equation.

\subsubsection*{Discretization}

Discretizing equation~\ref{eq:fld_solver_diffusion_equation} using a finite difference grid is straightforward and results in the following partial differential equation:
\begin{align}
\frac{1}{h_x^2}D_{i-\frac{1}{2}}\phi_{i-1}
+\frac{1}{h_x^2}D_{i+\frac{1}{2}}\phi_{i+1}
+\frac{1}{h_y^2}D_{j-\frac{1}{2}}\phi_{j-1}
\\
+\frac{1}{h_y^2}D_{j+\frac{1}{2}}\phi_{j+1}
+\frac{1}{h_z^2}D_{k-\frac{1}{2}}\phi_{k-1}
+\frac{1}{h_z^2}D_{k+\frac{1}{2}}\phi_{k+1}
\\
-\left(
\frac{1}{h_x^2}D_{i-\frac{1}{2}}+\frac{1}{h_x^2}D_{i+\frac{1}{2}}
+\frac{1}{h_y^2}D_{j-\frac{1}{2}}+\frac{1}{h_y^2}D_{j+\frac{1}{2}}
+\frac{1}{h_z^2}D_{k-\frac{1}{2}}+\frac{1}{h_z^2}D_{k+\frac{1}{2}}
\right)
\phi_{ijk}
\\
= -\sigma_{a, ijk}\phi_{ijk} + q_{ijk}
\label{eq:fld_solver_discretized_diffusion_equation}
\end{align}
The subscript to the diffusion coefficient $D_F$ has been omitted for readability. Since it is defined at voxel centers, its off-center values are found by interpolation. For example,
\begin{align}
D_{i+\frac{1}{2}} = \frac{1}{2}\left( D_{i}+D_{i+1}\right)\ .
\label{eq:fld_solver_D_interpolated}
\end{align}
The flux-limiter in equation~\ref{eq:fld_solver_flux_limiter} is simply discretized by using the discretized transport measure $R_{ijk}$:
\begin{align}
\lambda(R_{ijk}) &= \frac{1}{R_{ijk}}\left(\mathbf{\operatorname{coth}}R_{ijk} - \frac{1}{R_{ijk}}\right)
\label{eq:fld_solver_discrete_flux_limiter}
\end{align}
Note, that the transport measure $R$ will cause a division by zero if it becomes zero. This problem will be addressed by the way the transport measure $R$ in equation~\ref{eq:fld_solver_transport_measure_R} is discretized:
\begin{align}
R_{ijk} = \frac{\operatorname{max}\left(\norm{\nabla\phi_{ijk}},\epsilon\right)}{\operatorname{max}\left(\sigma_{t,ijk}\phi_{ijk},\epsilon\right)}
\quad
\text{with}
\quad
\nabla\phi_{ijk} = \frac{1}{2}
\left(
\begin{array}{c}
\frac{1}{h_x}\phi_{i+1}-\frac{1}{h_x}\phi_{i-1} \\
\frac{1}{h_y}\phi_{j+1}-\frac{1}{h_y}\phi_{j-1} \\
\frac{1}{h_z}\phi_{k+1}-\frac{1}{h_z}\phi_{k-1} \\
\end{array}
\right)
\label{eq:fld_solver_discrete_R}
\end{align}
The introduction of a minimum threshold $\epsilon$ in the nominator and denominator will prevent $R$ from becoming zero or breaking down due to division by zero. The minimum threshold $\sigma_{min}$ of the extinction coefficient will be applied on top.


\subsubsection*{Non-linear Gauss-Seidel Solver}

The solver follows an iterative segregated approach for solving non-linear partial differential equations (Mazumder~\cite{Mazumder2015}). The idea is to update the diffusion coefficient from the current solution $\phi_{ijk}$ (initialized by an initial guess) and then update the solution using the updated diffusion coefficient values $D_{ijk}$ and repeat this update procedure until convergence to a final solution.

The iteration method used is the Gauss-Seidel fixpoint iteration scheme. The non-linearity is integrated by updating the diffusion coefficient in place during iteration over all voxels for a single Gauss-Seidel step.

The Gauss-Seidel update step is found by isolating $\phi_{ijk}$ in the discretized diffusion equation~\ref{eq:fld_solver_discretized_diffusion_equation}:
\begin{equation}
\label{eq:fld_solver_gs_phi_update}
\resizebox{1.0\hsize}{!}{$
\phi_{ijk} =
\\
\frac
{
\frac{1}{h_x^2}D_{i-\frac{1}{2}}\phi_{i-1}
+\frac{1}{h_x^2}D_{i+\frac{1}{2}}\phi_{i+1}
+\frac{1}{h_y^2}D_{j-\frac{1}{2}}\phi_{j-1}
+\frac{1}{h_y^2}D_{j+\frac{1}{2}}\phi_{j+1}
+\frac{1}{h_z^2}D_{k-\frac{1}{2}}\phi_{k-1}
+\frac{1}{h_z^2}D_{k+\frac{1}{2}}\phi_{k+1}
+q_{ijk}
}
{
-\left(
\frac{1}{h_x^2}D_{i-\frac{1}{2}}+\frac{1}{h_x^2}D_{i+\frac{1}{2}}
+\frac{1}{h_y^2}D_{j-\frac{1}{2}}+\frac{1}{h_y^2}D_{j+\frac{1}{2}}
+\frac{1}{h_z^2}D_{k-\frac{1}{2}}+\frac{1}{h_z^2}D_{k+\frac{1}{2}}
-\sigma_{a, ijk}
\right)
}
$}
\end{equation}
Then, the discrete transport measure $R_{jk}$ is updated, using the newly retrieved value for $\phi_{ijk}$. With this, the new diffusion coefficient is computed:
\begin{align}
D_{ijk} = \frac{\lambda\left(R_{ijk}\right)}{\sigma_t'}
\label{eq:fld_solver_discrete_D}
\end{align}
This updating step is executed for every voxel of the finite difference grid and values are updated in place.

The values $\phi_{ijk}$ and $D_{ijk}$ are initialized to small tolerances $\phi_{ijk} = \epsilon \bar{j}\Delta l$ and $D_{ijk} = \epsilon \Delta l$ for all voxels. Initially, these grid values do not satisfy equation~\ref{eq:fld_solver_gs_phi_update} and equation~\ref{eq:fld_solver_discrete_D}, but over several iterations, they converge to a consistent solution of both equations over the whole grid. 

Boundary conditions are accounted for by updating the values for $\phi_{ijk}$ and $D_{ijk}$ at the boundary voxels. For Dirichlet boundary conditions the values are set directly. For Neumann boundary conditions the boundary voxels are set according to the next inner voxel (section~\ref{sec:pn_bc}). This update is done at the beginning of each Gauss-Seidel iteration.

A criteria is set to stop the Gauss-Seidel iterations. For real-time applications a specific user-decided number of iterations might be a reasonable choice. Here, convergence to machine precision may be compromised to retain interactivity. Another option is to compute the root mean square of the residual and stop the Gauss-Seidel algorithm as soon as this falls below a user-defined threshold.

To improve the rate of convergence, successive over-relaxation (SOR) is used (Hadjidimos~\cite{Hadjidimos00}). Defining the over-relaxation parameter $\vec{\omega}$, where $0<\vec{\omega}<2$, the update of $\phi_{ijk}$ at each stencil is modified to:
\begin{equation}
\label{eq:fld_solver_sor_update}
\phi_{ijk} \leftarrow \vec{\omega} \,\phi'_{ijk} + (1-\vec{\omega})\phi_{ijk} 
\end{equation}
where $\phi'_{ijk}$ is the updated value from equation~\ref{eq:fld_solver_gs_phi_update}.

To further speed up the computation time a Red-Black Gauss-Seidel update process is used. This allows to update half of the voxels in parallel (Olshanskii et al.~\cite{Olshanskii14}). The voxels are separated into two disjunct groups according to a checkerboard pattern (hence the name Red-Black). Voxels within a group can be updated in parallel, because the stencil only requires information from the neighboring voxels in each dimension and those always belong to the other voxel group. Firstly, all voxels of one group are processed in parallel and their $\phi_{ijk}$ and $D_{ijk}$ values are updated. Then all the voxels of the other group are updated in parallel. The two passes constitute one iteration. The full algorithm is outlined in listing~\ref{algorithm:fld_solver}.



\begin{algorithm}[!ht]
\caption{Non-linear Gauss-Seidel SOR FLD solver}
\begin{algorithmic}
\State Initialize voxel grids $\phi$, $D$
\Repeat
    \State Update boundary voxels
    \ParFor{all red non-boundary voxels $ijk$}
    \State Compute $R_{ijk}$ (equation~\ref{eq:fld_solver_discrete_R})
    \State Compute $\lambda\left(R_{ijk}\right)$ (equation~\ref{eq:fld_solver_discrete_flux_limiter})
    \State Compute and update $D_{ijk}$ (equation~\ref{eq:fld_solver_discrete_D})
    \State Compute $D_{i\pm\frac{1}{2}}$, $D_{j\pm\frac{1}{2}}$ and $D_{k\pm\frac{1}{2}}$ (according to equation~\ref{eq:fld_solver_D_interpolated})
    \State Compute $\phi_{ijk}'$ (equation~\ref{eq:fld_solver_gs_phi_update})
    \State Compute and update $\phi_{ijk}$ using equation~\ref{eq:fld_solver_sor_update}
    \EndParFor
    \ParFor{all black non-boundary voxels $ijk$}
    \State Same as above but just for the black voxels
    \EndParFor
    \State If required, compute root mean square of residual\;
\Until{convergence criteria is met}
\end{algorithmic}
\label{algorithm:fld_solver}
\end{algorithm}

\section{Results}
\label{sec:fld_results}

With the solver introduced in the previous section it can now be run for comparison on the problems, which were used for the $P_N$-method in chapter~\ref{sec:pnmethod} and the diffusion approximation in chapter~\ref{sec:diffusion_approximation}.

The rendering integration aspects to be considered are identical to the diffusion approximation (section~\ref{sec:da_results}). As with the diffusion approximation, flux-limited diffusion will give a solution for the zero moment $\phi$ from which the second moment can be recovered using equation~\ref{eq:diffusion_ficks_law}. Both moments can be used to reconstruct the truncated spherical harmonics expansion of the radiance field by using equation~\ref{eq:moment_expansion_L}. The difference to the diffusion approximation solution is that the fluence should be more accurate as the flux-limit constraint had been accounted for.

\subsection{Point Source Problem}
\label{sec:pn_results_pointsource}

At first it shall be turned towards the point source problem in order to validate the results from the flux-limited diffusion solver and assess its accuracy by comparing against an analytical ground truth solution.
\begin{figure}[h]
\centering
\begin{subfigure}{0.49\columnwidth}
%\includegraphics[width=\columnwidth]{images/checkerboard2d_p1_neumann_staggered_starmap.png}
\missingfigure{pointsource plots FLD DA maybe a seperate plot for plotting R?}
\caption{TODO}
\label{fig:fld_results_pointsource_1}
\end{subfigure}%
\hspace{0.01\columnwidth}
\begin{subfigure}{0.49\columnwidth}
%\includegraphics[width=\columnwidth]{images/checkerboard2d_p1_neumann_staggered.png}
\missingfigure{R along radius from point source}
\caption{TODO}
\label{fig:fld_results_pointsource_2}
\end{subfigure}%
\caption{TODO}
\label{fig:fld_results_pointsource}
\end{figure}

%\ref{fig:fld_results_pointsource_1}\subref{fig:fld_results_pointsource_1}

As seen in figure~\ref{fig:fld_results_pointsource_1}, the result from flux-limited diffusion agree well with the ground truth and diffusion approximation results. Like with the diffusion approximation, a perfect match to the ground truth is not to be expected as flux-limited diffusion is still an approximation based on the spherical harmonics expansion of the radiative transfer equation truncated after the first moment. However, very close to the point source the flux-limited diffusion is more accurate than classical diffusion. This is explained by the fact that the fluence gradient becomes arbitrary large as the point light is approached, due to the geometry term (figure~\ref{fig:fld_results_pointsource_2}). This causes the transport measure $R$ to likewise become arbitrary large, which shows that pure streaming transport will dominate arbitrarily close to the point source.

In comparison to the results from $P_N$-method and $P_5$ in particular, the $P_N$-method with higher truncation order is more accurate than flux-limited diffusion. This an important observation since it means that increasing the truncation order will alleviate the problems caused by not accounting for the flux-limit constraint. However, flux-limited diffusion has a signficicant advantage over $P_N$ due to its computational and memory efficiency.

\subsection{Checkerboard Problem}
\label{sec:pn_results_checkerboard}

For the checkerboard problem, the hand-crafted stencil for flux-limited diffusion had been modified to ignore all derivative terms in the z-dimension. The results in figure~\ref{fig:fld_results_checkerboard_1} show how...
\TD{discuss fld solution on checkerboard}
\begin{figure}[h]
\centering
\begin{subfigure}{0.49\columnwidth}
%\includegraphics[width=\columnwidth]{images/checkerboard2d_p1_neumann_staggered_starmap.png}
\missingfigure{checkerboard plots FLD}
\caption{TODO}
\label{fig:fld_results_checkerboard_1}
\end{subfigure}%
\hspace{0.01\columnwidth}
\begin{subfigure}{0.49\columnwidth}
%\includegraphics[width=\columnwidth]{images/checkerboard2d_p1_neumann_staggered.png}
\missingfigure{checkerboard plots CDA for comparison}
\caption{TODO}
\label{fig:fld_results_checkerboard_2}
\end{subfigure}%
\caption{TODO}
\label{fig:fld_results_checkerboard}
\end{figure}

\subsection{Procedural Cloud}
\label{sec:pn_results_clouds}

Finally the flux-limited diffusion is run on the procedural cloud dataset described in section~\ref{sec:pn_results_clouds} to assess how it performs in a more practical setting. As seen in figure~\ref{fig:fld_results_nebulae}, flux-limited diffusion does a significantly better job at conserving energy than classical diffusion or $P_5$. This is explained by the fact that the dataset contains very strong density gradients and close-to vacuum regions in which diffusive transport is not able to capture light transport well. While $P_5$ produces better results than classical diffusion, it is somehow surprising to see how much better flux-limited diffusion is able to capture the directly lit regions of the cloud.

Further flux-limited diffusion well captures the indirectly illuminated parts at the bottom of the dataset. Here flux-limited diffusion also offers a signficiant improvement over classical diffusion approximation. Upon close inspection and comparison with $P_5$, it can be seen that the flux-limited diffusion solution appears very flat while the $P_5$ solution exhibits finer variations which closer resemble the path-traced resuls. This is explained by the fact that the angular domain is better resolved with $P_5$. The energy conserving nature of flux-limited diffusion still produced a result, which much closer resembles the path-traced result than $P_5$.

\begin{figure}[h]
\centering
\begin{subfigure}{0.31\columnwidth}
%\includegraphics[width=\columnwidth]{images/checkerboard2d_p1_neumann_staggered_starmap.png}
\missingfigure{checkerboard plots FLD}
\caption{TODO}
\label{fig:fld_results_nebulae_1}
\end{subfigure}
\hspace{0.01\columnwidth}
\begin{subfigure}{0.31\columnwidth}
%\includegraphics[width=\columnwidth]{images/checkerboard2d_p1_neumann_staggered.png}
\missingfigure{nebulae plots CDA for comparison}
\caption{TODO}
\label{fig:fld_results_nebulae_2}
\end{subfigure}
\hspace{0.01\columnwidth}
\begin{subfigure}{0.31\columnwidth}
%\includegraphics[width=\columnwidth]{images/checkerboard2d_p1_neumann_staggered.png}
\missingfigure{nebulae plots P5 for comparison}
\caption{TODO}
\label{fig:fld_results_nebulae_2}
\end{subfigure}%
\caption{TODO}
\label{fig:fld_results_nebulae}
\end{figure}

In terms of performance chacteristics it can be expected that flux-limited diffusion requires more computational effort than the diffusion approximation, due to the non-linear nature of flux-limited diffusion coefficient, which requires to be updated for every voxel in each Gauss-Seidel iteration. This impact appears moderate when compared to the visual improvement it brings. When compared to $P_N$ it can be concluded that flux-limited diffusion offers a much better result at a significantly lower computational cost.
\begin{table}[!h]
	\centering
	\caption[TODO]{Performance characteristics of flux-limited diffusion for the procedural cloud dataset.}
	\label{tab:results_cloud}
	% \flushleft
	\begin{tabular}{l r}
    \hline
	\textbf{N}
    & 1
    \\
    \hline
    Number of rows/columns in $A$
    & ?
    \\
    Size of linear system (in MB)
    & ?
    \\
    Solve time (in min)
    & ?
	\end{tabular}
\end{table}
%\TD{complete table maybe table showing the performance characteristic for all three datasets}
%\TD{mention how the performance of FLD allows aplication of transfer function and solve for different color channels}
%\TD{mention publication}
%\TD{mention application in elementacular}
\section{Discussion}
\label{sec:fld_discussion}



\TD{write discussion}




