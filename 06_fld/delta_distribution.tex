\section{Streaming Limit Approximation}
\label{sec:fld_streaming_limit_approximation}

In the previous section, the transport measure $R$ was introduced, which contains the factor $\norm{\nabla\phi}/\phi$. This factor approaches one as transport becomes less diffusive and enters the streaming regime. In the case of $\norm{\nabla\phi}/\phi=1$, the length of the flux-vector matches the amount of total power. In this case, light is coming from a single direction. The radiance distribution of this configuration is given as
\begin{align}
\hat{L}\left(\vec{\omega}\right)=\delta_{\Omega}\left(\vec{\omega},\vec{n}\right).
\end{align}
where $\delta_{\Omega}$ is the angular Dirac delta distribution into direction $\vec{n}$. The spatial dependency is omitted and not relevant for this discussion. The moments of this distribution are:
\begin{align}
\label{eq:zero_moment_Lhat}
\hat{L}_0\left(\vec{\omega}\right)&=\int_{\Omega}{\delta_{\Omega}\left(\vec{\omega},\vec{n}\right)\ud\vec{\omega}} = 1\\
\label{eq:first_moment_Lhat}
\hat{L}_1\left(\vec{\omega}\right)&=\int_{\Omega}{\delta_{\Omega}\left(\vec{\omega},\vec{n}\right)\vec{\omega}\ud\vec{\omega}} = \vec{n}\\
\label{eq:second_moment_Lhat}
\hat{L}_2\left(\vec{\omega}\right)&=\int_{\Omega}{\delta_{\Omega}\left(\vec{\omega},\vec{n}\right)\vec{\omega}_i\vec{\omega}_j\ud\vec{\omega}} = \vec{n}_i\vec{n}_j
\end{align}
The zero moment expresses that all power given by $\phi$ comes from a single direction. The second equation shows, that the first moment of a delta distribution is identical to the vector which defines that distribution ($\vec{n}$ in our case). It shall be concluded that the length of the flux-vector equals the zero moment and its direction is $\vec{n}$ (all under the assumption of a delta radiance distribution):
\begin{align}
\label{eq:iso_delta_normE}
\vec{E} = \hat{L}_1\phi = \vec{n}\phi  &\implies \norm{\vec{E}} = \phi\\
&\implies \vec{E} \parallel \vec{n} \qquad \text{($\vec{E}$ and $\vec{n}$ are parallel)}
\end{align}
The second moment equation shows that the second moment is found by the outer product of the defining vector. Therefore, assuming a delta distribution results in the following Eddington tensor (equation~\ref{eq:eddington_tensor}):
\begin{align}
T_{ij} = \vec{n}_i\vec{n}_j
\label{eq:iso_delta_T}
\end{align}
The form of the Eddington tensor for a delta distribution of radiance in angular domain is given. In the end, this tensor is to be used to approximate the second moment of the radiance field $P$ in equation~\ref{eq:general_diffusion_equation} by $T\phi\approx P$. However, this way the direction $\vec{n}$ is introduced as another unknown. The next step in the derivation is, to reformulate $T$ and eliminate of the unknown $\vec{n}$, which is possible under certain assumptions.
%An important consideration is the fact, that the vector $\vec{n}$ is scaled to infinite length under the integral sign.
% \emph{This means our delta distribution is represented by a single vector of infinite length with direction $\vec{n}$}.

Inserting the approximation $T\phi$ into the flux-vector definition (equation~\ref{eq:me_first_resolved_E}) and assuming an isotropic emission $Q$ gives:
\begin{align*}
\vec{E}&= -\frac{1}{\sigma_t'}\operatorname{div}\left (T\phi\right )
\end{align*}
The key assumption used is that the spatial variations of $T$ can be neglected. Fundamentally, the approach is to assume, that the radiance field $L$ can be separated into a product of two functions: one depending on angle and another function depending on position. This allows to express the divergence of the second moment as a matrix product with a gradient vector:
\begin{align}
\vec{E}&= -T\left(\frac{1}{\sigma_t'}\nabla\phi\right )
\label{eq:second_moment_iso2}
\end{align}
For the derivation the normalized flux will be addressed, which is given by scaling the flux-vector with $1/\phi$:
\begin{align}
\widehat{\vec{E}} = \frac{\vec{E}}{\phi}= -T\underbrace{\left(\frac{\nabla\phi}{\sigma_t'\phi}\right )}_{=\vec{R}}
\label{eq:second_moment_iso3}
\end{align}
Equation~\ref{eq:first_moment_Lhat} showed, that $\vec{n}$ points into the same direction as the flux-vector $\vec{E}$. This means that the result of the transformation $T$ will be a vector, that is parallel to $\vec{n}$. Since the Eddington tensor has been constructed with $T_{ij}=\vec{n}_i\vec{n}_j$, it is known that $\vec{n}$ is the only eigenvector of $T$, with an eigenvalue greater than zero. It can be concluded that (under the assumption of negligible spatial variation of $T$) the transformation $T$ is applying a scaling operation and the dimensionless gradient $\vec{R}$ is parallel to $E$. Therefore the product of tensor $T$ is expressed with $\vec{R}$ as a scaling operation and equation~\ref{eq:second_moment_iso3} then becomes:
\begin{align}
\widehat{\vec{E}} = -\lambda\left(\frac{\nabla\phi}{\sigma_t'\phi}\right )
\label{eq:second_moment_iso4}
\end{align}
With an assumption about the spatial variation of $T$, the normalized flux-vector $\hat{L}_1$ could be expressed with respect to the zero moment $\phi$ and its spatial derivatives. What remains to be done is to find an expression for the eigenvalue $\lambda$. Then an expression for $\vec{E}$ (by using $\vec{E}=\hat{L}_1\phi$) would be found, which has no self-dependence and therefore can be used to substitute $\vec{E}$ into the zero moment equation.

Apparently, finding $\lambda$ is easy in the case of a delta distribution of radiance. In that case it is known from equation~\ref{eq:iso_delta_normE}, that $\norm{\vec{E}}=\phi$ and therefore $\norm{\vec{E}/\phi}=1$. This requires that:
\begin{align*}
%\norm[\big]{\vec{f}}=\norm[\Bigg]
\norm{\vec{E}}
=
\norm{-\lambda\left(\frac{\nabla\phi}{\sigma_t'\phi}\right )}=1
\implies
\lambda=\frac{\sigma_t'\phi}{\norm{\nabla\phi}}
\end{align*}
Note that lambda is under the vector norm, which would make the sign of lambda ambiguous (it could be positive or negative: in both cases the length of the normalized flux-vector would be one). However, as mentioned earlier, the way $T=\vec{n}_i\vec{n}_j$ is defined allows the conclusion that $\lambda$ is the only Eigenvalue of $T$ and must be greater one. This gives the final expression for the flux-vector $\vec{E}$:
\begin{align*}
\vec{E}=\widehat{\vec{E}}\phi= -\lambda\left(\frac{\nabla\phi}{\sigma_t'\phi}\right )\phi= -\frac{\nabla\phi}{\norm{\nabla\phi}}\phi
,
\end{align*}
which states that the flux-vector is the unit vector pointing into the direction of the gradient of $\phi$ scaled by $\phi$ itself. Inserting this into the collapsed $P_1$-equation (equation~\ref{eq:general_diffusion_equation}) gives:
\begin{align}
\label{eq:iso_delta_advection_equation}
\nabla\cdot\left(\frac{\nabla\phi}{\norm{\nabla\phi}}\phi\right) &= \sigma_a\phi - Q_0
\end{align}
If we consider a purely scattering medium without emission ($\vec{\omega}_a=0$ and $Q_0=0$) and define $\vec{u}=\frac{\nabla\phi}{\norm{\nabla\phi}}$ to be the unit fluence gradient vector field, then it becomes apparent that this equation resembles some form of a steady state advection equation:
\begin{align}
\label{eq:iso_delta_advection_equation2}
\nabla\cdot\left(\vec{u}\phi\right) &= 0
\end{align}

In case of a radiance delta distribution in angular domain and with the assumption of negligible spatial variation of $T$, the moment equation expresses how the zero moment quantity $\phi$ is moved around by its normalized gradient.

This section highlights that the streaming transport is best represented by advection in case of the two term expansion of the radiative transfer equations. The core idea behind flux-limited diffusion is, to mix diffusive transport and advective transport, depending on local properties at $\vec{x}$. This is developed in the next section.