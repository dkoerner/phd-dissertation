\section{Streaming Limit Approximation}
\label{sec:fld_streaming_limit_approximation}

\TD{give reference}

In the previous section, we introduced the transport measure $R$, which contains the factor $\norm{\nabla\phi}/\phi$. This factor approaches one as transport becomes less diffusive and enters the streaming regime. In the case of $\norm{\nabla\phi}/\phi=1$, the length of the flux-vector matches the amount of total power. In this case, light is coming from a single direction. The radiance distribution of this configuration is given as
\begin{align}
\hat{L}\left(\omega\right)=\delta_{\Omega}\left(\omega,\vec{n}\right)
\end{align}
where $\delta_{\Omega}$ is the angular Dirac delta distribution into direction $\vec{n}$. Spatial dependency is omitted and not relevant for this discussion. The moments of this distribution are:
\begin{align}
\label{eq:zero_moment_Lhat}
\hat{L}_0\left(\omega\right)&=\int_{\Omega}{\delta_{\Omega}\left(\omega,\vec{n}\right)\ud\omega} = 1\\
\label{eq:first_moment_Lhat}
\hat{L}_1\left(\omega\right)&=\int_{\Omega}{\delta_{\Omega}\left(\omega,\vec{n}\right)\omega\ud\omega} = \vec{n}\\
\label{eq:second_moment_Lhat}
\hat{L}_2\left(\omega\right)&=\int_{\Omega}{\delta_{\Omega}\left(\omega,\vec{n}\right)\omega_i\omega_j\ud\omega} = \vec{n}_i\vec{n}_j
\end{align}
The zero moment expresses that all power given by $\phi$ comes from a single direction. The second equation shows, that the first moment of a delta distribution is identical to the vector which defines that distribution ($\vec{n}$ in our case). Also important is, that we can conclude that the length of the flux-vector equals the zero moment and its direction is $\vec{n}$ (all under the assumption of a delta radiance distribution):
\begin{align}
\label{eq:iso_delta_normE}
\vec{E} = \hat{L}_1\phi = \vec{n}\phi  &\implies \norm{\vec{E}} = \phi\\
&\implies \vec{E} \parallel \vec{n} \qquad \text{($\vec{E}$ and $\vec{n}$ are parallel)}
\end{align}
We see from the second moment equation, that the second moment is found by the outer product of the defining vector. Therefore, assuming a delta distribution results in the following Eddington tensor (see definition in equation~\ref{eq:eddington_tensor}):
\begin{align}
T_{ij} = \vec{n}_i\vec{n}_j
\label{eq:iso_delta_T}
\end{align}
Now we know the form of the Eddington tensor for a delta distribution of radiance in angular domain. In the end, this tensor is to be used to approximate the second moment of the radiance field $P$ in equation~\ref{eq:general_diffusion_equation} by $T\phi\approx P$. However, this way we introduce the direction $\vec{n}$ as another unknown. The next step in the derivation therefore is, to reformulate $T$ and eliminate of the unknown $\vec{n}$, which is possible under certain assumptions.
%An important consideration is the fact, that the vector $\vec{n}$ is scaled to infinite length under the integral sign.
% \emph{This means our delta distribution is represented by a single vector of infinite length with direction $\vec{n}$}.

Inserting the approximation $T\phi$ into the flux-vector definition (equation~\ref{eq:me_first_resolved_E}) and assuming an isotropic emission $Q$ gives:
\begin{align*}
\vec{E}&= -\frac{1}{\sigma_t'}\operatorname{div}\left (T\phi\right )
\end{align*}
The key assumption used is that the spatial variations of $T$ can be neglected. Fundamentaly, the approach therefore is to assume, that the radiance field $L$ can be seperated into a product of two functions. One depending on angle and another function depending on position. This allows us to express the divergence of the second moment as a matrix product with a gradient vector:
\begin{align}
\vec{E}&= -T\left(\frac{1}{\sigma_t'}\nabla\phi\right )
\label{eq:second_moment_iso2}
\end{align}
For the derivation we look at the normalized flux, which we find by scaling the flux-vector by $1/\phi$:
\begin{align}
\widehat{\vec{E}} = \frac{\vec{E}}{\phi}= -T\underbrace{\left(\frac{\nabla\phi}{\sigma_t'\phi}\right )}_{=\vec{R}}
\label{eq:second_moment_iso3}
\end{align}
We know from equation~(\ref{eq:first_moment_Lhat}), that $\vec{n}$ points into the same direction as the flux-vector $\vec{E}$. This means, that the result of the transformation $T$ will be a vector which is parallel to $\vec{n}$. Since we construct the Eddington tensor with $T_{ij}=\vec{n}_i\vec{n}_j$, we know that $\vec{n}$ is the only eigenvector of $T$, with an eigenvalue greater than zero. Therefore we can conclude, that (under the assumption of negligible spatial variation of $T$) the transformation $T$ is applying a scaling operation and the dimensionless gradient $\vec{R}$ is parallel to $E$. We therefore can express the product of tensor $T$ with $\vec{R}$ as a scaling operation and equation~(\ref{eq:second_moment_iso3}) then becomes:
\begin{align}
\widehat{\vec{E}} = -\lambda\left(\frac{\nabla\phi}{\sigma_t'\phi}\right )
\label{eq:second_moment_iso4}
\end{align}
So by making an assumption about the spatial variation of $T$, we were able to express our normalized flux-vector $\hat{L}_1$ with respect to the zero moment $\phi$ and its spatial derivatives. What remains to be done is to find an expression for the eigenvalue $\lambda$. Then we would have found an expression for $\vec{E}$ (by using $\vec{E}=\hat{L}_1\phi$), which does not depend on itsself in anyway and therefore can be used to substitute $\vec{E}$ into the zero moment equation.

Apparently, finding $\lambda$ is easy in the case of a delta distribution of radiance. In that case we know from equation~(\ref{eq:iso_delta_normE}), that $\norm{\vec{E}}=\phi$ and therefore $\norm{\vec{E}/\phi}=1$. This requires that:
\begin{align*}
%\norm[\big]{\vec{f}}=\norm[\Bigg]
\norm{\vec{E}}
=
\norm{-\lambda\left(\frac{\nabla\phi}{\sigma_t'\phi}\right )}=1
\implies
\lambda=\frac{\sigma_t'\phi}{\norm{\nabla\phi}}
\end{align*}
Note that lambda is under the vector norm, which would make the sign of lambda ambiguous (it could be positive or negative: in both cases the length of the normalized flux-vector would be one). However, as mentioned earlier, the way we define $T=\vec{n}_i\vec{n}_j$ allows us to conclude that $\lambda$ is the only Eigenvalue of $T$ and must greater one. This gives our final expression for the flux-vector $\vec{E}$:
\begin{align*}
\vec{E}=\widehat{\vec{E}}\phi= -\lambda\left(\frac{\nabla\phi}{\sigma_t'\phi}\right )\phi= -\frac{\nabla\phi}{\norm{\nabla\phi}}\phi
\end{align*}
which states that the flux-vector is the unit vector pointing into the direction of the gradient of $\phi$ scaled by $\phi$ itsself. Inserting this into the collapsed $P_1$-equation (equation~\ref{eq:general_diffusion_equation}) gives:
\begin{align}
\label{eq:iso_delta_advection_equation}
\nabla\cdot\left(\frac{\nabla\phi}{\norm{\nabla\phi}}\phi\right) &= \sigma_a\phi - Q_0
\end{align}
We see that in case of a delta distribution of radiance in angular domain and with the assumption of negligible spatial variation of $T$, the moment equation turns into an advection equation, where the zero moment quantity $\phi$ is moved around by its normalized gradient.

We have seen in this section, that the streaming transport is best represented by advection in case of the two term expansion of the radiative transfer equations. The core idea behind flux-limited diffusion is, to mix diffusive transport and advective transport, depending on local properties at $\vec{x}$. This is developed in the next section.