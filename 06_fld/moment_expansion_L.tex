\section{Moment Expansion of the Radiance Field}
\label{sec:da_moment_expansion_L}

In section~\ref{sec:sh} it has been established that spherical functions can be represented using a linear combination of spherical harmonics basis functions (equation~\ref{eq:sh_reconstruction}). The coefficients of these basis functions are found by projection (equation~\ref{eq:sh_projection}). Another option for representing spherical functions is the coordinate free form given by tensor calculus. This form is derived by representing the spherical harmonics basis functions themself as linear combination of products of the components of the basis vectors, which establish the underlying coordinate system. Because this representation allows to change the coordinate system independently from the representation, it is referred to as being coordinate free. For a refresher on tensor calculus, an introductory standard such as Grinfeld~\cite{Grinfeld13} is recommended. As with the spherical harmonics expansion, the coordinate free representation can be found for the radiance field, as well as the individual terms of the radiative transfer equation. This is referred to as the moment expansion or multipole expansion of the radiance field and the radiative transfer equation respectively.

The moment expansion is a different approach to represent the spherical harmonics expansion of the radiance field and the radiative transfer equation. Its benefit lies in its notational efficiency, especially for lower order. However, for the truncation order of $N=2$ and higher, it introduces additional redundancy which makes the standard spherical harmonics expansion much better suited when deriving computational methods from the theory, such as the $P_N$-method in chapter~\ref{sec:pnmethod}. Because the flux-limited diffusion is related to the first order truncation $N=1$, the moment expansion is the standard way of representing the theory. No additional redundancy is introduced and the notational efficiency allows to express the underlying intuition easier.

The moment expansion is derived, starting from the spherical harmonics reconstruction of the radiance field:
\begin{align}
\label{eq:fld_moment_expansion_sh_expansion}
L\left(\vec{\omega}\right) =
\sum_{l=0}^{\infty}\sum_{m=-l}^{l}L^{l,m}\SHBC^{l,m}
\left(\vec{\omega}\right)
\end{align}
with $L^{l,m}$ being the spherical harmonics basis function coefficients, which are found by spherical harmonics projection (equation~\ref{eq:sh_projection}).

The main step in the derivation of the moment expansion is the replacement of the spherical harmonics basis function $\SHBC^{l,m}$ by a sum of tensor contractions:
\begin{align}
\label{eq:fld_moment_expansion_sum}
\SHBC^{l,m}(\vec{\omega}) =
\sum_{j=0}^\infty{y^{l,m}_j\odot N_j(\vec{\omega})}
\end{align}
The operator symbol $\odot$ denotes a tensor contraction. In this instance, it is a sum over products of the individual components of the tensors $y^{l,m}_j$ and $N_j$. Therefore, each contraction collapses into a scalar value. These values are then added up over all moments $j$. The tensor $N_j$ is a tensor of rank $j$ with $N_0=1$, and is constructed by a sequence of outer products of the direction vector $\vec{\omega}$:
\begin{align}
N_j\left(\vec{\omega}\right)
=N_{k_1, k_2, ..., k_{j-1}, k_j} 
=\vec{\omega}_{k_1}\vec{\omega}_{k_2}...\vec{\omega}_{k_{j-1}}\vec{\omega}_{k_j} 
\end{align}
The index set $\{k_1, k_2, ..., k_{j-1}, k_j\}$ identifies specific tensor components, with each index running over all components of $\vec{\omega}$ ($k_i \in \{x, y, z\}$).

The tensors $y^{l,m}_j$ are found by expanding $\SHBC^{l,m}(\vec{\omega})$ into a sum of tensor components of $N_j$. The factors to these components can be easily extracted and constitute the components of $y^{l,m}_j$. An expansion is possible by using the following non-recursive definition of the Legendre-polynomial, which is derived with the use of the multiple-angle formula\footnote{\url{http://mathworld.wolfram.com/Multiple-AngleFormulas.html} (date of access: 07/26/2018)}:
\begin{align}
P^{lm}(\theta, \phi) = \operatorname{sin}^m(\phi)\sum_j^{\lfloor\frac{l-m}{2}\rfloor}a^{lmj}\operatorname{cos}^{l-m-2j}(\theta)
\qquad \text{for } m \ge 0
\end{align}
Inserting this into the definition of the spherical harmonics basis function $\SHBC^{l,m}$ (equation~\ref{eq:sh_definition_C}) results in:
\begin{align}
\SHBC^{l,m}(\theta, \phi) = C^{l,m}\left(e^{i\phi}\operatorname{sin}\left(\phi\right)\right)^m \sum_j^{\lfloor\frac{l-m}{2}\rfloor}a^{lmj}\operatorname{cos}^{l-m-2j}(\theta)
\qquad \text{for } m \ge 0
\end{align}
As a next step, $\SHBC^{l,m}$ will be expressed in terms of unit direction vector $\vec{\omega}$, instead of spherical coordinates. This is done by using the identity $e^{i\phi}\operatorname{sin}\left(\phi\right) = \vec{\omega}_x + i\vec{\omega}_y$ and $\operatorname{cos}(\theta) = \vec{\omega}_z$:
\begin{align}
\SHBC^{l,m}(\vec{\omega}) &=  C^{l,m}\left( \vec{\omega}_x + i\vec{\omega}_y\right)^m \sum_j^{\lfloor\frac{l-m}{2}\rfloor}a^{lmj}\vec{\omega}_z^{l-m-2j}
&\qquad \text{for } m \ge 0
\label{eq:da_moment_expansion_L_shbasis2}
\\
&= \left(-1\right)^m\overline{\SHBC^{l,\abs{m}}}
&\qquad \text{for } m < 0
\end{align}
Expanding the expression for concrete values of $l$ and $m$, will give a sequence of terms, which involve components of $N_j$, multiplied with specified factors. By seeing the expanded formula as a tensor contraction, the components of $y^{l,m}_j$ can be extracted. For $l=0$ and $m=0$ equation~\ref{eq:da_moment_expansion_L_shbasis2} results in:
\begin{align}
\SHBC^{0,0}(\vec{\omega}) = C^{{0,0}}a^{{0,0,0}} = \frac{1}{\sqrt{4\pi}}N_0\quad,
\end{align}
from wich the following can be inferred:
\begin{align}
y^{0,0}_0 = \frac{1}{\sqrt{4\pi}}
\end{align}
For $l=1$ the result is:
\begin{align}
\SHBC^{1,0}(\vec{\omega}) &= C^{{1,0}}a^{{1,0,0}}\vec{\omega}_z = \sqrt{\frac{3}{4\pi}}\vec{\omega}_z
\\
\SHBC^{1,1}(\vec{\omega}) &= C^{{1,1}}a^{{1,1,0}}\vec{\omega}_x+C^{{1,1}}a^{{1,1,0}}\vec{\omega}_yi = -\sqrt{\frac{3}{8\pi}}\vec{\omega}_x - \sqrt{\frac{3}{8\pi}}\vec{\omega}_yi
\\
\SHBC^{1,-1}(\vec{\omega}) &= -\overline{\SHBC^{1,1}}(\vec{\omega})
= \sqrt{\frac{3}{8\pi}}\vec{\omega}_x - \sqrt{\frac{3}{8\pi}}\vec{\omega}_yi
\end{align}
From which the following can be inferred:
\begin{align}
y^{1,-1}_1 = \begin{pmatrix}\sqrt{\frac{3}{8\pi}}  \\ -\sqrt{\frac{3}{8\pi}}i \\ 0 \end{pmatrix}
\qquad
y^{1,0}_1 = \begin{pmatrix}0  \\ 0 \\ \sqrt{\frac{3}{4\pi}} \end{pmatrix}
\qquad
y^{1,1}_1 = \begin{pmatrix}-\sqrt{\frac{3}{8\pi}}  \\ -\sqrt{\frac{3}{8\pi}}i \\ 0 \end{pmatrix}
\end{align}
This scheme continues and becomes more involved for higher moments. For example with $l=3$ an $m=0$:
\begin{align}
\SHBC^{3,0}(\vec{\omega}) &= \underbrace{C^{{3,0}}a^{{3,0,0}}}_{\in y^{3,0}_3}{\vec{\omega}_z}^{3}+\underbrace{C^{{3,0}}a^{{3,0,1}}}_{\in y^{3,0}_1}\vec{\omega}_z
\end{align}
Here a term containing ${\vec{\omega}_z}^{3}$ is presented, which is an element of $N_3$ and another term containing $\vec{\omega}_z$, which is an element of $N_1$. This shows the expression of the the spherical harmonics basis functions as a sum over tensor contractions of different ranks, as Equation~\ref{eq:fld_moment_expansion_sum}.

Inserting equation~\ref{eq:fld_moment_expansion_sum} into equation~\ref{eq:fld_moment_expansion_sh_expansion} gives:
\begin{align}
L\left(\vec{\omega}\right) =
\sum_{l=0}^{\infty}\sum_{m=-l}^{l}L^{l,m}\sum_{j=0}^\infty{y^{l,m}_j\odot N_j(\vec{\omega})}
\end{align}
After rearranging terms that leads to:
\begin{align}
L\left(\vec{\omega}\right) =
\sum_{l=0}^{\infty}\sum_{m=-l}^{l}\sum_{j=0}^\infty{L^{l,m}y^{l,m}_j\odot N_j(\vec{\omega})}
\end{align}
For each spherical harmonics basis $l,m$, all rank $j$ tensors will be iterated, and contracted with their respective $N_j$ (after applying the respective spherical harmonics coefficient as weight). Since the indices $l$ and $j$ run over the same range, these can be swapped. This is identical to rearranging the order of terms:
\begin{align}
L\left(\vec{\omega}\right) =
\sum_{l=0}^{\infty}\underbrace{\sum_{j=0}^\infty\sum_{m=-j}^{j}{L^{j,m}y^{j,m}_l}}_{=f_l}
\odot N_l(\vec{\omega})
\end{align}
For each spherical harmonics band $l$, all spherical harmonics bases $j,m$ will be iterated, weighted and the rank $l$ tensor of each particular basis contracted with $N_l$. Further, the sum of equal rank tensors can be factorized into the moment tensor $f_l$ to receive the moment expansion of the radiance field:
\begin{align}
L\left(\vec{\omega}\right) =
\sum_{l=0}^{\infty}f_l\odot N_l(\vec{\omega})
\end{align}
with
\begin{align}
f_l = \sum_{j=0}^\infty\sum_{m=-j}^{j}{L^{j,m}y^{j,m}_l}
\end{align}
For $f_0$ we get:
\begin{align}
f_0 &= L^{0,0}y^{0,0}_0 = \int_\Omega{L\left(\vec{\omega}\right)\SHBC^{0,0}\left(\vec{\omega}\right)\ud\vec{\omega}}\frac{1}{\sqrt{4\pi}}\\
&=
\frac{1}{4\pi}\int_\Omega{L\left(\vec{\omega}\right)\ud\vec{\omega}} =
\frac{1}{4\pi}\phi
\end{align}
For $f_1$, $L^{1,-1}y^{1,-1}_1$, $L^{1,0}y^{1,0}_1$ and $L^{1,1}y^{1,1}_1$ shall be added. Starting with $L^{1,-1}y^{1,-1}_1$, following result is given:
\begin{align}
L^{1,-1}y^{1,-1}_1 &= 
\int_\Omega{L\left(\vec{\omega}\right)\SHBC^{1,-1}\left(\vec{\omega}\right)\ud\vec{\omega}}
\begin{pmatrix}\sqrt{\frac{3}{8\pi}}  \\ -\sqrt{\frac{3}{8\pi}}i \\ 0 \end{pmatrix}
\\
&= 
\frac{3}{8\pi}\int_\Omega{L\left(\vec{\omega}\right)\operatorname{sin}\theta e^{-i\phi}\ud\vec{\omega}}
\begin{pmatrix}1  \\ -i \\ 0 \end{pmatrix}
\end{align}
Using the trigonometric identities $e^{-i\phi} = \operatorname{cos}\phi - i\operatorname{sin}\phi$ and $e^{i\phi} = \operatorname{cos}\phi + i\operatorname{sin}\phi$, the outcome is:
\begin{align}
L^{1,-1}y^{1,-1}_1 &= 
\begin{pmatrix}
\frac{3}{8\pi}\int_\Omega{L\left(\vec{\omega}\right)\operatorname{sin}\theta\operatorname{cos}\phi\ud\vec{\omega}} - \frac{3}{8\pi}i\int_\Omega{L\left(\vec{\omega}\right)\operatorname{sin}\theta\operatorname{sin}\phi\ud\vec{\omega}}
\\
-\frac{3}{8\pi}i\int_\Omega{L\left(\vec{\omega}\right)\operatorname{sin}\theta\operatorname{cos}\phi\ud\vec{\omega}} - \frac{3}{8\pi}\int_\Omega{L\left(\vec{\omega}\right)\operatorname{sin}\theta\operatorname{sin}\phi\ud\vec{\omega}}
\\
0
\end{pmatrix}
\end{align}
We then use the identity:
\begin{align}
\vec{\omega} &= 
\begin{pmatrix}
\operatorname{sin}\theta\operatorname{cos}\phi
\\
\operatorname{sin}\theta\operatorname{sin}\phi
\\
\operatorname{cos}\theta
\end{pmatrix}\ ,
\end{align}
to arrive at:
\begin{align}
L^{1,-1}y^{1,-1}_1 &= 
\begin{pmatrix}
\frac{3}{8\pi}\int_\Omega{L\left(\vec{\omega}\right)\vec{\omega}_x\ud\vec{\omega}} - \frac{3}{8\pi}i\int_\Omega{L\left(\vec{\omega}\right)\vec{\omega}_y\ud\vec{\omega}}
\\
-\frac{3}{8\pi}i\int_\Omega{L\left(\vec{\omega}\right)\vec{\omega}_x\ud\vec{\omega}} - \frac{3}{8\pi}\int_\Omega{L\left(\vec{\omega}\right)\vec{\omega}_y\ud\vec{\omega}}
\\
0
\end{pmatrix}
\end{align}
Similar procedures for $L^{1,0}y^{1,0}_1$ and $L^{1,1}y^{1,1}_1$, lead to:
\begin{align}
L^{1,0}y^{1,0}_1 &= 
\begin{pmatrix}
0\\
0\\
\frac{3}{4\pi}\int_\Omega{L\left(\vec{\omega}\right)\vec{\omega}_z\ud\vec{\omega}}
\end{pmatrix}
\end{align}
and
\begin{align}
L^{1,1}y^{1,1}_1 &= 
\begin{pmatrix}
\frac{3}{8\pi}\int_\Omega{L\left(\vec{\omega}\right)\vec{\omega}_xi\ud\vec{\omega}} + \frac{3}{8\pi}i\int_\Omega{L\left(\vec{\omega}\right)\vec{\omega}_y\ud\vec{\omega}}
\\
\frac{3}{8\pi}i\int_\Omega{L\left(\vec{\omega}\right)\vec{\omega}_xi\ud\vec{\omega}} + \frac{3}{8\pi}\int_\Omega{L\left(\vec{\omega}\right)\vec{\omega}_y\ud\vec{\omega}}
\\
0
\end{pmatrix}
\end{align}
Subsequently $f_1$ can be merged. Note, how all the imaginary terms cancel out:
\begin{align}
f_1 &= 
L^{1,-1}y^{1,-1}_1+L^{1,0}y^{1,0}_1+L^{1,1}y^{1,1}_1
=
\frac{3}{4\pi}
\begin{pmatrix}
\int_\Omega{L\left(\vec{\omega}\right)\vec{\omega}_x\ud\vec{\omega}}
\\
\int_\Omega{L\left(\vec{\omega}\right)\vec{\omega}_y\ud\vec{\omega}}
\\
\int_\Omega{L\left(\vec{\omega}\right)\vec{\omega}_z\ud\vec{\omega}}
\end{pmatrix}
=
\frac{3}{4\pi}\vec{E}
\end{align}
With $f_0$ and $f_1$, the moment expansion of the radiance field truncated after the first moment can be written down and used for reconstruction in the diffusion approximation:
\begin{align}
\nonumber
L\left(\vec{\omega}\right) &= 
f_0\odot N_0 + f_1\odot N_1
\\
\nonumber
&=
\frac{1}{4\pi}\phi + \frac{3}{4\pi}\vec{E} \odot \vec{\omega}
\\
\label{eq:moment_expansion_L}
&=
\frac{1}{4\pi}\phi + \frac{3}{4\pi}\left(w \cdot \vec{E}\right)
\end{align}
The tensor $f_0$ is expressed in terms of the fluence $\phi$ and the tensor $f_1$ in terms of the flux vector $\vec{E}$. These quantities are often referred to as the zero and first moment of the radiance field respectively, they generalize to higher order and are denoted $L_i$. They are computed using the moment projection operator $\mu_i$:
\begin{align}
\label{eq:moment_expansion_mu}
\mu_i\left[L\right] = L_i = \int_\Omega{L\left(\vec{\omega}\right)N_i\left(\vec{\omega}\right)\ud\vec{\omega}}
\end{align}
An important quantity for deriving flux-limited diffusion is the second moment of the radiance field, the radiative pressure tensor $L_2$. Its intuition is very similar to the stress tensor in continuum mechanics.

