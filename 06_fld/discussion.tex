\section{Discussion}
\label{sec:fld_discussion}

This chapter introduced flux-limited diffusion to the problem of rendering in computer graphics. It has its origins in the astrophysics domain (Levermore et al.~\cite{Levermore81}) and can be seen as an extension to the classical diffusion approximation, which was covered in chapter~\ref{sec:da_discussion}. In addition to revisiting the theory in the context of computer graphics, the major contribution of this chapter is the introduction of a new method for solving the non-linear flux-limited diffusion equation, which resulted in a publication~\cite{Koerner14}.

The results shown in the previous section indicate that flux-limited diffusion produces significantly more accurate results than classical diffusion at a moderate increase in computational cost. Therefore, classical diffusion is only an option if a maximum in performance is required with a high tolerance towards significant approximation error. Otherwise flux-limited diffusion is the better alternative. In addition flux-limited diffusion outperforms the $P_N$-method for some higher levels than $P_1$ (on which flux-limited diffusion is based upon). It produces equally accurate results and appears to better preserve energy at a much smaller resource footprint.

The flux-limited diffusion theory was shown to be based on the idea of spatially blending advection and diffusion. This mixing is based on a local quantity. While this thesis is concerned with generating images based on physical based light transport, an alternative route to explore may be non-photrealistic rendering. In particular the work on diffusion curves by Orzan~\cite{Orzan08} could be combined with flux-limited diffusion. The idea behind diffusion curves is to use diffusion to define the pixel colors in an image. User-defined curves---essentially emitters in two-dimensional space---determine the final image content. Recently Prevost et al.~\cite{Prevost15} extended this idea by using raytracing in the two-dimensional image space. By using flux-limited diffusion this technique may be extended with advection and may allow for more expressive imaging.

