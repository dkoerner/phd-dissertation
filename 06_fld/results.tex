\section{Results}
\label{sec:fld_results}

With the solver introduced in the previous section it can now be run for comparison on the problems, which were used for the $P_N$-method in chapter~\ref{sec:pnmethod} and the diffusion approximation in chapter~\ref{sec:diffusion_approximation}.

The rendering integration aspects to be considered are identical to the diffusion approximation (section~\ref{sec:da_results}). As with the diffusion approximation, flux-limited diffusion will give a solution for the zero moment $\phi$ from which the second moment can be recovered using equation~\ref{eq:diffusion_ficks_law}. Both moments can be used to reconstruct the truncated spherical harmonics expansion of the radiance field by using equation~\ref{eq:moment_expansion_L}. The difference to the diffusion approximation solution is that the fluence should be more accurate as the flux-limit constraint had been accounted for.

\subsection{Point Source Problem}
\label{sec:pn_results_pointsource}

At first it shall be turned towards the point source problem in order to validate the results from the flux-limited diffusion solver and assess its accuracy by comparing against an analytical ground truth solution.
\begin{figure}[h]
\centering
\begin{subfigure}{0.49\columnwidth}
%\includegraphics[width=\columnwidth]{images/checkerboard2d_p1_neumann_staggered_starmap.png}
\missingfigure{pointsource plots FLD DA maybe a seperate plot for plotting R?}
\caption{TODO}
\label{fig:fld_results_pointsource_1}
\end{subfigure}%
\hspace{0.01\columnwidth}
\begin{subfigure}{0.49\columnwidth}
%\includegraphics[width=\columnwidth]{images/checkerboard2d_p1_neumann_staggered.png}
\missingfigure{R along radius from point source}
\caption{TODO}
\label{fig:fld_results_pointsource_2}
\end{subfigure}%
\caption{TODO}
\label{fig:fld_results_pointsource}
\end{figure}

%\ref{fig:fld_results_pointsource_1}\subref{fig:fld_results_pointsource_1}

As seen in figure~\ref{fig:fld_results_pointsource_1}, the result from flux-limited diffusion agree well with the ground truth and diffusion approximation results. Like with the diffusion approximation, a perfect match to the ground truth is not to be expected as flux-limited diffusion is still an approximation based on the spherical harmonics expansion of the radiative transfer equation truncated after the first moment. However, very close to the point source the flux-limited diffusion is more accurate than classical diffusion. This is explained by the fact that the fluence gradient becomes arbitrary large as the point light is approached, due to the geometry term (figure~\ref{fig:fld_results_pointsource_2}). This causes the transport measure $R$ to likewise become arbitrary large, which shows that pure streaming transport will dominate arbitrarily close to the point source.

In comparison to the results from $P_N$-method and $P_5$ in particular, the $P_N$-method with higher truncation order is more accurate than flux-limited diffusion. This an important observation since it means that increasing the truncation order will alleviate the problems caused by not accounting for the flux-limit constraint. However, flux-limited diffusion has a signficicant advantage over $P_N$ due to its computational and memory efficiency.

\subsection{Checkerboard Problem}
\label{sec:pn_results_checkerboard}

For the checkerboard problem, the hand-crafted stencil for flux-limited diffusion had been modified to ignore all derivative terms in the z-dimension. The results in figure~\ref{fig:fld_results_checkerboard_1} show how...
\TD{discuss fld solution on checkerboard}
\begin{figure}[h]
\centering
\begin{subfigure}{0.49\columnwidth}
%\includegraphics[width=\columnwidth]{images/checkerboard2d_p1_neumann_staggered_starmap.png}
\missingfigure{checkerboard plots FLD}
\caption{TODO}
\label{fig:fld_results_checkerboard_1}
\end{subfigure}%
\hspace{0.01\columnwidth}
\begin{subfigure}{0.49\columnwidth}
%\includegraphics[width=\columnwidth]{images/checkerboard2d_p1_neumann_staggered.png}
\missingfigure{checkerboard plots CDA for comparison}
\caption{TODO}
\label{fig:fld_results_checkerboard_2}
\end{subfigure}%
\caption{TODO}
\label{fig:fld_results_checkerboard}
\end{figure}

\subsection{Procedural Cloud}
\label{sec:pn_results_clouds}

Finally the flux-limited diffusion is run on the procedural cloud dataset described in section~\ref{sec:pn_results_clouds} to assess how it performs in a more practical setting. As seen in figure~\ref{fig:fld_results_nebulae}, flux-limited diffusion does a significantly better job at conserving energy than classical diffusion or $P_5$. This is explained by the fact that the dataset contains very strong density gradients and close-to vacuum regions in which diffusive transport is not able to capture light transport well. While $P_5$ produces better results than classical diffusion, it is somehow surprising to see how much better flux-limited diffusion is able to capture the directly lit regions of the cloud.

Further flux-limited diffusion well captures the indirectly illuminated parts at the bottom of the dataset. Here flux-limited diffusion also offers a signficiant improvement over classical diffusion approximation. Upon close inspection and comparison with $P_5$, it can be seen that the flux-limited diffusion solution appears very flat while the $P_5$ solution exhibits finer variations which closer resemble the path-traced resuls. This is explained by the fact that the angular domain is better resolved with $P_5$. The energy conserving nature of flux-limited diffusion still produced a result, which much closer resembles the path-traced result than $P_5$.

\begin{figure}[h]
\centering
\begin{subfigure}{0.31\columnwidth}
%\includegraphics[width=\columnwidth]{images/checkerboard2d_p1_neumann_staggered_starmap.png}
\missingfigure{checkerboard plots FLD}
\caption{TODO}
\label{fig:fld_results_nebulae_1}
\end{subfigure}
\hspace{0.01\columnwidth}
\begin{subfigure}{0.31\columnwidth}
%\includegraphics[width=\columnwidth]{images/checkerboard2d_p1_neumann_staggered.png}
\missingfigure{nebulae plots CDA for comparison}
\caption{TODO}
\label{fig:fld_results_nebulae_2}
\end{subfigure}
\hspace{0.01\columnwidth}
\begin{subfigure}{0.31\columnwidth}
%\includegraphics[width=\columnwidth]{images/checkerboard2d_p1_neumann_staggered.png}
\missingfigure{nebulae plots P5 for comparison}
\caption{TODO}
\label{fig:fld_results_nebulae_2}
\end{subfigure}%
\caption{TODO}
\label{fig:fld_results_nebulae}
\end{figure}

In terms of performance chacteristics it can be expected that flux-limited diffusion requires more computational effort than the diffusion approximation, due to the non-linear nature of flux-limited diffusion coefficient, which requires to be updated for every voxel in each Gauss-Seidel iteration. This impact appears moderate when compared to the visual improvement it brings. When compared to $P_N$ it can be concluded that flux-limited diffusion offers a much better result at a significantly lower computational cost.
\begin{table}[!h]
	\centering
	\caption[TODO]{Performance characteristics of flux-limited diffusion for the procedural cloud dataset.}
	\label{tab:results_cloud}
	% \flushleft
	\begin{tabular}{l r}
    \hline
	\textbf{N}
    & 1
    \\
    \hline
    Number of rows/columns in $A$
    & ?
    \\
    Size of linear system (in MB)
    & ?
    \\
    Solve time (in min)
    & ?
	\end{tabular}
\end{table}
%\TD{complete table maybe table showing the performance characteristic for all three datasets}
%\TD{mention how the performance of FLD allows aplication of transfer function and solve for different color channels}
%\TD{mention publication}
%\TD{mention application in elementacular}