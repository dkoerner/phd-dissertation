\section{Moment Expansion of the Radiance Field}
\label{sec:da_moment_expansion_L}

%In this section, the moment expansion of the radiance field is introduced. It represents the radiance field after being discetized in angular domain using spherical harmonics. The difference to the standard spherical harmonics expansion is, that writing the radiance field in terms of its moments is notationally more efficient and therefore is particularily useful for theoretical treatments like in this chapter. However, the downside is, that it introduces a lot of reduncancy, which makes working with the spherical harmonics expansion more practical for implementation purposes.

We derive the moment expansion by starting from the spherical harmonics reconstruction of the radiance field:
\begin{align}
\label{eq:fld_moment_expansion_sh_expansion}
L\left(\omega\right) =
\sum_{l=0}^{\infty}\sum_{m=-l}^{l}L^{l,m}\SHBC^{l,m}
\left(\omega\right)
\end{align}
with $L^{l,m}$ being the spherical harmonics basis function coefficients, which are found by spherical harmonics projection (equation~\ref{eq:sh_projection}).

The major step in the derivation of the moment expansion is the replacement of the spherical harmonics basis function $\SHBC^{l,m}$ by a sum of tensor contractions:
\begin{align}
\label{eq:fld_moment_expansion_sum}
\SHBC^{l,m}(\omega) =
\sum_{j=0}^\infty{y^{l,m}_j\odot N_j(\omega)}
\end{align}
The operator symbol $\odot$ denotes a tensor contraction. In this instance, it is a sum over products of the individual components of the tensors $y^{l,m}_j$ and $N_j$. Each contraction therefore collapses into a scalar value. These values are then added up over all moments $j$. The tensor $N_j$ is a tensor of rank $j$ with $N_0=1$ and it is constructed by a sequence of outer products of the direction vector $\omega$:
\begin{align}
N_j\left(\omega\right)
=N_{k_1, k_2, ..., k_{j-1}, k_j} 
=\omega_{k_1}\omega_{k_2}...\omega_{k_{j-1}}\omega_{k_j} 
\end{align}
The index set $\{k_1, k_2, ..., k_{j-1}, k_j\}$ identifies specific tensor components, with each index running over all components of $\omega$ ($k_i \in \{x, y, z\}$).

The tensors $y^{l,m}_j$ are found by expanding $\SHBC^{l,m}(\omega)$ into a sum of tensor components of $N_j$. The factors to these components can be easily extracted and constitute the components of $y^{l,m}_j$. The expansion can be done by using the following non-recursive definition of the Legendre-polynomial which is derived using the multiple-angle formula\footnote{\url{http://mathworld.wolfram.com/Multiple-AngleFormulas.html}}:
\begin{align}
P^{lm}(\theta, \phi) = \operatorname{sin}^m(\phi)\sum_j^{\lfloor\frac{l-m}{2}\rfloor}a^{lmj}\operatorname{cos}^{l-m-2j}(\theta)
\qquad \text{for } m \ge 0
\end{align}
Inserting this into the definition of the spherical harmonics basis function $\SHBC^{l,m}$ (see equation~\ref{eq:sh_definition_C}) gives:
\begin{align}
\SHBC^{l,m}(\theta, \phi) = C^{l,m}\left(e^{i\phi}\operatorname{sin}\left(\phi\right)\right)^m \sum_j^{\lfloor\frac{l-m}{2}\rfloor}a^{lmj}\operatorname{cos}^{l-m-2j}(\theta)
\qquad \text{for } m \ge 0
\end{align}
As a next step we express $\SHBC^{l,m}$ in terms of unit direction vector $\omega$, instead of spherical coordinates. This is done by using the identity $e^{i\phi}\operatorname{sin}\left(\phi\right) = \omega_x + i\omega_y$ and $\operatorname{cos}(\theta) = \omega_z$:
\begin{align}
\SHBC^{l,m}(\omega) &=  C^{l,m}\left( \omega_x + i\omega_y\right)^m \sum_j^{\lfloor\frac{l-m}{2}\rfloor}a^{lmj}\omega_z^{l-m-2j}
&\qquad \text{for } m \ge 0
\label{eq:da_moment_expansion_L_shbasis2}
\\
&= \left(-1\right)^m\overline{\SHBC^{l,\abs{m}}}
&\qquad \text{for } m < 0
\end{align}
Expanding the expression above for concrete values of $l$ and $m$ will give a sequence of terms, which involve components of $N_j$, multiplied with some factors. By seeing the expanded formula as a tensor contraction, the components of $y^{l,m}_j$ can be easily extracted. For $l=0$ and $m=0$ equation~\ref{eq:da_moment_expansion_L_shbasis2} results in:
\begin{align}
\SHBC^{0,0}(\omega) = C^{{0,0}}a^{{0,0,0}} = \frac{1}{\sqrt{4\pi}}N_0\quad,
\end{align}
from wich we can infer
\begin{align}
y^{0,0}_0 = \frac{1}{\sqrt{4\pi}}
\end{align}
For $l=1$ we get:
\begin{align}
\SHBC^{1,0}(\omega) &= C^{{1,0}}a^{{1,0,0}}\omega_z = \sqrt{\frac{3}{4\pi}}\omega_z
\\
\SHBC^{1,1}(\omega) &= C^{{1,1}}a^{{1,1,0}}\omega_x+C^{{1,1}}a^{{1,1,0}}\omega_yi = -\sqrt{\frac{3}{8\pi}}\omega_x - \sqrt{\frac{3}{8\pi}}\omega_yi
\\
\SHBC^{1,-1}(\omega) &= -\overline{\SHBC^{1,1}}(\omega)
= \sqrt{\frac{3}{8\pi}}\omega_x - \sqrt{\frac{3}{8\pi}}\omega_yi
\end{align}
From which we can infer:
\begin{align}
y^{1,-1}_1 = \begin{pmatrix}\sqrt{\frac{3}{8\pi}}  \\ -\sqrt{\frac{3}{8\pi}}i \\ 0 \end{pmatrix}
\qquad
y^{1,0}_1 = \begin{pmatrix}0  \\ 0 \\ \sqrt{\frac{3}{4\pi}} \end{pmatrix}
\qquad
y^{1,1}_1 = \begin{pmatrix}-\sqrt{\frac{3}{8\pi}}  \\ -\sqrt{\frac{3}{8\pi}}i \\ 0 \end{pmatrix}
\end{align}
This scheme continues and becomes more involved for higher moments. For example with $l=3$ an $m=0$ we get:
\begin{align}
\SHBC^{3,0}(\omega) &= \underbrace{C^{{3,0}}a^{{3,0,0}}}_{\in y^{3,0}_3}{\omega_z}^{3}+\underbrace{C^{{3,0}}a^{{3,0,1}}}_{\in y^{3,0}_1}\omega_z
\end{align}
Here we have a term containing ${\omega_z}^{3}$, which is an element of $N_3$ and another term containing $\omega_z$, which is an element of $N_1$. This shows how the spherical harmonics basis functions are expressed as a sum over tensor contractions of different ranks, as shown in Equation~\ref{eq:fld_moment_expansion_sum}.

Inserting equation~\ref{eq:fld_moment_expansion_sum} into equation~\ref{eq:fld_moment_expansion_sh_expansion} gives:
\begin{align}
L\left(\omega\right) =
\sum_{l=0}^{\infty}\sum_{m=-l}^{l}L^{l,m}\sum_{j=0}^\infty{y^{l,m}_j\odot N_j(\omega)}
\end{align}
After rearranging terms we get:
\begin{align}
L\left(\omega\right) =
\sum_{l=0}^{\infty}\sum_{m=-l}^{l}\sum_{j=0}^\infty{L^{l,m}y^{l,m}_j\odot N_j(\omega)}
\end{align}
For each spherical harmonics basis $l,m$, we iterate over all rank $j$ tensors, and contract them with their respective $N_j$ (after applying the respective spherical harmonics coefficient as weight). Since the indices $l$ and $j$ run over the same range, we are allowed to swap them. This is identical to simply rearranging the order of terms:
\begin{align}
L\left(\omega\right) =
\sum_{l=0}^{\infty}\underbrace{\sum_{j=0}^\infty\sum_{m=-j}^{j}{L^{j,m}y^{j,m}_l}}_{=f_l}
\odot N_l(\omega)
\end{align}
For each spherical harmonics band $l$, we iterate over all spherical harmonics bases $j,m$ and weight and contract the rank $l$ tensor of each particular basis with $N_l$. We can further factorize the sum of equal rank tensors into the moment tensor $f_l$ to finally get the moment expansion of the radiance field:
\begin{align}
L\left(\omega\right) =
\sum_{l=0}^{\infty}f_l\odot N_l(\omega)
\end{align}
with
\begin{align}
f_l = \sum_{j=0}^\infty\sum_{m=-j}^{j}{L^{j,m}y^{j,m}_l}
\end{align}
For $f_0$ we get:
\begin{align}
f_0 &= L^{0,0}y^{0,0}_0 = \int_\Omega{L\left(\omega\right)\SHBC^{0,0}\left(\omega\right)\ud\omega}\frac{1}{\sqrt{4\pi}}\\
&=
\frac{1}{4\pi}\int_\Omega{L\left(\omega\right)\ud\omega} =
\frac{1}{4\pi}\phi
\end{align}
For $f_1$, we need to add $L^{1,-1}y^{1,-1}_1$, $L^{1,0}y^{1,0}_1$ and $L^{1,1}y^{1,1}_1$. Starting with $L^{1,-1}y^{1,-1}_1$, we have:
\begin{align}
L^{1,-1}y^{1,-1}_1 &= 
\int_\Omega{L\left(\omega\right)\SHBC^{1,-1}\left(\omega\right)\ud\omega}
\begin{pmatrix}\sqrt{\frac{3}{8\pi}}  \\ -\sqrt{\frac{3}{8\pi}}i \\ 0 \end{pmatrix}
\\
&= 
\frac{3}{8\pi}\int_\Omega{L\left(\omega\right)\operatorname{sin}\theta e^{-i\phi}\ud\omega}
\begin{pmatrix}1  \\ -i \\ 0 \end{pmatrix}
\end{align}
Using the trigonometric identities $e^{-i\phi} = \operatorname{cos}\phi - i\operatorname{sin}\phi$ and $e^{i\phi} = \operatorname{cos}\phi + i\operatorname{sin}\phi$, we get:
\begin{align}
L^{1,-1}y^{1,-1}_1 &= 
\begin{pmatrix}
\frac{3}{8\pi}\int_\Omega{L\left(\omega\right)\operatorname{sin}\theta\operatorname{cos}\phi\ud\omega} - \frac{3}{8\pi}i\int_\Omega{L\left(\omega\right)\operatorname{sin}\theta\operatorname{sin}\phi\ud\omega}
\\
-\frac{3}{8\pi}i\int_\Omega{L\left(\omega\right)\operatorname{sin}\theta\operatorname{cos}\phi\ud\omega} - \frac{3}{8\pi}\int_\Omega{L\left(\omega\right)\operatorname{sin}\theta\operatorname{sin}\phi\ud\omega}
\\
0
\end{pmatrix}
\end{align}
Now we use that
\begin{align}
\omega &= 
\begin{pmatrix}
\operatorname{sin}\theta\operatorname{cos}\phi
\\
\operatorname{sin}\theta\operatorname{sin}\phi
\\
\operatorname{cos}\theta
\end{pmatrix}\quad,
\end{align}
to arrive at:
\begin{align}
L^{1,-1}y^{1,-1}_1 &= 
\begin{pmatrix}
\frac{3}{8\pi}\int_\Omega{L\left(\omega\right)\omega_x\ud\omega} - \frac{3}{8\pi}i\int_\Omega{L\left(\omega\right)\omega_y\ud\omega}
\\
-\frac{3}{8\pi}i\int_\Omega{L\left(\omega\right)\omega_x\ud\omega} - \frac{3}{8\pi}\int_\Omega{L\left(\omega\right)\omega_y\ud\omega}
\\
0
\end{pmatrix}
\end{align}
Following similar procedures for $L^{1,0}y^{1,0}_1$ and $L^{1,1}y^{1,1}_1$, we get:
\begin{align}
L^{1,0}y^{1,0}_1 &= 
\begin{pmatrix}
0\\
0\\
\frac{3}{4\pi}\int_\Omega{L\left(\omega\right)\omega_z\ud\omega}
\end{pmatrix}
\end{align}
and
\begin{align}
L^{1,1}y^{1,1}_1 &= 
\begin{pmatrix}
\frac{3}{8\pi}\int_\Omega{L\left(\omega\right)\omega_xi\ud\omega} + \frac{3}{8\pi}i\int_\Omega{L\left(\omega\right)\omega_y\ud\omega}
\\
\frac{3}{8\pi}i\int_\Omega{L\left(\omega\right)\omega_xi\ud\omega} + \frac{3}{8\pi}\int_\Omega{L\left(\omega\right)\omega_y\ud\omega}
\\
0
\end{pmatrix}
\end{align}
We now can put $f_1$ together. Note how all the imaginary terms cancel out:
\begin{align}
f_1 &= 
L^{1,-1}y^{1,-1}_1+L^{1,0}y^{1,0}_1+L^{1,1}y^{1,1}_1
=
\frac{3}{4\pi}
\begin{pmatrix}
\int_\Omega{L\left(\omega\right)\omega_x\ud\omega}
\\
\int_\Omega{L\left(\omega\right)\omega_y\ud\omega}
\\
\int_\Omega{L\left(\omega\right)\omega_z\ud\omega}
\end{pmatrix}
=
\frac{3}{4\pi}\vec{E}
\end{align}
With $f_0$ and $f_1$, we can write down the moment expansion of the radiance field truncated after the first moment, which is used for reconstruction in the diffusion approximation:
\begin{align}
\nonumber
L\left(\omega\right) &= 
f_0\odot N_0 + f_1\odot N_1
\\
\nonumber
&=
\frac{1}{4\pi}\phi + \frac{3}{4\pi}\vec{E} \odot \omega
\\
\label{eq:moment_expansion_L}
&=
\frac{1}{4\pi}\phi + \frac{3}{4\pi}\left(w \cdot \vec{E}\right)
\end{align}
The tensor $f_0$ is expressed in terms of the fluence $\phi$ and the tensor $f_1$ in terms of the flux vector $\vec{E}$. These quantities are often referred to as the zero and first moment of the radiance field respectively. These moments generalize to higher order and are denoted $L_i$. They are computed using the moment projection operator $\mu_i$:
\begin{align}
\label{eq:moment_expansion_mu}
\mu_i\left[L\right] = L_i = \int_\Omega{L\left(\omega\right)N_i\left(\omega\right)\ud\omega}
\end{align}
An important quantity for deriving flux-limited diffusion is the second moment of the radiance field, the radiative pressure tensor $L_2$. Its intuition is very similar to that of the stress tensor in continuum mechanics.

