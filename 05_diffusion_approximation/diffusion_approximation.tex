\chapter{Diffusion Approximation for Multiple Scattering in Participating Media}
%
\label{sec:diffusion_approximation}

In the previous chapter the angular variable of the radiative transfer equation was discretized into a set of coupled partial differential equations, the $P_N$-equations. Further a method has been introduced for discretizing these equations into a system of linear equations which could be solved using standard methods. Since the number of equations depends on the truncation order $N$, higher $N$ will result in increasingly larger systems. The results showed that the amount of computational resources required makes this approach very unattractive for practical rendering applications, especially for higher $N$ (see section~\ref{sec:pn_results}). This chapter recaps the diffusion approximation, a theory which is being derived from the general $P_N$-equations with a truncation order of $N=1$. It is particularly appealing because it reduces to a simple diffusion equation which can be solved very efficiently. This theory therefore is an important step in trying to derive a practical deterministic method from the spherical harmonics expansion of the radiative transfer equation.

The diffusion approximation has been introduced to computer graphics by Stam~\cite{Stam95} for dealing with multiple scattering in participating media. Therefore the theory and method for solving it is not new. However, in this thesis a complete chapter is dedicated to this theory for two reasons. First the idea of this thesis is to present a comprehensive picture on spherical harmonics methods and diffusion approximation is a very popular and important part of it. The second reason is that diffusion approximation serves as a bridge to flux-limited diffusion theory which is introduced in the next chapter.

In section~\ref{sec:sh} it has been established that spherical functions can be represented using a linear combination of spherical harmonics basis functions (see equation~\ref{eq:sh_reconstruction}). The coefficients to these basis functions are found by projection (see equation~\ref{eq:sh_projection}). Another way for representing spherical functions is the coordinate free form given by tensor calculus. This form is derived by representing the spherical harmonics basis functions themself as linear combination of products of the components of the underlying vectors which represent the coordinate system. Because this representation allows to change the coordinate system independently from the representation, it is referred to as being coordinate free. For a refresher on tensor calculus an introductory standard such as Grinfeld~\cite{Grinfeld13} is recommended. As with the spherical harmonics expansion, the coordinate free representation can be found for the radiance field as well as the individual terms of the radiative transfer equation. This is referred to as the moment expansion or multipole expansion of the radiance field and the radiative transfer equation respectively.

The moment expansion is a different way of representing the spherical harmonics expansion of the radiance field and the radiative transfer equation. Its benefit lies in its notational efficieny, especially for lower order. However, for truncation order of $N=2$ and higher, it introduces additional redundancy which makes the standard spherical harmonics expansion much better suited when deriving computational methods from the theory, such as the $P_N$-method in chapter~\ref{sec:pnmethod}. Because the diffusion approximation introduced in this chapter and flux-limited diffusion in the next chapter is related to the first order truncation $N=1$, the moment expansion is the standard way of representing the theory. For both theories, no additional redundancy is introduced and the notational efficiency allows to express the underlying intuition more easily.

The value this chapter contributes to the thesis is the derviation of the moment expansion of the radiance field in section~\ref{sec:da_moment_expansion_L}, followed by the derivation of the moment expansion of the radiative transfer equation in section~\ref{sec:da_moment_expansion_RTE}. The diffusion approximation is derived as a naiv solution to the moment closure problem outlined in section~\ref{sec:moment_closure}. In section~\ref{sec:da_solver} a very efficient solver is presented for which results are given in section~\ref{sec:da_results}. The chapter closes with a discussion and motivates the introduction of flux-limited diffusion in chapter~\ref{sec:fld}.

\section{Moment Expansion of the Radiance Field}

%In this section, the moment expansion of the radiance field is introduced. It represents the radiance field after being discetized in angular domain using spherical harmonics. The difference to the standard spherical harmonics expansion is, that writing the radiance field in terms of its moments is notationally more efficient and therefore is particularily useful for theoretical treatments like in this chapter. However, the downside is, that it introduces a lot of reduncancy, which makes working with the spherical harmonics expansion more practical for implementation purposes.

We derive the moment expansion by starting from the spherical harmonics reconstruction of the radiance field:
\begin{align}
\label{eq:fld_moment_expansion_sh_expansion}
L\left(\omega\right) =
\sum_{l=0}^{\infty}\sum_{m=-l}^{l}L^{l,m}\SHBC^{l,m}
\left(\omega\right)
\end{align}
with $L^{l,m}$ being the spherical harmonics basis function coefficients, which are found by spherical harmonics projection (equation~\ref{eq:sh_projection}).

The major step in the derivation of the moment expansion is the replacement of the spherical harmonics basis function $\SHBC^{l,m}$ by a sum of tensor contractions:
\begin{align}
\label{eq:fld_moment_expansion_sum}
\SHBC^{l,m}(\omega) =
\sum_{j=0}^\infty{y^{l,m}_j\odot N_j(\omega)}
\end{align}
The operator symbol $\odot$ denotes a tensor contraction. In this instance, it is a sum over products of the individual components of the tensors $y^{l,m}_j$ and $N_j$. Each contraction therefore collapses into a scalar value. These values are then added up over all moments $j$. The tensor $N_j$ is a tensor of rank $j$ with $N_0=1$ and it is constructed by a sequence of outer products of the direction vector $\omega$:
\begin{align}
N_j\left(\omega\right)
=N_{k_1, k_2, ..., k_{j-1}, k_j} 
=\omega_{k_1}\omega_{k_2}...\omega_{k_{j-1}}\omega_{k_j} 
\end{align}
The index set $\{k_1, k_2, ..., k_{j-1}, k_j\}$ identifies specific tensor components, with each index running over all components of $\omega$ ($k_i \in \{x, y, z\}$).

The tensors $y^{l,m}_j$ are found by expanding $\SHBC^{l,m}(\omega)$ into a sum of tensor components of $N_j$. The factors to these components can be easily extracted and constitute the components of $y^{l,m}_j$. The expansion can be done by using the following non-recursive definition of the Legendre-polynomial which is derived using the multiple-angle formula\footnote{\url{http://mathworld.wolfram.com/Multiple-AngleFormulas.html}}:
\begin{align}
P^{lm}(\theta, \phi) = \operatorname{sin}^m(\phi)\sum_j^{\lfloor\frac{l-m}{2}\rfloor}a^{lmj}\operatorname{cos}^{l-m-2j}(\theta)
\qquad \text{for } m \ge 0
\end{align}
Inserting this into the definition of the spherical harmonics basis function $\SHBC^{l,m}$ (see equation~\ref{eq:sh_definition_C}) gives:
\begin{align}
\SHBC^{l,m}(\theta, \phi) = C^{l,m}\left(e^{i\phi}\operatorname{sin}\left(\phi\right)\right)^m \sum_j^{\lfloor\frac{l-m}{2}\rfloor}a^{lmj}\operatorname{cos}^{l-m-2j}(\theta)
\qquad \text{for } m \ge 0
\end{align}
As a next step we express $\SHBC^{l,m}$ in terms of unit direction vector $\omega$, instead of spherical coordinates. This is done by using the identity $e^{i\phi}\operatorname{sin}\left(\phi\right) = \omega_x + i\omega_y$ and $\operatorname{cos}(\theta) = \omega_z$:
\begin{align}
\SHBC^{l,m}(\omega) &=  C^{l,m}\left( \omega_x + i\omega_y\right)^m \sum_j^{\lfloor\frac{l-m}{2}\rfloor}a^{lmj}\omega_z^{l-m-2j}
&\qquad \text{for } m \ge 0
\\
&= \left(-1\right)^m\overline{\SHBC^{l,\abs{m}}}
&\qquad \text{for } m < 0
\end{align}
Expanding the expression above for concrete values of $l$ and $m$ will give a sequence of terms, which involve components of $N_j$, multiplied with some factors. By seeing the expanded formula as a tensor contraction, the components of $y^{l,m}_j$ can be easily extracted. For $l=0$ and $m=0$, we get:
\begin{align}
\SHBC^{0,0}(\omega) = C^{{0,0}}a^{{0,0,0}} = \frac{1}{\sqrt{4\pi}}N_0\quad,
\end{align}
from wich we can infer
\begin{align}
y^{0,0}_0 = \frac{1}{\sqrt{4\pi}}
\end{align}
For $l=1$ we get:
\begin{align}
\SHBC^{1,0}(\omega) &= C^{{1,0}}a^{{1,0,0}}\omega_z = \sqrt{\frac{3}{4\pi}}\omega_z
\\
\SHBC^{1,1}(\omega) &= C^{{1,1}}a^{{1,1,0}}\omega_x+C^{{1,1}}a^{{1,1,0}}\omega_yi = -\sqrt{\frac{3}{8\pi}}\omega_x - \sqrt{\frac{3}{8\pi}}\omega_yi
\\
\SHBC^{1,-1}(\omega) &= -\overline{\SHBC^{1,1}}(\omega)
= \sqrt{\frac{3}{8\pi}}\omega_x - \sqrt{\frac{3}{8\pi}}\omega_yi
\end{align}
From which we can infer:
\begin{align}
y^{1,-1}_1 = \begin{pmatrix}\sqrt{\frac{3}{8\pi}}  \\ -\sqrt{\frac{3}{8\pi}}i \\ 0 \end{pmatrix}
\qquad
y^{1,0}_1 = \begin{pmatrix}0  \\ 0 \\ \sqrt{\frac{3}{4\pi}} \end{pmatrix}
\qquad
y^{1,1}_1 = \begin{pmatrix}-\sqrt{\frac{3}{8\pi}}  \\ -\sqrt{\frac{3}{8\pi}}i \\ 0 \end{pmatrix}
\end{align}
This scheme continues and becomes more involved for higher moments. For example with $l=3$ an $m=0$ we get:
\begin{align}
\SHBC^{3,0}(\omega) &= \underbrace{C^{{3,0}}a^{{3,0,0}}}_{\in y^{3,0}_3}{\omega_z}^{3}+\underbrace{C^{{3,0}}a^{{3,0,1}}}_{\in y^{3,0}_1}\omega_z
\end{align}
Here we have a term containing ${\omega_z}^{3}$, which is an element of $N_3$ and another term containing $\omega_z$, which is an element of $N_1$. This shows how the spherical harmonics basis functions are expressed as a sum over tensor contractions of different ranks, as shown in Equation~\ref{eq:fld_moment_expansion_sum}.

Inserting equation~\ref{eq:fld_moment_expansion_sum} into equation~\ref{eq:fld_moment_expansion_sh_expansion} gives:
\begin{align}
L\left(\omega\right) =
\sum_{l=0}^{\infty}\sum_{m=-l}^{l}L^{l,m}\sum_{j=0}^\infty{y^{l,m}_j\odot N_j(\omega)}
\end{align}
After rearranging terms we get:
\begin{align}
L\left(\omega\right) =
\sum_{l=0}^{\infty}\sum_{m=-l}^{l}\sum_{j=0}^\infty{L^{l,m}y^{l,m}_j\odot N_j(\omega)}
\end{align}
For each spherical harmonics basis $l,m$, we iterate over all rank $j$ tensors, and contract them with their respective $N_j$ (after applying the respective spherical harmonics coefficient as weight). Since the indices $l$ and $j$ run over the same range, we are allowed to swap them. This is identical to simply rearranging the order of terms:
\begin{align}
L\left(\omega\right) =
\sum_{l=0}^{\infty}\underbrace{\sum_{j=0}^\infty\sum_{m=-j}^{j}{L^{j,m}y^{j,m}_l}}_{=f_l}
\odot N_l(\omega)
\end{align}
For each spherical harmonics band $l$, we iterate over all spherical harmonics bases $j,m$ and weight and contract the rank $l$ tensor of each particular basis with $N_l$. We can further factorize the sum of equal rank tensors into the moment tensor $f_l$ to finally get the moment expansion of the radiance field:
\begin{align}
L\left(\omega\right) =
\sum_{l=0}^{\infty}f_l\odot N_l(\omega)
\end{align}
with
\begin{align}
f_l = \sum_{j=0}^\infty\sum_{m=-j}^{j}{L^{j,m}y^{j,m}_l}
\end{align}
For $f_0$ we get:
\begin{align}
f_0 &= L^{0,0}y^{0,0}_0 = \int_\Omega{L\left(\omega\right)\SHBC^{0,0}\left(\omega\right)\ud\omega}\frac{1}{\sqrt{4\pi}}\\
&=
\frac{1}{4\pi}\int_\Omega{L\left(\omega\right)\ud\omega} =
\frac{1}{4\pi}\phi
\end{align}
For $f_1$, we need to add $L^{1,-1}y^{1,-1}_1$, $L^{1,0}y^{1,0}_1$ and $L^{1,1}y^{1,1}_1$. Starting with $L^{1,-1}y^{1,-1}_1$, we have:
\begin{align}
L^{1,-1}y^{1,-1}_1 &= 
\int_\Omega{L\left(\omega\right)\SHBC^{1,-1}\left(\omega\right)\ud\omega}
\begin{pmatrix}\sqrt{\frac{3}{8\pi}}  \\ -\sqrt{\frac{3}{8\pi}}i \\ 0 \end{pmatrix}
\\
&= 
\frac{3}{8\pi}\int_\Omega{L\left(\omega\right)\operatorname{sin}\theta e^{-i\phi}\ud\omega}
\begin{pmatrix}1  \\ -i \\ 0 \end{pmatrix}
\end{align}
Using the trigonometric identities $e^{-i\phi} = \operatorname{cos}\phi - i\operatorname{sin}\phi$ and $e^{i\phi} = \operatorname{cos}\phi + i\operatorname{sin}\phi$, we get:
\begin{align}
L^{1,-1}y^{1,-1}_1 &= 
\begin{pmatrix}
\frac{3}{8\pi}\int_\Omega{L\left(\omega\right)\operatorname{sin}\theta\operatorname{cos}\phi\ud\omega} - \frac{3}{8\pi}i\int_\Omega{L\left(\omega\right)\operatorname{sin}\theta\operatorname{sin}\phi\ud\omega}
\\
-\frac{3}{8\pi}i\int_\Omega{L\left(\omega\right)\operatorname{sin}\theta\operatorname{cos}\phi\ud\omega} - \frac{3}{8\pi}\int_\Omega{L\left(\omega\right)\operatorname{sin}\theta\operatorname{sin}\phi\ud\omega}
\\
0
\end{pmatrix}
\end{align}
Now we use that
\begin{align}
\omega &= 
\begin{pmatrix}
\operatorname{sin}\theta\operatorname{cos}\phi
\\
\operatorname{sin}\theta\operatorname{sin}\phi
\\
\operatorname{cos}\theta
\end{pmatrix}\quad,
\end{align}
to arrive at:
\begin{align}
L^{1,-1}y^{1,-1}_1 &= 
\begin{pmatrix}
\frac{3}{8\pi}\int_\Omega{L\left(\omega\right)\omega_x\ud\omega} - \frac{3}{8\pi}i\int_\Omega{L\left(\omega\right)\omega_y\ud\omega}
\\
-\frac{3}{8\pi}i\int_\Omega{L\left(\omega\right)\omega_x\ud\omega} - \frac{3}{8\pi}\int_\Omega{L\left(\omega\right)\omega_y\ud\omega}
\\
0
\end{pmatrix}
\end{align}
Following similar procedures for $L^{1,0}y^{1,0}_1$ and $L^{1,1}y^{1,1}_1$, we get:
\begin{align}
L^{1,0}y^{1,0}_1 &= 
\begin{pmatrix}
0\\
0\\
\frac{3}{4\pi}\int_\Omega{L\left(\omega\right)\omega_z\ud\omega}
\end{pmatrix}
\end{align}
and
\begin{align}
L^{1,1}y^{1,1}_1 &= 
\begin{pmatrix}
\frac{3}{8\pi}\int_\Omega{L\left(\omega\right)\omega_xi\ud\omega} + \frac{3}{8\pi}i\int_\Omega{L\left(\omega\right)\omega_y\ud\omega}
\\
\frac{3}{8\pi}i\int_\Omega{L\left(\omega\right)\omega_xi\ud\omega} + \frac{3}{8\pi}\int_\Omega{L\left(\omega\right)\omega_y\ud\omega}
\\
0
\end{pmatrix}
\end{align}
We now can put $f_1$ together. Note how all the imaginary terms cancel out:
\begin{align}
f_1 &= 
L^{1,-1}y^{1,-1}_1+L^{1,0}y^{1,0}_1+L^{1,1}y^{1,1}_1
=
\frac{3}{4\pi}
\begin{pmatrix}
\int_\Omega{L\left(\omega\right)\omega_x\ud\omega}
\\
\int_\Omega{L\left(\omega\right)\omega_y\ud\omega}
\\
\int_\Omega{L\left(\omega\right)\omega_z\ud\omega}
\end{pmatrix}
=
\frac{3}{4\pi}\vec{E}
\end{align}
With $f_0$ and $f_1$, we can write down the moment expansion of the radiance field truncated after the first moment, which is used for reconstruction in the diffusion approximation:
\begin{align}
\nonumber
L\left(\omega\right) &= 
f_0\odot N_0 + f_1\odot N_1
\\
\nonumber
&=
\frac{1}{4\pi}\phi + \frac{3}{4\pi}\vec{E} \odot \omega
\\
\label{eq:moment_expansion_L}
&=
\frac{1}{4\pi}\phi + \frac{3}{4\pi}\left(w \cdot \vec{E}\right)
\end{align}
The tensor $f_0$ is expressed in terms of the fluence $\phi$ and the tensor $f_1$ in terms of the flux vector $\vec{E}$. These quantities are often referred to as the zero and first moment of the radiance field respectively. These moments generalize to higher order and are denoted $L_i$. They are computed using the moment projection operator $\mu_i$:
\begin{align}
\label{eq:moment_expansion_mu}
\mu_i\left[L\right] = L_i = \int_\Omega{L\left(\omega\right)N_i\left(\omega\right)\ud\omega}
\end{align}
An important quantity for deriving flux-limited diffusion is the second moment of the radiance field, the radiative pressure tensor $L_2$. Its intuition is very similar to that of the stress tensor in continuum mechanics.
\TD{introduce radiative pressure tensor properly}

\section{Moment Expansion of the Radiative Transfer Equation}
\label{sec:da_moment_expansion_RTE}

%In the previous section, we learned about the moment expansion of the radiance field $L$, which is closely related to the spherical harmonics expansion. It differs in that it uses coordinate free tensor notation, instead of spherical harmonics basis functions. It is much better for notation and therefore useful for the discussion and presentation of the theory. This is particularily true when working with lower order moments. 

As with the radiance field, the radiative transfer equation can be developed into its moments. The result is a very concise representation of the $P_N$-equations, which are particularily expressive for lower truncation order, such as $P_1$. The derivation steps of the spherical harmonics expansion first replaced the radiance field quantity in the RTE by its truncated expansion, followed by the expansion of the individual terms of the RTE. The derivation steps of the moment expansion are similar with the only difference, that the moment expansion (equation~\ref{eq:moment_expansion_mu}) is applied to the RTE terms (equation~\ref{eq:rte}) first and the replacement is done as a second step.

The $n$-th moment of the RTE is found by applying the moment projection operator $\mu_n$ to all the terms of the RTE:
\begin{align*}
\mu_n\left[(\omega\cdot\nabla)L(\vec{x}, \omega)\right]=&\\
\mu_n\left[-\sigma_t(\vec{x})L(\vec{x}, \omega) + \sigma_s(\vec{x})\int_{\Omega}f_p(\vec{x}, \omega'\rightarrow\omega)L(\vec{x}, \omega')\ud\omega' + Q(\vec{x}, \omega)\right]
%\label{eq:rte_ani}
\end{align*}
Since $\mu_n$ is linear, the LHS gives:
\begin{align*}
\mu_n[(\omega \cdot \nabla)L(\vec{x}, \omega)] & = \int_{\Omega}{(\omega\cdot\nabla)L(\vec{x}, \omega)N_n\ud\omega}\\
							     & =\nabla\int_{S^2}{L(\vec{x}, \omega)N_n\omega\ud\omega} \\
							     & =\nabla\mu_{n+1}\left[L\right]
\end{align*}
The term $\nabla\mu_{n+1}$ is a tensor divergence. The general moment equation of the RTE for the $n$-th moment therefore is:
\begin{align}
\nonumber
\nabla\mu_{n+1}\left[L\right] =&
-\sigma_t(\vec{x})\mu_n\left[L(\vec{x}, \omega)\right]\\
\label{eq:gme}
&+\sigma_s(\vec{x})\mu_n\left[\int_{\Omega}f_p(\vec{x}, \omega'\rightarrow\omega) L(\vec{x}, \omega')\ud\omega'\right]\\
\nonumber
&+\mu_n\left[Q(\vec{x}, \omega)\right]
\end{align}
Generally, the moment equations of the RTE relate the divergence of the $n+1$ moment of the radiance field to the $n$-th moment of changes to the radiance field due to inscattering, absorption and emission. The inscattering term actually convolves the radiance field with the phase function and applies the scattering coefficient as a weighting function on top. Each moment produces a number of equations, which is identical to the number of tensor components for a tensor of the rank associated with that moment.

For the derivation of the diffusion approximation in this chapter, we expand the moment expansion of the RTE into its first two moments. Expanding equation~\ref{eq:gme} with $n=0$ gives the zero moment equation:
\begin{align}
\nonumber
\nabla\vec{E}(\vec{x})=&
-\sigma_t(\vec{x})\mu_0\left[L(\vec{x}, \omega)\right]
\\
\label{eq:p1_zero}
&+ \sigma_s(\vec{x})\mu_0\left[\int_{\Omega}f_p(\vec{x}, \omega'\rightarrow\omega)L(\vec{x}, \omega')\ud\omega'\right] 
\\
\nonumber
&+ \mu_0\left[Q(\vec{x}, \omega)\right]
\end{align}
The first moment equation is likewise found by using $n=1$ in equation~\ref{eq:gme}:
\begin{align}
\nonumber
\nabla P =&
-\sigma_t(\vec{x})\mu_1\left[L(\vec{x}, \omega)\right]
\\ &
\label{eq:p1_firstme}
+\sigma_s(\vec{x})\mu_1\left[\int_{\Omega}f_p(\vec{x}, \omega'\rightarrow\omega)L(\vec{x}, \omega')\ud\omega'\right]
\\ &
\nonumber
+ \mu_1\left[Q(\vec{x}, \omega)\right]
\end{align}
$P=\mu_2[L]$ is the radiation pressure tensor, a $3\times3$ - matrix. The divergence of a second rank tensor is defined by tensor calculus to be the vector containing the divergence of each single column vector of that tensor. Therefore, $\nabla P = \partial_i P_{ij}$.

The next step is to replace the radiance field $L$ by its two term expansion (equation~\ref{eq:moment_expansion_L}). Doing this with the zero moment expansion of the RTE (equation~\ref{eq:p1_zero}) gives:
\begin{align*}
\nabla\vec{E}\left(\vec{x}\right)=
&
-\sigma_t(\vec{x})\mu_0\left[\frac{1}{4\pi}\phi\left(\vec{x}\right) + \frac{3}{4\pi}\vec{E}\left(\vec{x}\right)\right]
\\
&
+\sigma_s(\vec{x})\mu_0\left[\int_{\Omega}f_p(\vec{x}, \omega'\rightarrow\omega)\frac{1}{4\pi}\phi(\vec{x})\ud\omega'\right]
\\
&
+\sigma_s(\vec{x})\mu_0\left[\int_{\Omega}f_p(\vec{x}, \omega'\rightarrow\omega)\frac{3}{4\pi}\vec{E}(\vec{x})\ud\omega'\right]
\\
&
\nonumber
+ Q_0\left(\vec{x}\right)
\\
=&
-\frac{1}{4\pi}\phi\sigma_t(\vec{x})\mu_0\left[1\right] - \frac{3}{4\pi}\vec{E}\left(\vec{x}\right)\sigma_t(\vec{x})\mu_1\left[1\right]
\\
&
+\frac{1}{4\pi}\phi(\vec{x})\sigma_s(\vec{x})\mu_0\left[\int_{\Omega}f_p(\vec{x}, \omega'\rightarrow\omega)\ud\omega'\right]
\\
&
+\frac{3}{4\pi}\vec{E}(\vec{x})\sigma_s(\vec{x})\mu_1\left[1\right]\int_{\Omega}f_p(\vec{x}, \omega'\rightarrow\omega)\omega'\ud\omega'
\\
&
+ Q_0\left(\vec{x}\right)
\\
=&
-\phi(\vec{x})\sigma_t(\vec{x})
+\phi(\vec{x})\sigma_s(\vec{x})
+Q_0\left(\vec{x}\right)
\\
=&
-\phi(\vec{x})\sigma_a(\vec{x})
+Q_0\left(\vec{x}\right)
\end{align*}
Here we used that $\mu_1[1] = 0$ and that the phase function is normalized, which results in:
\begin{align*}
\mu_0\left[\int_{\Omega}f_p(\vec{x}, \omega'\rightarrow\omega)\ud\omega'\right] = \mu_0\left[1\right] = 4\pi
\end{align*}
We likewise replace the radiance field with its two moment expansion in the first moment expansion of the RTE (equation~\ref{eq:p1_firstme}):
\begin{align*}
\nabla P =&
-\sigma_t(\vec{x})\mu_1\left[\frac{1}{4\pi}\phi\left(\vec{x}\right) + \frac{3}{4\pi}\vec{E}\left(\vec{x}\right)\right]
\\&
+\sigma_s(\vec{x})\mu_1\left[\int_{\Omega}f_p(\vec{x}, \omega'\rightarrow\omega)\left(\frac{1}{4\pi}\phi\left(\vec{x}\right) + \frac{3}{4\pi}\vec{E}\left(\vec{x}\right)\right)\ud\omega'\right]
\\&
\nonumber
+ Q_1\left(\vec{x}\right)
\\
=&
-\sigma_t(\vec{x})\mu_1\left[\frac{1}{4\pi}\phi\left(\vec{x}\right)\right]
-\sigma_t(\vec{x})\mu_1\left[\frac{3}{4\pi}\vec{E}\left(\vec{x}\right)\right]
\\
&
+\frac{1}{4\pi}\phi\left(\vec{x}\right)\sigma_s(\vec{x})\mu_1\left[\int_{\Omega}f_p(\vec{x}, \omega'\rightarrow\omega)\ud\omega'\right]
\\
&
+\frac{3}{4\pi}\vec{E}\left(\vec{x}\right)\sigma_s(\vec{x})\mu_1\left[\int_{\Omega}f_p(\vec{x}, \omega'\rightarrow\omega)\omega\ud\omega'\right]
\\
&
+Q_1\left(\vec{x}\right)
\\
=&
-\frac{1}{4\pi}\phi\left(\vec{x}\right)\sigma_t(\vec{x})\mu_1\left[1\right]
-\frac{3}{4\pi}\vec{E}\left(\vec{x}\right)\sigma_t(\vec{x})\mu_2\left[1\right]
\\
&
+\frac{1}{4\pi}\phi\left(\vec{x}\right)\sigma_s(\vec{x})\mu_1\left[1\right]
\\
&
+\frac{3}{4\pi}\vec{E}\left(\vec{x}\right)\sigma_s(\vec{x})\mu_1\left[\mu_1\left[f\right]\right]
\\
&
+Q_1\left(\vec{x}\right)
\\
=&
-\frac{3}{4\pi}\vec{E}\left(\vec{x}\right)\sigma_t(\vec{x})\mu_2\left[1\right]
+\frac{3}{4\pi}\vec{E}\left(\vec{x}\right)g\sigma_s(\vec{x})\mu_2\left[1\right]
\\
=&
\left(-\sigma_t(\vec{x})\mathbf{I} + g\sigma_s(\vec{x})\mathbf{I}\right)\vec{E}\left(\vec{x}\right)
+Q_1\left(\vec{x}\right)
\\
=&
-\sigma_t'(\vec{x})\vec{E}\left(\vec{x}\right)
+Q_1\left(\vec{x}\right)
\end{align*}
Here we used that
\begin{align*}
\mu_2[1] = \frac{4\pi}{3}\mathbf{I}
\end{align*}
and we further used that the first moment of a phase function, which only depends on the angle between incident and outgoing direction is its mean cosine $g$ (see equation~\ref{eq:foundations_mean_cosine}).

The quantity $\sigma_t'$ is called the reduced extinction coefficient and it is defined as
\begin{align*}
\sigma_t' = \sigma_t - g\sigma_s
\end{align*}
With that we have the two term expansion of the radiative transfer equation
\begin{align}
\label{eq:me_zero}
\nabla\vec{E}\left(\vec{x}\right)&=
-\phi(\vec{x})\sigma_a(\vec{x})
+Q_0\left(\vec{x}\right)
\\
\label{eq:me_first}
\nabla P\left(\vec{x}\right) &= -\sigma_t'(\vec{x})\vec{E}\left(\vec{x}\right)
+Q_1\left(\vec{x}\right)
\end{align}
These equations are the direct counterpart to the $P_1$-equations in moment expansion form. The diffusion approximation can be derived by taking the first moment equations and inserting them into the zero moment equations by substituting $\vec{E}$. The same step can be carried out for the $P_1$-equations (using spherical harmonics coefficients). However, the notation of the moment expansion is much more expressive, which is why this form is much more popular in the literature.

We start by resolving equation~\ref{eq:me_first} for $\vec{E}$:
\begin{align}
\label{eq:me_first_resolved_E}
\vec{E}\left(\vec{x}\right) =
-\frac{1}{\sigma_t'\left(\vec{x}\right)}
\left(
\nabla P\left(\vec{x}\right)
-Q_1\left(\vec{x}\right)
\right)
\end{align}
Then we take equation~\ref{eq:me_first_resolved_E} and use it to substitute $\vec{E}$ in equation~\ref{eq:me_zero}. This way we eliminate the unknown $\vec{E}$ and get a single scalar partial differential equation
\begin{align}
\nabla
\left(
-\frac{1}{\sigma_t'\left(\vec{x}\right)}
\left(
\nabla P\left(\vec{x}\right)
-Q_1\left(\vec{x}\right)
\right)
\right)&=
-\phi(\vec{x})\sigma_a(\vec{x})
+Q_0\left(\vec{x}\right)
\end{align}
which we can further rearrange into
\begin{align}
\label{eq:general_diffusion_equation}
\nabla
\left(
-\frac{1}{\sigma_t'\left(\vec{x}\right)}
\nabla P\left(\vec{x}\right)
\right)&=
-\phi(\vec{x})\sigma_a(\vec{x})
+Q_0\left(\vec{x}\right)
+\nabla Q_1\left(\vec{x}\right)
\end{align}
The result comes very close to the popular diffusion approximation equation. A single unknown remains: the second moment of the radiance field $P$, which is called the radiative pressure tensor. Resolving, or better approximating, this unknown is what leads to a rich variety of methods and theories, including flux-limited diffusion, which will be introduced in chapter~\ref{sec:fld}.





%\int_{\Omega}f_p(\vec{x}, \omega'\rightarrow\omega)
%

%+\sigma_s(\vec{x})\mu_1\left[\int_{\Omega}f_p(\vec{x}, \omega'\rightarrow\omega)\frac{3}{4\pi}\sigma_t(\vec{x})\vec{E}\left(\vec{x}\right)\right)\ud\omega'\right]

\section{Moment Closure Problem and Isotropic Closure}
\label{sec:moment_closure}

The diffusion approximation theories are derived from truncating the moment expansion of the RTE after its first moment and eliminating the unknown $\vec{E}$ by inserting the first moment equations into the zero moment equation to arrive at a single partial differential equation, which contains the zero moment $\phi$ and the second moment $P$ of the radiance field as unknowns. In order to be able to solve this partial differential equation for $\phi$, it has to be closed by eliminating the second moment $P$ as an unknown. Finding or making an educated guess for the closure $P$ is called the moment closure problem. In this section, we will outline the closure for the classical diffusion approximation and revise it in chapter~\ref{sec:fld} to introduce more sophisticated closures.

When trying to make an educated guess for $P$, there is actually some information available which could be used: the lower moments. The zero moment $\phi$ expresses the power at $\vec{x}$. That is, the total amount of energy arriving from all directions. With that, the radiance field can be factorized into the total power $\phi$, and its distribution over solid angle $\hat{L}$:
\begin{align*}
L(\vec{x}, \omega) = \hat{L}(\vec{x}, \omega)\phi(\vec{x})
\end{align*}

The total power $\phi$ is an unknown, which we do not seek to eliminate. Therefore, we more specifically are looking to find an educated guess for how this power is distributed over solid angle. This is given by the normalized radiance field $\hat{L}$. It is a probability distribution and its first moment is called the normalized flux $\widehat{\vec{E}}$ and will become important later with more advanced diffusion theories:
\begin{align*}
\widehat{\vec{E}}(\vec{x}) = \hat{L}_1(\vec{x}) = \int_{\Omega}\frac{L(\vec{x}, \omega)}{\phi(\vec{x})}\omega\ud\omega = \frac{1}{\phi(\vec{x})}\int_{\Omega}L(\vec{x}, \omega)\omega\ud\omega = \frac{\vec{E}(\vec{x})}{\phi(\vec{x})}
\end{align*}
Seperating the radiance field into power and its distribution over solid angle, allows us to factorize its second moment, the unknown $P$, in the same fashion:
\begin{align*}
P(\vec{x}) &=
\int_{\Omega}{\hat{L}(\vec{x}, \omega)\phi(\vec{x})N_2\ud\omega}\\
&= \int_{\Omega}{\hat{L}(\vec{x}, \omega)N_2\ud\omega}\phi(\vec{x})\\
&= \hat{L}_2(\vec{x}, \omega)\phi(\vec{x})
\end{align*}
The second moment of the radiance field is expressed as the second moment of the angular radiance distribution, multiplied by the power $\phi$ at $\vec{x}$. This means, that finding a closure at the end boils down to finding or estimating a spherical distribution function. The second moment of that estimated distribution is called the Eddington tensor~$T$:
\begin{align}
\label{eq:eddington_tensor}
T(\vec{x}) \approx \hat{L}_2(\vec{x}, \omega) =
\frac{P(\vec{x})}{\phi(\vec{x})} =
\widehat{P}(\vec{x})
\end{align}
The Eddington tensor is used to distribute the known power $\phi$ over angle and approximate $P$:
\begin{align*}
T(\vec{x})\phi(\vec{x}) \approx 
P(\vec{x})
\end{align*}
The simple key assumption behind the classical diffusion approximation is, that the radiance field $L$ is constant and does not change for different angles $\omega$:
\begin{align*}
L(\vec{x}, \omega) = L(\vec{x})
\end{align*}
This assumption allows finding an estimate $T$ for the second moment of the radiance distribution. We start by assuming unit power $\phi = 1$:
\begin{align*}
1 = \phi\left(\vec{x}\right) = \int_\Omega{L\left(\vec{x}, \omega\right)\ud\omega}
\end{align*}
Since $L$ is constant over solid angle ($L(\vec{x}, \omega)=L(\vec{x})$), we can pull it out of the integral to get:
\begin{align*}
1 = \phi\left(\vec{x}\right) = L\left(\vec{x}\right)\int_\Omega{\ud\omega}\implies L\left(\vec{x}\right) = \frac{1}{4\pi}
\end{align*}
We use our constant radiance field, to compute the components of the radiative pressure tensor:
\begin{align*}
P\left(\vec{x}\right) 
= \int_\Omega{L\left(\vec{x}\right)N_2\ud\omega}
= \frac{1}{4\pi}\int_\Omega{N_2\ud\omega}
= \frac{1}{4\pi}\frac{4\pi}{3}\mathbf{I}
= \frac{1}{3}\mathbf{I}
\end{align*}
Since we assumed unit power $\phi(\vec{x})=1$ without loss of generality, we have:
\begin{align*}
T(\vec{x})\phi(\vec{x}) = P(\vec{x}) \implies T(\vec{x})=\frac{1}{3}\mathbf{I}
\end{align*}
Inserting $P=T\phi=\frac{1}{3}\mathbf{I}\phi$ into equation~\ref{eq:general_diffusion_equation}, gives the diffusion equation for anisotropic emission sources
\begin{align}
\label{eq:diffusion_equation_anisotropic_Q}
\nabla
\left(
-\frac{1}{3\sigma_t'\left(\vec{x}\right)}
\nabla \phi\left(\vec{x}\right)
\right)&=
-\phi(\vec{x})\sigma_a(\vec{x})
+Q_0\left(\vec{x}\right)
+\nabla Q_1\left(\vec{x}\right)
\end{align}
and becomes the popular diffusion approximation formula for isotropic emission sources ($Q_1=\vec{0}$):
\begin{align}
\label{eq:diffusion_equation_anisotropic_Q}
\nabla
\left(
D
\nabla \phi\left(\vec{x}\right)
\right)&=
-\phi(\vec{x})\sigma_a(\vec{x})
+Q_0\left(\vec{x}\right)
\end{align}
with the classic linear diffusion coefficient
\begin{align}
D=-\frac{1}{3\sigma_t'\left(\vec{x}\right)}
\label{eq:da_D}
\end{align}
Inserting the isotropic second moment into the first moment expansion of the radiative transfer equation (equation~\ref{eq:me_first_resolved_E}) gives us an expression for the flux-vector as defined by the diffusion approximation theory:
\begin{align}
\label{eq:diffusion_ficks_law}
\vec{E}\left(\vec{x}\right) \approx -D\left(\vec{x}\right)\nabla\phi\left(\vec{x}\right)
\end{align}
This gives some intuition about the diffusion approximation. Instead of depending on the global radiance field $L$, the flux-vector depends on the local gradient of the fluence to determine the transport of radiative energy at position $\vec{x}$.
\begin{figure}[h]
\centering
\missingfigure{Visualize the idea behind diffusion approximation: to approximate global transport by using a local gradient of the fluence. mention ficks law of diffusion}
\caption{Some caption}
\label{fig:da_moment_problem_flux_as_fluence_gradient}
\end{figure}
Solving the diffusion equation is very straightforward and in the next section a very efficient solver is introduced based on the multigrid method.



\todo[inline]{Explain/Visualize the idea behind diffusion approximation: to approximate global transport by using a local gradient of the fluence. mention ficks law of diffusion}
\begin{align}
\label{eq:diffusion_ficks_law}
\vec{E}\left(\vec{x}\right) \approx -D\left(\vec{x}\right)\nabla\phi\left(\vec{x}\right)
\end{align}
\todo[inline]{Discuss how the assumption of isotropy is valid in highly scattering participating media}

\section{Multigrid solver}
\label{sec:da_solver}

In this section a multigrid solver for the diffusion approximation is developed. Multigrid schemes are the most efficient iterative methods for solving diffusion-type equations. The core idea is to propagate the error on a coarse grid during each iteration and take that propagated error into account when doing an iteration step on the original high resolution grid. Particularly useful is the fact that the multigrid method can be employed on any system of linear equations as long as a so called restriction and interpolation operator is being provided. 

The automatic discretization machinery which was developed as part of the $P_N$-solver in section~\ref{sec:pn_solver} takes arbitrary systems of potentially coupled, linear partial differencial equations as input. The diffusion equation which has been derived in the previous section is a linear scalar partial differencial equation and therefore can be fed into the $P_N$-solver from section~\ref{sec:pn_solver} to produce valid stencil code (independent of mesh resolution).

The idea behind the multigrid method is to accelerate the convergence by using a coarse grid on which information propagates much faster throughout the spatial domain. The coarse solution is then used for updating the original grid with finer resolution in each iteration.
\begin{figure}[h]
\centering
\missingfigure{multigrid mesh setup}
\caption{Some caption}
\label{fig:da_solver_multigrid_mesh}
\end{figure}

Quantities on the original fine mesh are denoted with the subscript $F$. We therefore have for the original system of equations:
\begin{align}
\nonumber
A_F\vec{u}_F = \vec{b}_F
\end{align}
The multigrid method is an iterative method. This is expressed with a superscript index $i$ which denotes the current multigrid iteration. The first step in the multigrid cycle is to partially converge the solution by applying a number of iterations of an iterative method, such as Conjugate-Gradient or Gauss-Seidel. The partially converged solution ín the first multigrid cycle is denoted $\vec{u}_F^i$. Then the residual is computed on the fine mesh:
\begin{align}
\nonumber
\vec{r}_F^i = \vec{b}_F-A_F\vec{u}_F^i
\end{align}
The idea behind multigrid is to use the coarse grid to smooth out the error and use the result as a correction for the next iterations on the fine mesh. To facilitate this, the residual on the fine mesh is first transferred onto the coarse mesh. This is done by downsampling the fine mesh residual (often also called interpolation). This operation can be represented as a non-symmetric matrix $M$. Quantities on the coarse mesh are denoted with the subscript $C$:
\begin{align}
\nonumber
\vec{r}_C^i = M_{F\rightarrow C} \vec{r}_F^i
\end{align}
After bringing the residual onto the coarse mesh, the next step is to solve the correction equation to full convergence. The correction equation is derived by looking at the system on the coarse mesh using the partially converged solution $\vec{u}_C^i$:
\begin{align}
\nonumber
A_C\vec{u}_C^i = \vec{b}_C
\end{align}
Rearranging this equation allows us to use the downsampled residual as the right hand side vector:
\begin{align}
\nonumber
\vec{b}_C - A_C\vec{u}_C^i = \vec{r}_C^i
\end{align}
We substitute $\vec{b}_C$ with $A_C\vec{u}_C=\vec{b}_C$ (using the fully converged yet unknown solution on the coarse grid $\vec{u}_C$) and get:
\begin{align}
A_C\vec{u}_C - A_C\vec{u}_C^i &= \vec{r}_C^i
\nonumber
\\
A_C\left(\vec{u}_C-\vec{u}_C^i\right) &= \vec{r}_C^i
\nonumber
\\
A_C\vec{e}_C^i &= \vec{r}_C^i
\label{eq:da_correction_equation}
\end{align}
The quantity $\vec{e}_C^i$ is the unknown error between the final solution $\vec{u}_C$ and its partially converged approximation $\vec{u}_C^i$. Equation~\ref{eq:da_correction_equation} is called the correction equation and is solved to full convergence. The computed error is upsampled onto the finer grid using the upsampling operator $M_{C\rightarrow F}$:
\begin{align}
\vec{e}_F^i = M_{C\rightarrow F}\vec{e}_C^i
\nonumber
\end{align}
Using the upsampled error we can apply a correction step to the solution on the fine grid:
\begin{align}
\vec{e}_F^i &= \vec{u}_F - \vec{u}_F^i
\nonumber
\\
\vec{u}_F^{i+1} &=  \vec{e}_F^i + \vec{u}_F^i
\nonumber
\end{align}
This process is then repeated by starting a new multigrid iteration with the corrected approximation $\vec{u}_F^{i+1}$. In summary a single multigrid iteration consists of the following steps:
\begin{enumerate}
\item Solve the original system on the fine grid using a small number of iteration steps to bring the solution $\vec{u}_F^i$ to partial convergence.
\item Compute the residual $\vec{r}_F^i=\vec{b}_F - A_F\vec{u}_F^i$ on the fine grid and transfer it onto the coarse grid $\vec{r}_C^i = M_{F\rightarrow C}\vec{r}_F^i$.
\item Solve the correction equation $A_C\vec{e}_C^i = \vec{r}_C^i$ for $\vec{e}_C^0$ to full convergence on the coarse grid.
\item Transfer the result to the fine grid $\vec{e}_F^i = M_{C\rightarrow F}\vec{e}_C^i$.
\item Use $\vec{e}_F^i$ to apply the correction $\vec{u}_F^{i+1} =  \vec{e}_F^i + \vec{u}_F^i$.
\end{enumerate}
What has been outlined here is a two-level multigrid step which could be extended to more levels by applying the same idea recursively to the coarse mesh solve. This is often done so that the linear system of the coarsest level is so small that a direct method for solving it can be applied very efficiently. This results in the characteristic V-cycle shown in figure~\ref{fig:da_solver_multigrid_vcycles}.
\begin{figure}[h]
\centering
\missingfigure{multigrid v-cycle}
\caption{Some caption}
\label{fig:da_solver_multigrid_vcycles}
\end{figure}

Building the upsampling and downsampling matrices $M_{C\rightarrow F}$ and $M_{F\rightarrow C}$ is very similar to the problem of interpolating discretized variables at specific staggered grid locations for the automatic discretization in section~\ref{sec:pn_stencil_gen} (see figure~\ref{fig:pn_discretization_interpolation}). In fact, the \emph{PNSystem} class presented in section~\ref{sec:pn_framework} has been extended to facilitate the automatic generation of the upsampling and downsampling operator matrices from a given discretization and a given vector of coefficients with their respective staggered grid locations. In order for the interpolation weights to not vary per voxel, the fine grid has to be exactly of double resolution. Further the number of multigrid levels defines the number of subdivisions required and imposes additional constraints on the resolution. The resolution has to be of power of two in order support the maximum number of multigrid levels possible.
\begin{figure}[h]
\centering
\missingfigure{interpolation matrix M, matrix layout, interpolation, throwing away coefficients}
\caption{Some caption}
\label{fig:da_solver_multigrid_M}
\end{figure}



\section{Results}
\label{sec:da_results}

\TD{Compare against P1 from PN method and show that results are identical}
\TD{mention vacuum and division by zero}

\section{Discussion}
\label{sec:da_discussion}