\phantomsection
\chapter*{Abstract}
\addcontentsline{toc}{chapter}{Abstract}

%
%\vspace{-2cm}
%
This thesis introduces novel numerical methods for light transport simulation in computer graphics which are based on the spherical harmonics discretization of angular variables in the governing equations.

Light transport simulation is required for generating realistic images from virtual scenes and therefore ubiquitous in digital media today. Also, an increasing presence of light transport simulation in the domain of scientific visualization can be observed. This can be explained by the fact that the visual system of a human being is optimized for reading real world cues, like surface smoothness from specularity, or a shape from indirect illumination. The early stage of realistic image generation in computer graphics relied on ad-hoc techniques and heuristics due to limitations in computing power. With the growth in computing power, a steady increase in model accuracy and physical rigor has become feasible.

Today, the accuracy of human perception defines the domain of light transport simulation in computer graphics and remains the single aspect that sets it apart from transport simulation in other domains, such as nuclear science or astrophysics. With increased conformity towards traditional physics in terms of models, terminology and methodology, the opportunity increased to draw inspiration from other fields, their large body of ideas and methods. In fact, an extensive amount of research in light transport simulation for computer graphics is concerned with the translation, reframing, and enhancement of those techniques by exploiting limited requirements in accuracy.

In other domains, such as nuclear sciences, astrophysics or medical physics research, the spherical harmonics based angular discretization of the underlying light transport equations led to a large variety of methods. One of those methods, the diffusion approximation, has been introduced to graphics with great success in the past. A reason for the lack of application of other methods in graphics lies in the complexity of the underlying theory, which further results in large and unwieldy systems of equations. Nevertheless, it is important to study them thoroughly and understand their characteristics with regard to applications in computer graphics, where deterministic methods play a significant role and have led to major advances in the past.

This thesis gives an extensive derivation of the underlying theory of spherical harmonic methods and presents it in the context of light transport simulation for computer graphics. A deep and holistic view of the spherical harmonics methods is taken and their application to computer graphics is investigated. The results are two novel deterministic methods for solving light transport problems in graphics.

First, the thesis introduces the $P_N$-Method to the field of computer graphics. This method allows for spherical harmonics expansion up to an arbitrary level. To handle the complex equations, a fully automated discretization framework is devised as part of a solver that is driven by a computer algebra representation of the underlying transport equations. This novel strategy may inspire new methods for unsolved numerical problems in computer graphics. The chapter further presents a rendering framework and shows how to integrate the solution from any method based on spherical harmonics representation by separating direct from indirect illumination. The results show solutions for a typical volume rendering problem and comparisons against groundtruth solutions for canonical problems.

The $P_N$-method is based on the generalized spherical harmonics expansion up to arbitrary level. While expanding to higher order produces more accurate results, the method also becomes more and more unpractical due a system of equations which increases in size with higher order. In the fourth chapter, this thesis therefore examines flux-limited diffusion and introduces it to the field of computer graphics. It is based on the idea of truncating the spherical harmonics expansion at a very low order and reducing the error by making assumptions about the higher order contribution of light. The result is a novel method, which allows for the interactive computation of multiple scattering in highly scattering, heterogeneous participating media\mydash a challenging problem for standard Monte-Carlo methods.