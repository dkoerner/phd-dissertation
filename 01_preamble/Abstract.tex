\chapter*{Abstract}
%
%\vspace{-2cm}
%
This thesis introduces novel numerical methods for light transport simulation in computer graphics which are based on the spherical harmonics discretization of angular variables in the governing equations.

Light transport simulation is required for generating realistic images from virtual scenes and therefore ubiquitous in digital media today. Also an increasing presence of light transport simulation in the domain of scientific visualization can be observed. This can be explained by the fact that the visual system of a human being is optimized for reading real world cues, like surface smoothness from specularity, or a shape from indirect illumination. The early stage of realistic image generation in computer graphics relied on ad-hoc techniques and heuristics due to limitations in computing power. With the growth of compute capabilities, a steady increase in model accuracy and physical rigor was visible.

Today, the accuracy of human perception defines the domain of light transport simulation in computer graphics and remains the single aspect, which sets it apart from transport simulation in other domains, such as nuclear science or astrophysics. With increased conformity towards traditional physics in terms of models, terminology and methodology, the opportunity arose to draw inspiration from other fields, their large body of ideas and methods. In fact, an extensive amount of research in light transport simulation for computer graphics is concerned with the translation, reframing and enhancement of those techniques by exploiting limited requirements in accuracy.

In other domains, such as nuclear sciences, astrophysics or medical physics research, the spherical harmonics based angular discretization of the underlying light transport equations led to a large variety of methods. One of those methods, the diffusion approximation, has been introduced to graphics with great success in the past. One reason for the lack of application of other methods in graphics lies in the complexity of the underlying theory, which further results in large and unwieldy systems of equations. Nevertheless, it is important to study them thoroughly and understand their characteristics with regard to applications in computer graphics, where deterministic methods play a significant role and have led to major advances in the past.

For the first time, an extensive derivation of the underlying theory of spherical harmonic methods is given and presented in the context of light transport simulation for computer graphics. This thesis takes a deep and holistic view at the spherical harmonics methods and investigates their application to computer graphics. The results are two novel deterministic methods for solving light transport problems in graphics. The first new method allows for spherical harmonics expansion up to arbitrary level. To handle the complex equations, a fully automated discretization framework as part of the solver is devised, which is driven by a computer algebra representation of the underlying transport equations. This novel strategy may inspire new methods for unsolved numerical problems in computer graphics. The second new method, which is introduced in this thesis, is derived by looking at the shortcomings of diffusion approximation and investigate an improvement from other domains called flux-limited diffusion. The result is a novel method, which allows for interactive computation of multiple scattering in highly scattering, heterogeneous participating media---a challenging problem for standard Monte-Carlo methods.