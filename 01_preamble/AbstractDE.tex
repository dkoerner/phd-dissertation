\selectlanguage{ngerman}
%
\chapter*{Zusammenfassung (German Abstract)}
% \vspace{0cm}
%
%  Ä   Ö   Ü   ä   ö   ü   ß
%
Diese Dissertation führt neue numerische Methoden zur Lichttransportsimulation in der Computergraphik ein, welche auf der spherical harmonics Diskretisierung der Winkelvariablen basieren.

Lichttransportsimulation ist in den digitalen Medien allgegenwärtig und notwendig, um realistische Bilder virtueller Szenen zu generieren. Auch wird sie zunehmend in der wissenschaftlichen Visualisierung eingesetzt. Dies erklärt sich dadurch, dass die visuelle Wahrnehmung des Menschen auf das Lesen realer Umgebungen trainiert ist, wie zum Beispiel die Glattheit einer glänzenden Oberfläche, oder die Form, welche durch indirekte Beleuchtung entsteht.

Die frühen Anfänge der Generierung von realistischen Bildern in der Computergraphik stützten sich aufgrund der limitierten Rechenleistung auf ad-hoc Techniken und Heuristiken. Mit schneller werdenden Computern kam ein Wachstum an Modelgenauigkeit und ein Anspruch an physikalische Korrektheit einher.

Gegenwärtig wird das Feld der Lichttransportsimulation in der Computergraphik durch die Grenzen der menschlichen Wahrnehmung definiert. Dies ist ein Aspekt, welcher eine Abgrenzung zur Simulation von Lichttransport in anderen Fachbereichen, wie Astrophysik oder Kernphysik ermöglicht. Durch die Angleichung an die traditionelle Physik in Bezug auf Modelle, Terminologie und Methodik, ergibt sich zunehmend die Gelegenheit, Ideen und Methoden anderer Fachbereiche in der Computergraphik anzuwenden.

Tatsächlich befasst sich ein großer Teil der Forschung zur Lichttransportsimulation in der Computergraphik mit der Übersetzung, Neuausrichtung und Verbesserung der Methoden, unter Berücksichtigung der schwächeren Anforderungen an Genauigkeit.

In anderen Fachbereichen wie der Kernphysik, Astrophysik oder Medizinforschung, haben spherical harmonics basierte Winkeldiskretisierungen, der zugrunde liegenden Transportgleichung, zu einer großen Zahl an Methoden geführt. Eine dieser Methoden, die Diffusionsapproximation, wurde in der Vergangenheit mit großem Erfolg in der Computergraphik eingeführt. Der feehlende Einsatz von weiteren Methoden in der Computergraphik liegt darin, dass die zugrunde liegende Theorie sehr komplex ist und in große, sperrige Gleichungssysteme mündet. Denoch ist es wichtig, diese Methoden gründlich zu untersuchen und ihre Eigenschaften in Hinblick auf die Anwendung auf bestehende Probleme in der Computergraphik zu verstehen, in der deterministische Methoden von Bedeutung sind.

Erstmalig wird eine ausführliche Herleitung der Theorie der Spherical Harmonics Methoden im Kontext der Lichttransportsimulation in der Computergraphik angegeben. Diese Dissertation wirft einen tiefgehenden und ganzheitlichen Blick auf die gesamte Klasse an Spherical Harmonics Methoden und erforscht ihre Anwendung in der Computergraphik. Die Ergebnisse sind zwei neue deterministische Methoden zur Lösung von Lichttransportproblemen. Die erste Methode erlaubt eine Spherical Harmonics Expansion beliebiger Ordnung. Um die komplexen Gleichungen zu bewältigen, wurde ein vollautomatisches Diskretisierungsframework entworfen, welches auf einer Computer Algebra Repräsentation der Transportgleichungen aufbaut. Die zweite Methode, welche in dieser Dissertation eingeführt wird, basiert auf einer Schwäche der Diffusionsapproximation und der Untersuchung einer Verbesserung, der Flux-limitierten Diffusion. Das Ergebnis ist eine neue Methode, welche die Berechnung der Mehrfachstreuung in einem stark-streuendem, heterogenem Medium erlaubt---ein schwieriges Problem für gewöhnliche Monte-Carlo Methoden.
%
\selectlanguage{english}