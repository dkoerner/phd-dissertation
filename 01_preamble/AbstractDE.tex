\selectlanguage{ngerman}
%
\chapter*{Zusammenfassung (German Abstract)}
% \vspace{0cm}
%
%  Ä   Ö   Ü   ä   ö   ü   ß
%
Diese Dissertation führt neue numerische Methoden zur Lichttransportsimulation in der Computergraphik ein, welche auf der spherical harmonics Diskretizierung der Winkelvariablen basieren.

Lichttransportsimulation ist notwendig, um realistische Bilder virtueller Szenen zu generieren und daher in den digitalen Medien heute allgegenwärtig. Darüber hinaus wird es zunehmend in der wissenschaftlichen Visualisierung eingesetzt. Dies erklärt sich dadurch, dass die visuelle Wahrnehmung des Menschen auf das Lesen realer Umgebungen trainiert ist, wie zum Beispiel die Glattheit einer Oberfläche vom Glanz, oder die Form von indirekter Beleuchtung.

Die frühen Anfänge der Generierung von realistischen Bildern in der Computergraphik stützten sich aufgrund der limitierten Rechenleistung auf ad-hoc Techniken und Heuristiken. Mit schneller werdenden Computern kam auch ein Wachstum an Modelgenauigkeit und ein Anspruch an physikalische Korrektheit einher.

Heute wird das Feld der Lichttransportsimulation in der Computergraphik durch die Grenzen der menschlichen Wahrnehmung definiert. Dies bleibt der einzige Aspekt, welcher eine Abgrenzung zur Simulation von Lichttransport in anderen Fachbereichen, wie Astrophysik oder Kernphysik ermöglicht. Durch die Angleichung an die traditionelle Physik im Bezug auf Modelle, Terminologie und Methodik, ergibt sich immer mehr die Gelegenheit, Ideen und Methoden aus anderen Fachbereichen in der Computergraphik anzuwenden.

Tatsächlich ist ein großer Teil der Forschung zur Lichttransportsimulation in der Computergraphik mit der Übersetzung, Neuausrichtung und Verbesserung solcher Methoden, unter Berücksichtigung der schwächeren Anforderungen an Genauigkeit, befasst.

In anderen Fachbereichen wie der Kernphysik, Astrophysik oder Medizinforschung, haben spherical harmonics basierte Winkeldiskretisierungen der zugrunde liegenden Transportgleichung zu einer großen Zahl an Methoden geführt. Eine dieser Methoden, die Diffusionsapproximation, wurde in der Vergangenheit mit großem Erfolg in der Computergraphik eingeführt. Der Grund, warum weitere Methoden noch nicht in der Computergraphik eingesetzt wurden, ist der, dass die zugrunde liegende Theorie sehr komplex ist und in große, sperrige Gleichungssysteme mündet. Denoch ist es wichtig, diese Methoden gründlichen zu untersuchen und ihre Eigenschaften in Hinblick auf den Einsatz für Probleme in der Computergraphik zu verstehen, wo deterministische Methoden eine bedeutende Rolle spielen.

Erstmalig wird eine ausführliche Herleitung der zugrunde liegenden Theorie der Spherical Harmonics Methoden im Kontext der Lichttransportsimulation in der Computergraphik angegeben. Diese Dissertation wirft einen tiefgehenden und ganzheitlichen Blick auf die gesamte Klasse an Spherical Harmonics Methoden und erforscht ihre Anwendung in der Computergraphik. Die Ergebnisse sind zwei neue deterministische Methoden zur Lösung von Lichttransportproblemen. Die erste neue Methode erlaubt eine Spherical Harmonics expansion beliebiger Ordnung. Um die komplexen Gleichungen zu bewältigen, wurde ein vollautomatisches Diskretisierungsframework entworfen, welches auf einer Computer Algebra Repräsentation der zugrunde liegenden Transportgleichungen aufbaut. Die zweite neue Methode, welcher in dieser Dissertation eingeführt wird, basiert auf einer Schwäche der Diffusionsapproximation und der Untersuchung einer Verbesserungen, der Flux-limitierten Diffusion. Das Ergebnis ist eine neue Methode, welche die Berechnung der Mehrfachstreuung in einem stark-streuendem, heterogenem Medium erlaubt---einem schwierigem Problem für standard Monte-Carlo Methoden.
%
\selectlanguage{english}