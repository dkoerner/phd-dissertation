\selectlanguage{ngerman}
%
\phantomsection
\chapter*{Zusammenfassung (German Abstract)}
\addcontentsline{toc}{chapter}{Zusammenfassung (German Abstract)
}
% \vspace{0cm}
%
%  Ä   Ö   Ü   ä   ö   ü   ß
%
Diese Dissertation führt neue numerische Methoden zur Lichttransportsimulation in der Computergraphik ein, welche auf der Diskretisierung der Winkelvariablen durch Kugeloberflächenfunktionen in den zugrunde liegenden Gleichungen basieren.

Lichttransportsimulation ist in den digitalen Medien allgegenwärtig und notwendig, um realistische Bilder virtueller Szenen zu generieren. Auch wird sie zunehmend in der wissenschaftlichen Visualisierung eingesetzt. Dies erklärt sich dadurch, dass die visuelle Wahrnehmung des Menschen auf das Lesen realer Umgebungen trainiert ist, wie zum Beispiel die Glattheit einer glänzenden Oberfläche, oder die Form, welche durch indirekte Beleuchtung entsteht. Die frühen Anfänge der Generierung von realistischen Bildern in der Computergraphik stützten sich aufgrund der limitierten Rechenleistung auf ad-hoc Techniken und Heuristiken. Mit schneller werdenden Computern kam ein Wachstum an Modellgenauigkeit und ein Anspruch an physikalische Korrektheit einher.

Gegenwärtig wird das Feld der Lichttransportsimulation in der Computergraphik durch die Grenzen der menschlichen Wahrnehmung definiert. Dies ist ein Aspekt, welcher eine Abgrenzung zur Simulation von Lichttransport in anderen Fachbereichen wie Astrophysik oder Kernphysik darstellt. Durch die Angleichung an die traditionelle Physik in Bezug auf Modelle, Terminologie und Methodik, ergibt sich zunehmend die Gelegenheit, Ideen und Methoden anderer Fachbereiche in der Computergraphik anzuwenden. Tatsächlich befasst sich ein großer Teil der Forschung zur Lichttransportsimulation in der Computergraphik mit der Übersetzung, Neuausrichtung und Verbesserung der Methoden unter Berücksichtigung der schwächeren Anforderungen an die Genauigkeit.

In anderen Fachbereichen wie der Kernphysik, Astrophysik oder Medizinforschung haben Winkeldiskretisierungen mit Kugeloberflächenfunktionen zu einer großen Zahl an Methoden geführt. Eine dieser Methoden, die Diffusionsapproximation, wurde in der Vergangenheit mit großem Erfolg in der Computergraphik eingeführt. Der Grund, warum andere Methode nicht in der Computergraphik eingesetzt werden, liegt unter anderem daran, dass die zugrunde liegende Theorie sehr komplex ist und in große, sperrige Gleichungssysteme mündet. Denoch ist es wichtig, diese Methoden gründlich zu untersuchen und ihre Eigenschaften in Hinblick auf die Anwendung auf bestehende Probleme in der Computergraphik zu verstehen, wo deterministische Methoden bisher von großer Bedeutung waren.

Diese Dissertation gibt eine ausführliche Herleitung der Theorie der Lichttransportsimulation auf Basis von Diskretisierung mit Kugeloberflächenfunktionen und präsentiert sie im Kontext der Lichttransportsimulation für Computergraphik. Sie wirft einen tiefgehenden und ganzheitlichen Blick auf die gesamte Klasse an Methoden, welche diese Diskretisierung nutzen und untersucht ihre Anwendung in der Computergraphik. Das Ergebnis sind zwei neue deterministische Methoden zur Lösung von Lichttransportproblemen in der Computergraphik.

Zuerst führt diese Dissertation die $P_N$-Methode in die Computergraphik ein. Diese Methode erlaubt eine Entwicklung in Kugeloberflächenfunktionen bis zu beliebig hoher Ordnung. Um die komplexen Gleichungen zu verarbeiten, ist ein vollautomatisches Diskretisierungsframwork als Teil eines Lösers entwickelt worden, welcher von einer Computer-Algebra-Repräsentation der zugrunde liegendenden Gleichungen gespeist wird. Diese neue Strategie inspiriert möglicherweise neue Methoden für ungelöste numerische Probleme in der Computergraphik. Das Kapitel präsentiert zudem ein Rendering-framework und zeigt, wie Lösungen von Methoden, welche auf der Entwicklung in Kugeloberflächenfunktionen basieren, für das Rendering verwendet werden können. Es werden Ergebnisse für typische Volume-Rendering-Probleme gezeigt, als auch Vergleiche mit Referenzlösungen kanonischer Probleme.

Die $P_N$-Methode basiert auf der Entwicklung in Kugeloberflächenfunktionen. Während die Entwicklung in höhere Ordnungen bessere Ergebnisse produziert, wird die Methode gleichzeit unpraktischer wegen eines immer größeren Gleichungssystems, das zu lösen ist. Im vierten Kapitel wird daher die flux-limitierte Diffusion untersucht und in die Computergraphik eingeführt. Sie basiert auf der Idee, die Entwicklung in Kugeloberflächenfunktionen auf eine niedrige Ordnung zu beschränken und den Approximationsfehler mit Hilfe von Annahmen über die Lichtverteilung in der höheren Ordnung zu reduzieren. Das Ergebnis ist eine neue Methode, welche erlaubt, die Mehrfachstreuung in einem hochstreuenden partizipierenden Medium zu berechnen\mydash eine Herausforderung für gewöhnliche Monte-Carlo Methoden.
%
\selectlanguage{english}