\section{Complex-valued $P_N$-equations}
\label{sec:pn_cvalued}

Equipped with key properties of the spherical harmonics expansion, the complex-valued $P_N$-equations are derived in this section. The derivation follows two basic steps, which are executed separately on each term of the radiative transfer equation. The first step is to substitute each angular dependent radiative transfer quantity, such as the radiance field $L$, by its (truncated) spherical harmonics projection (equation~\ref{eq:sh_projection}). The second step is to apply the spherical harmonics projection to each term of the radiative transfer equation. This results in terms, which will depend on the spherical harmonics indices $l,m$ and therefore form a system of equations. The size of this system is driven by the truncation order $N$. The higher the value $N$, the higher frequencies in angular domain are taken into account for an approximation of the radiative transport.
% --------------------------------------------------------------
\subsubsection*{Transport Term}
The transport term of the RTE is given as
\begin{align*}
(\vec{\omega}\cdot\nabla)L(\vec{x}, \vec{\omega})
\end{align*}
Replacing $L$ with its spherical harmonics expansion results in:
\begin{align*}
\left(\vec{\omega}\cdot\nabla\right)
\left(
\sum_{l,m}
{
L^{l,m}\left(\vec{x}\right )
\SHBC^{l,m}\left(\vec{\omega}\right)
}
\right)
\end{align*}
Next, the whole term is projected into spherical harmonics, which means a multiplication with $\overline{\SHBC^{l'm'}}$ and integration over solid angle:
\begin{align*}
\int_\Omega
{
\overline{Y^{l'm'}}(\vec{\omega}\cdot\nabla)
\sum_{l,m}
{
L^{l,m}\left(\vec{x}\right)
\SHBC^{l,m}\left(\vec{\omega}\right)
}
}
\ud\vec{\omega}
\end{align*}
The spatial derivative can be extracted to get:
\begin{align*}
\nabla\cdot\int_\Omega
{
\vec{\omega}\overline{\SHBC^{l'm'}}
\sum_{l,m}
{
L^{l,m}\left(\vec{x}\right)
\SHBC^{l,m}\left(\vec{\omega}\right)
}
\ud\vec{\omega}
}
\end{align*}
When applying the recursion relation (equation~\ref{eq:recursive_relation}) the following will be shown:
\begin{equation*}
\resizebox{1.0\hsize}{!}{$
\begin{pmatrix}
\frac{1}{2}\partial_x\\
\frac{i}{2}\partial_y\\
\partial_z
\end{pmatrix}
\cdot
\int_\Omega
\begin{pmatrix}
\ c^{l'-1, m'-1}\overline{\SHBC^{l'-1,m'-1}} - d^{l'+1, m'-1}\overline{\SHBC^{l'+1,m'-1}} - e^{l'-1, m'+1}\overline{\SHBC^{l'-1,m'+1}} + f^{l'+1, m'+1}\overline{\SHBC^{l'+1,m'+1}}\\
-c^{l'-1, m'-1}\overline{\SHBC^{l'-1,m'-1}} + d^{l'+1, m'-1}\overline{\SHBC^{l'+1,m'-1}} - e^{l'-1, m'+1}\overline{\SHBC^{l'-1,m'+1}} + f^{l'+1, m'+1}\overline{\SHBC^{l'+1,m'+1}} \\
a^{l'-1, m'}\overline{\SHBC^{l'-1,m'}}+b^{l'+1, m'}\overline{\SHBC^{l'+1,m'}}
\end{pmatrix}
\sum_{l,m}{
L^{l,m}\left(\vec{x}\right )\SHBC^{l,m}\left(\vec{\omega}\right)
}
\ud\vec{\omega}
$}
\end{equation*}
Integrating the vector term over solid angle, can be expressed as seperate solid angle integrals over each component. These integrals over a sum of terms are split into seperate integrals:
\begin{align*}
\begin{pmatrix}
\frac{1}{2}\partial_x\\
\frac{i}{2}\partial_y\\
\partial_z
\end{pmatrix}
\cdot
\begin{pmatrix}
\ c^{l'-1, m'-1}\sum_{l,m}{L^{l,m}\left(\vec{x}\right )\int_\Omega{\overline{\SHBC^{l'-1,m'-1}}\left(\vec{\omega}\right)\SHBC^{l,m}\left(\vec{\omega}\right)\ud\vec{\omega}}} \quad - \quad ...\\
-c^{l'-1, m'-1}\sum_{l,m}{L^{l,m}\left(\vec{x}\right )\int_\Omega{\overline{\SHBC^{l'-1,m'-1}}\left(\vec{\omega}\right)\SHBC^{l,m}\left(\vec{\omega}\right)\ud\vec{\omega}}} \quad + \quad ... \\
a^{l'-1, m'}\sum_{l,m}{L^{l,m}\left(\vec{x}\right )\int_\Omega{\overline{\SHBC^{l'-1,m'}}\left(\vec{\omega}\right)\SHBC^{l,m}\left(\vec{\omega}\right)\ud\vec{\omega}}} \quad + \quad ...
\end{pmatrix}
\end{align*}
Applying the orthogonality property to the solid angle integrals will select specific $l,m$ in each term:
\begin{equation*}
\resizebox{1.0\hsize}{!}{$
\begin{pmatrix}
\frac{1}{2}\partial_x\\
\frac{i}{2}\partial_y\\
\partial_z
\end{pmatrix}
\cdot
\begin{pmatrix}
\ c^{l-1, m-1}L^{l-1,m-1} - d^{l+1, m-1}L^{l+1,m-1} - e^{l-1, m+1}L^{l-1,m+1} + f^{l+1, m+1}L^{l+1,m+1}\\
-c^{l-1, m-1}L^{l-1,m-1} + d^{l+1, m-1}L^{l+1,m-1} - e^{l-1, m+1}L^{l-1,m+1} + f^{l+1, m+1}L^{l+1,m+1} \\
a^{l-1, m}L^{l-1,m}+b^{l+1, m}L^{l+1,m}
\end{pmatrix}
$}
\end{equation*}
Which gives the final moment equation for the transport term:
\begin{align}
&
\frac{1}{2}c^{l-1, m-1}\partial_x L^{l-1,m-1}
-\frac{1}{2}d^{l+1, m-1}\partial_x L^{l+1,m-1}
-\frac{1}{2}e^{l-1, m+1}\partial_x L^{l-1,m+1}
\nonumber
\\
&
+\frac{1}{2}f^{l+1, m+1}\partial_x L^{l+1,m+1}
-\frac{i}{2}c^{l-1, m-1}\partial_y L^{l-1,m-1}
+\frac{i}{2}d^{l+1, m-1}\partial_y L^{l+1,m-1}
\nonumber
\\
&
-\frac{i}{2}e^{l-1, m+1}\partial_y L^{l-1,m+1}
+\frac{i}{2}f^{l+1, m+1}\partial_y L^{l+1,m+1}
+
a^{l-1, m}\partial_z L^{l-1,m}+b^{l+1, m}\partial_z L^{l+1,m}
\label{eq:sh_complex_transport}
\end{align}

% --------------------------------------------------------------
\subsubsection*{Scattering Term}

The scattering term in the RTE is given as:
\begin{align*}
\sigma_s(\vec{x})\int_{\Omega}\phase(\vec{x}, \vec{\omega}'\cdot\vec{\omega})L(\vec{x}, \vec{\omega}')\ud\vec{\omega}'
\end{align*}
The phase function, used in isotropic scattering medium, only depends on the angle between incident and outgoing direction, and therefore is rotationally symmetric around the pole defining axis. This property allows to define a rotation $R(\vec{\omega})$, which rotates the phase function, such that the pole axis aligns with the outgoing direction vector $\vec{\omega}$. This rotation is expressed as:
\begin{align*}
\rho_{R(\vec{\omega})}(\phase)
\ ,
\end{align*}
which can be implemented by applying the inverse rotation $R(\vec{\omega})^{-1}$ to the arguments of $\phase$. With this rotated phase function, the integral of the scattering operator can be expressed as a convolution:
\begin{align}
\int_{\Omega'}\phase(\vec{x}, \vec{\omega}'\cdot\vec{\omega})L(\vec{x}, \vec{\omega}')\ud\vec{\omega}'
&=
\int_{\Omega'}{\rho_{R(\vec{\omega})}(\phase)(\operatorname{cos}\theta')L(\vec{x}, \vec{\omega}')\ud\vec{\omega}'}
\nonumber\\
&= \langle L,  \rho_{R(\vec{\omega})}(\phase)\rangle
\label{eq:complex_scatt_conv}
\end{align}
As the inner product integral of the convolution is evaluated, the phase function rotates along with the argument $\vec{\omega}$.

Now, the spherical harmonics expansions of $L$ and $\phase$ (equation~\ref{eq:sh_exp_phase}) in the definition for the inner product of the convolution (equation~\ref{eq:complex_scatt_conv}) is used:
\begin{align*}
\langle L,  \rho_{R(\vec{\omega})}(\phase)\rangle = \left < \sum_{l,m}{L^{l,m}(\vec{x}) \SHBC^{l,m}}, \rho_{R(\vec{\omega})}\left ( \sum_{l}{\phase^{l0}\SHBC^{l0}} \right )\right>
\end{align*}
Due to linearity of the inner product operator, the non-angular dependent parts of the expansions can be pulled out:
\begin{align*}
\langle L,  \rho_{R(\vec{\omega})}(\phase)\rangle
&=
\sum_{l,m}
{
L^{l,m}(\vec{x})
\left<
\SHBC^{l,m},
\rho_{R(\vec{\omega})}
\left(\sum_l{\phase^{l0} \SHBC^{l0}}\right)
\right>
}
\end{align*}
and further:
\begin{align*}
\langle L,  \rho_{R(\vec{\omega})}(\phase)\rangle
&=
\sum_{l'}
{
\sum_{l,m}
{
\phase^{l'0}L^{l,m}(\vec{x})
\left<\SHBC^{l,m}, \rho_{R(\vec{\omega})}\left( \SHBC^{l'0} \right)\right>
}
}
\end{align*}
The rotation $\rho_{R(\vec{\omega})}$ of a function with frequency $l$ results in a function of frequency $l$ (equation~\ref{eq:sh_rotation}). In addition the spherical harmonics basis functions $\SHBC^{l,m}$ are orthogonal. Therefore following applies:
\begin{align*}
\left<
\SHBC^{l,m}, \rho_{R(\vec{\omega})}\left(\SHBC^{l'm'}\right)
\right> = 0       \qquad    \text{for all}\ \ l\ne l' 
\end{align*}
which further simplifies the inner product integral to:
\begin{align*}
\langle L,  \rho_{R(\vec{\omega})}(\phase)\rangle
&=
\sum_{l,m}
{
\phase^{l0}L^{l,m}(\vec{x})
\left<
\SHBC^{l,m}, \rho_{R(\vec{\omega})}\left(\SHBC^{l0} \right )
\right>
}
\end{align*}
What remains to be resolved is the inner product. The spherical harmonics basis functions $ \SHBC^{l,m}$ are eigenfunctions of the inner product integral operator in the equation above (Dai~\cite{Dai13}):
\begin{align*}
\left<
\SHBC^{l,m}, \rho_{R(\vec{\omega})}\left ( \SHBC^{l0} \right )\right> = \lambda_l \SHBC^{l,m}
\end{align*}
with
\begin{align*}
\lambda_l=\sqrt{\frac{4\pi}{2l+1}}
\end{align*}
Replacing the inner product leads to:
\begin{align*}
\langle L,  \rho_{R(\vec{\omega})}(p)\rangle
&=
\sum_{l,m}
{
\lambda_l
\phase^{l0}L^{l,m}(\vec{x})
\SHBC^{l,m}
}
\end{align*}
This allows to express the scattering term using SH expansions of phase function $p$ and radiance field $L$:
\begin{align*}
\sigma_s(\vec{x})\int_{\Omega}\phase(\vec{x}, \vec{\omega}'\cdot\vec{\omega})L(\vec{x}, \vec{\omega}')\ud\vec{\omega}'
&=
\sigma_s(\vec{x})\langle L,  \rho_{R(\vec{\omega})}(p)\rangle
\\
&=
\sigma_s(\vec{x})
\sum_{l,m}
{
\lambda_l
\phase^{l0}L^{l,m}(\vec{x})
\SHBC^{l,m}
}
\end{align*}
However, a spherical harmonics expansion of the term itsself has not been done yet. It is still a scalar function that depends on direction $\vec{\omega}$. The scattering term is projected into spherical harmonics, by multiplying with $\overline {\SHBC^{l'm'}}$ and integrating over solid angle $\vec{\omega}$. Further all factors will be pulled out, which do not depend on $\vec{\omega}$ and apply the SH orthogonality property to arrive at the scattering term of the complex-valued $P_N$-equations:
\begin{align}
&
\int_{\Omega}
{
\overline{\SHBC^{l'm'}}(\vec{\omega})
\sigma_s(\vec{x})
\sum_{l,m}
{
\lambda_l
\phase^{l0}L^{l,m}(\vec{x})
\SHBC^{l,m}\left(\vec{\omega}\right)
}
\ud\vec{\omega}
}
\nonumber\\
=&
\lambda_l
\sigma_s(\vec{x})
\phase^{l0}L^{l,m}(\vec{x})
\sum_{l,m}
{
\int_{\Omega}
{
\overline{\SHBC^{l'm'}}(\vec{\omega})
\SHBC^{l,m}\left(\vec{\omega}\right)
\ud\vec{\omega}
}
}
\nonumber\\
=&
\lambda_l
\sigma_s(\vec{x})
\phase^{l0}L^{l,m}(\vec{x})
\sum_{l,m}
{
\delta_{ll'}\delta_{mm'}
}
\nonumber\\
=&
\lambda_l
\sigma_s(\vec{x})
\phase^{l0}L^{l,m}(\vec{x})
\label{eq:sh_complex_scattering}
\end{align}



% ------------------------------------------------------------
\subsubsection*{Collision and Emission Term}

The collision term of the RTE is given as:
\begin{align*}
-\sigma_t\left(\vec{x}\right)L\left(\vec{x}, \vec{\omega}\right)
\end{align*}
The radiance field $L$ is replaced with its spherical harmonics expansion:
\begin{align*}
-\sigma_t\left(\vec{x}\right)
\sum_{l,m}
{
L^{l,m}\left(\vec{x}\right )\SHBC^{l,m}\left(\vec{\omega}\right)
}
\end{align*}
Multiplying with $\overline{\SHBC^{l'm'}}$ and integrating over solid angle, after pulling some factors out of the integral and applying the orthogonality identity (equation~\ref{eq:sh_orthogonality}) results in:
\begin{align}
&-\sigma_t\left(\vec{x}\right)\sum_{l,m}{L^{l,m}\left(\vec{x}\right )\int_\Omega\overline{\SHBC^{l'm'}}\left(\vec{\omega}\right)\SHBC^{l,m}\left(\vec{\omega}\right)\ud\vec{\omega}}
\nonumber
\\
&= -\sigma_t\left(\vec{x}\right)\sum_{l,m}{L^{l,m}\left(\vec{x}\right )\delta_{ll'}\delta_{mm'}}
\nonumber
\\
&= -\sigma_t\left(\vec{x}\right)L^{l,m}\left(\vec{x}\right )
\label{eq:sh_complex_collision}
\end{align}
The derivation of the SH projected term is equal to the derivation of the projected collision term. After replacing the emission field with its SH projection and multiplying the term with the conjugate complex of $\SHBC$ results after application of the orthogonality property in:
\begin{align}
Q^{l,m}\left(\vec{x}, \vec{\omega}\right)
\label{eq:sh_complex_emission}
\end{align}


% --------------------------------------------------------------
\subsubsection*{Final Equation}

The results are the complex-valued $P_N$-equations, after putting all the projected terms together (equation~\ref{eq:sh_complex_transport},~\ref{eq:sh_complex_collision},~\ref{eq:sh_complex_scattering} and~\ref{eq:sh_complex_emission}):
\begin{align}
&
\frac{1}{2}c^{l-1, m-1}\partial_x L^{l-1,m-1}
-\frac{1}{2}d^{l+1, m-1}\partial_x L^{l+1,m-1}
-\frac{1}{2}e^{l-1, m+1}\partial_x L^{l-1,m+1}
\nonumber
\\
&
+\frac{1}{2}f^{l+1, m+1}\partial_x L^{l+1,m+1}
-\frac{i}{2}c^{l-1, m-1}\partial_y L^{l-1,m-1}
+\frac{i}{2}d^{l+1, m-1}\partial_y L^{l+1,m-1}
\nonumber
\\
&
-\frac{i}{2}e^{l-1, m+1}\partial_y L^{l-1,m+1}
+\frac{i}{2}f^{l+1, m+1}\partial_y L^{l+1,m+1}
+a^{l-1, m}\partial_z L^{l-1,m}
+b^{l+1, m}\partial_z L^{l+1,m}
\nonumber
\\
&
=
-\sigma_t\left(\vec{x}\right)L^{l,m}\left(\vec{x}\right )
+
\lambda_l
\sigma_s(\vec{x})
\phase^{l0}L^{l,m}(\vec{x}) + Q^{l,m}\left(\vec{x}, \vec{\omega}\right)
\label{eq:sh_pne_complex}
\end{align}