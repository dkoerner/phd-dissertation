\section{Results}
\label{sec:pn_results}

After outlining the $P_N$-solver in section~\ref{sec:pn_solver} and its integration into a rendering framework in the previous section, this section will give and discuss results produced by the system which was developed as part of this thesis.

\subsubsection{Point Source Problem}

The point source problem is a canonical problem which is well suited to validate the method and assess its accuracy. This is because the problem is simple and an analytical solution exists to allow groundtruth comparison. The medium is a homogeneous medium with infinite extent. This is approximated for the solver by a unit cube. The emission term is an ideal unit power point light with isotropic emission in angular domain and a delta function in spatial domain. This delta function is discretized into the finite difference grid by distributing its unit power over the volume of a single voxel:
\begin{align}
Q_{ijk} = \frac{1}{h_xh_yh_z}
\end{align}
Here the index {ijk} refers to the voxel in which the point light is located.

Note that here the radiance field is not seperated as outlined in section~\ref{sec:pn_rendering_integration}. The emission field represents only the discretized light source and therefore the resulting discretized radiance field represents all light transport, including single scattered light.

For this problem an analytical solution exists for the fluence, that is the integral of the radiance field over solid angle. The correct analytical solution is given by D'Eon et al.~\cite{dEon11}. They also introduce a very accurate approximation, called the Grosjean solution, which is simpler to evaluate and always convergent. The fluence is identical to the zero coefficient of the spherical harmonics expansion of the radiance field ($L^{0,0}$) which we therefore can compare directly with the Grosjean solution.
\begin{figure}[h]
\centering
\missingfigure{pointsource plots PN}
\caption{Some caption}
\label{fig:pn_results_pointsource_1}
\end{figure}
Figure~\ref{fig:pn_results_pointsource_1} compares the solution for $P_1$, $P_2$, $P_3$, $P_4$, $P_5$ against the Grosjean solution. The figure shows that with higher $N$, the $P_N$-solution becomes more accurate. What is also noticeable is that with increasing truncation order $N$, the $P_N$-solution oscillates around the exact solution and approaches it with alternating under- and overestimation, especially near the point source. Even truncation order will overestimate and odd truncation order will produce an underestimate. 

\subsubsection{Procedural Volume}

A more practical example which uses the rendering 

\TD{nebulae. explain setup and indirect illumination and vacuum region. maybe discuss how it was build? show solutions for p1-p5 and give tables with performance and memory statistics. give all pictures including results.}
\TD{convergence behaviour for different minimum thresholds.}

\subsubsection{Checkerboard Problem}
