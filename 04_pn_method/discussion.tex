\section{Discussion}

The solver, which has been introduced in this chapter produces results that are in good agreement with radiative transfer ground truth solutions. This has been demonstrated with the point source problem in section~\ref{sec:pn_results_pointsource}. The results for the checkerboard problem (section~\ref{sec:pn_results_checkerboard}) are in very good agreement with results from deterministic methods in other fields. This further demonstrates the correct derivation of the real-valued $P_N$-equations and implementation of the solver, developed as part of this thesis.

The procedural cloud results in section~\ref{sec:pn_results_clouds} show that in order to capture light in the visually important indirect regions of the dataset, truncation orders of $N=5$ or higher are necessary. The computation time and memory requirements in this case are signficiant as seen in table~\ref{tab:results_cloud}. These are very heavy requirements, which will limit the use of the $P_N$-method in practical applications. The next chapter covers the Diffusion Approximation, which has been introduced to computer graphics by Stam~\cite{Stam95}. It is derived as a degenerated case of the $P_1$-equations and reduces to an uncomplicated diffusion equation.