\chapter{$P_N$-Method}
%
\label{sec:pnmethod}

The $P_N$-method is a deterministic method for computing light transport in a participating media. It is based on the spherical harmonics discretization of the radiative transfer equation and its related quantities in angular domain. The method has its origins in astrophysics and is also popular in nuclear sciences and medical science. It came up in the field of computer graphics only in so far that Kajia~\cite{Kajiya84} brushed over the theory, but did not give any details on implementation or how to solve it. In fact, as Max~\cite{Max95} pointed out, it is not clear if Kajiya succeeded at all at applying the method, as all of the results in his paper were produced with a simpler method\footnote{\cite{Max95} p.4: \emph{``[...] Kajiya attempted to solve these equations for the case of isotropic scattering, but it is unclear whether he succeeded, since all the pictures in \cite{Kajiya84} were produced by the simpler 'slab' method.[...]"}}. This doubt is further strengthened by results in this thesis, which show that a straight forward finite difference discretization of the $P_N$-equations produces unuseable results, due to oscillation artifacts in the solution. 
\TD{reference for pn method in astrophysics, neutron transport, medical sciences etc.}

In this chapter, we will give a thorough derivation of the $P_N$-theory. The first section introduces the spherical harmonics expansion and important properties. Then the complex-valued $P_N$-equations are derived in section~\ref{sec:pn_cvalued}. The fact, that the radiance field is positive in every direction can be used to cut the number of unknowns in half. This is done by using the real-valued $P_N$-equations. In section~\ref{sec:pn_rvalued}, a very compact form of the real-valued $P_N$-equations is derived, which has not been given anywhere else in the literature before. After a short note on two-dimensional problems in section~\ref{sec:pn_2d}, The chapter continues by introducing a new method for solving the $P_N$-equations (section~\ref{sec:pn_solver}) and closes by discussing its integration into a rendering framework and giving results (section~\ref{sec:pn_results}).

% ============================================================
\section{Spherical Harmonics}

Spherical harmonics are the Fourier series for the sphere and therefore as equally popular for spherical problems. They are often derived as eigensolutions to the surface Laplacian, which would be like developing the Fourier series as eigensolutions of the operator $(d/dx)^2$ on a finite line with the boundary conditions that $y$ and $dy/dx$ match at the two ends (see Riley et al.~\cite{Riley2006}). The result of this derivation are complex-valued functions $\SHBC$ which are defined as
\begin{align}
\label{eq:sh_definition}
\SHBC^{l,m}(\theta, \phi)=
\begin{cases}
C^{l,m}e^{im\phi}P^{l,m}\left(\operatorname{cos}\theta\right), & \text{for $m\ge0$}\\
\left(-1\right)^m\overline{\SHBC^{l\left|m\right|}}(\theta, \phi), & \text{for $m<0$}
\end{cases}
\quad ,
\end{align}
where $P^{l,m}$ are the associated Legendre polynomials. While those can be defined in many different ways, the most numerically robust way to evaluate them is by using the following set of recurrence relations (see Press et al.~\cite{Press07}):
\begin{align}
P^{0,0}\left(\operatorname{cos}\theta\right) &=
1
\  ,
\nonumber
\\
P^{m,m}\left(\operatorname{cos}\theta\right) &=
\left(2m-1\right)!!\left(1-\operatorname{cos}^2\theta\right)^\frac{m}{2}
\  ,
\nonumber
\\
P^{m+1,m}\left(\operatorname{cos}\theta\right) &=
\operatorname{cos}\theta\left(2m+1\right)P^{m,m}\left(\operatorname{cos}\theta\right)
\ 
\nonumber
\\
P^{l,m}\left(\operatorname{cos}\theta\right) &=
\frac{\operatorname{cos}\theta\left(2l-1\right)}{l-m}
P^{l-1,m}\left(\operatorname{cos}\theta\right)
-
\frac{l+m-1}{l-m}
P^{l-2,m}\left(\operatorname{cos}\theta\right)
\  .
\label{eq:sh_Plm}
\end{align}
The factor $C^{l,m}$ in equation~\ref{eq:sh_definition} is defined as
\begin{align}
\label{eq:sh_definition_C}
C^{l,m}=(-1)^m\sqrt{\frac{2l+1}{4\pi}\frac{(l-m)!}{(l+m)!}}
\end{align}
Since spherical harmonics are used so widely, their definition can vary and one has to be careful when comparing them across the literature. This concerns the $(-1)^m$ factor, which is called the Condon-Shortley phase. Sometimes, this factor is part of the definition of the associated Legendre polynomial $P^{l,m}$ and therefore does not appear in $C^{l,m}$. More importantly, the definition of $C^{l,m}$ depends on how the spherical harmonics are expected to be normalized.

\TD{figure of sh visualization}

\subsubsection*{Normalization and Orthogonality}

The coefficient $C^{l,m}$ has been defined, such that the spherical harmonics function $\SHBC$ scales to the unit norm
\begin{align}
\label{eq:sh_normalization}
\norm{\SHBC^{l,m}} = \sqrt{\left<\SHBC^{l,m}, \SHBC^{l,m}\right>} = 1
\ .
\end{align}
Other definitions define this norm differently and therefore arrive at different $C^{l,m}$. Since the spherical harmonics functions $\SHBC$ are complex-valued functions on the unit sphere, they are elements of the Hilbert space, for which the inner product is defined as
\begin{align}
\label{eq:sh_inner_product}
\left<f,g\right> = \int_\Omega f\left(\omega\right)\overline{g}\left(\omega\right)\ud\omega
\end{align}
with $\overline{g}$ being defined as the complex conjugate of $g$.

The spherical harmonics functions form a orthogonal family and thus we have
\begin{align}
\left<\SHBC^{l_1, m_1},\SHBC^{l_2, m_2}\right>
=
0
\ ,\quad\text{ for } l_1 \ne l_2 \text{ and } m_1 \ne m_2
\end{align}
The properties of normalization (equation~\ref{eq:sh_normalization}) and orthogonality (equation~\ref{eq:sh_orthogonality}) give
\begin{align}
\label{eq:sh_orthogonality}
\left<\SHBC^{l_1, m_1},\SHBC^{l_2, m_2}\right>=
\int_{\Omega} \SHBC^{l_1m_1}\left(\omega\right) \overline{\SHBC^{l_2m_2}}\left(\omega\right) \mathbf{d}\omega = \delta_{l_1m_1}\delta_{l_2m_2}
\ .
\end{align}

\subsubsection*{Projection and Reconstruction}

Because the spherical harmonics are orthonormal, a spherical function $f\left(\omega\right)$ can be projected into spherical harmonics basis function coefficients $f^{l,m}$ using the inner product from equation~\ref{eq:sh_inner_product}. This gives
\begin{align}
\label{eq:sh_projection}
f^{l, m} = \left<f,\SHBC^{l, m}\right> = 
\int_\Omega f\left(\omega\right)\overline{\SHBC^{l,m}}\left(\omega\right)\ud\omega
\ .
\end{align}
Given these coefficients, $f$ can be fully reconstructed by summing up the weighted contributions from the spherical harmonics basis functions
\begin{align}
f\left(\omega\right) = 
\sum_{l=0}^{N}
{
\sum_{m=-l}^{l}
{
f^{l,m}\left(\vec{x}\right)\SHBC^{l,m}\left(\omega\right)
}
}
\ .
\end{align}
The function can be fully reconstructed if $N=\infty$. However, as with the Fourier-expansion, it is useful to truncate the expansion at a specific $N$ (giving $P_N$-method its name). This limits the number of coefficients at the expense of introducing an approximation error by cutting off higher frequencies. Also note the use of the complex conjugate $\overline{\SHBC}$ for computing the coefficients $f^{l,m}$. This comes from the definition of the inner product (equation~\ref{eq:sh_inner_product}), which is a result of Hilber space theory.

\TD{projection operator}
\begin{align}
\label{eq:sh_projection_operator}
TODO
\end{align}
\TD{number of coefficients for truncation order N}

\subsubsection*{Frequency-invariant Rotation and Rotational Symmetry}

Applying a rotation to the argument of a spherical harmonics basis function of band $l$ can be expressed as a linear combination of spherical harmonics basis functions of the same order:
\begin{align}
\label{eq:sh_rotation}
\SHBC^{l,m}\left(R\omega\right)=\sum_{j=-l}^{l}r^{l,j}\SHBC^{l,j}\left(\omega\right)
\ .
\end{align}
This is derived from the fact that the spherical harmonics basis functions of order $l$ form an irreducible basis for the group of 3D-rotations (see Corollary $17.17$ in \cite{Hall13}). It implies that a rotation of a function represented in spherical harmonics does not introduce or loose any frequencies from or to other spherical harmonic bands and therefore is not subject to aliasing. Note that in the graphics literature spherical harmonics are often referred to as being rotationally invariant (e.g.~\cite{Wojciech08} or~\cite{Green03}), which is not correct.

The spherical harmonics projection of a rotationally symmetric function $f$ simplifies to Zonal Spherical Harmonics, which are characterized by the fact that only coefficients with $f^{l,0}$ are needed for reconstruction. This will be needed for the derivation of the $P_N$-equations. In particular, the phase function depends only on the angle between incident direction $\omega_i$ and outgoing direction $\omega_o$, which means that the phase function will be rotationally symmetric around the outgoing direction if it is fixed. 

A rotation $R(\alpha)$ of angle $\alpha$ around the pole axis is expressed in spherical harmonics as
\begin{align*}
\rho_{R(\alpha)}(\SHBC^{l,m}) = e^{-i m\alpha}\SHBC^{l,m}
\ .
\end{align*}
If a function $f$ is rotationally symmetric around the pole axis, we have:
\begin{align*}
\rho_{R(\alpha)}(f) = f
\end{align*}
and in spherical harmonics this would be:
\begin{align*}
\sum_{l,m}
{
e^{-i m\alpha}
f^{l,m}
\SHBC^{l,m} }\left(\omega\right)
=
\sum_{l,m}
{
p^{l,m}
\SHBC^{l,m}\left(\omega\right)
}
\end{align*}
By equating coefficients we get:
\begin{align*}
f^{l,m} = f^{l,m}e^{-i m\alpha}
\end{align*}
Since $e^{-i m\alpha}=1$ for all $\alpha$ only when $m=0$, we can conclude that $f^{l,m} = 0$ for all $m\ne0$. This means that for a function, which is rotationally symmetric around the pole axis, only the $m=0$ coefficients will be valid.

As mentioned earlier, this will be useful during the derivation of the $P_N$-equation and its scattering term in particular. If we fix the outgoing direction $\omega_o$ of our phase function at the north pole ($\omega_o=\vec{e}_3$), then the reconstruction requires only the spherical harmonics coefficients with $m=0$:
\begin{align}
\label{eq:sh_exp_phase}
p(\omega_i) =
\sum_l
{
p^{l0}
\SHBC^{l0}(\omega_i)
}
\end{align}

\subsubsection*{Recursive Relation}

Another important property which will be important for the derivation of the $P_N$-equations and its derivative term in particular is the following recursive relation
\begin{equation}
\label{eq:recursive_relation}
\resizebox{1.0\hsize}{!}{$\omega\overline{\SHBC^{l,m}} = \frac{1}{2}
\begin{pmatrix}
\ c^{l-1, m-1}\overline{\SHBC^{l-1,m-1}} - d^{l+1, m-1}\overline{\SHBC^{l+1,m-1}} - e^{l-1, m+1}\overline{\SHBC^{l-1,m+1}} + f^{l+1, m+1}\overline{\SHBC^{l+1,m+1}}\\
i\left(-c^{l-1, m-1}\overline{\SHBC^{l-1,m-1}} + d^{l+1, m-1}\overline{\SHBC^{l+1,m-1}} - e^{l-1, m+1}\overline{\SHBC^{l-1,m+1}} + f^{l+1, m+1}\overline{\SHBC^{l+1,m+1}}\right) \\
2\left(a^{l-1, m}\overline{\SHBC^{l-1,m}}+b^{l+1, m}\overline{\SHBC^{l+1,m}}\right)
\end{pmatrix}$}
\end{equation}
with
\begin{equation*}
\resizebox{1.0\hsize}{!}{$
a^{l,m}= \sqrt{\frac{\left(l-m+1\right)\left(l+m+1\right)}{\left(2l+1\right)\left(2l-1\right)}} \qquad
b^{l,m}= \sqrt{\frac{\left(l-m\right)\left(l+m\right)}{\left(2l+1\right)\left(2l-1\right)}} \qquad
c^{l,m}= \sqrt{\frac{\left(l+m+1\right)\left(l+m+2\right)}{\left(2l+3\right)\left(2l+1\right)}}
$}
\end{equation*}
\begin{equation*}
\resizebox{1.0\hsize}{!}{$
d^{l,m}= \sqrt{\frac{\left(l-m\right)\left(l-m-1\right)}{\left(2l+1\right)\left(2l-1\right)}} \qquad
e^{l,m}= \sqrt{\frac{\left(l-m+1\right)\left(l-m+2\right)}{\left(2l+3\right)\left(2l+1\right)}} \qquad
f^{l,m}= \sqrt{\frac{\left(l+m\right)\left(l+m-1\right)}{\left(2l+1\right)\left(2l-1\right)}}
$}
\end{equation*}
Such recursion relations seem to be hard to find in the standard literature and are preserved through citations in relevant articles\footnote{according to email exchange with experts from nuclear sciences}, such as Seibold et al.~\cite{Seibold14} or Brunner et al.~\cite{Brunner05}. Note that the signs for the $x$- and $y$- component depend on the handedness of the coordinate system in which the spherical harmonics basis functions are defined.

%\subsubsection*{Expansion of a Delta Function}
\TD{expanding delta function, used for light sources (Q)}
\TD{eigenfunction property}
\TD{product integrals etc. as needed}

\section{Complex-valued $P_N$-equations}
\label{sec:pn_cvalued}

Equipped with key properties of the spherical harmonics expansion, the complex-valued $P_N$-equations are derived in this section. The derivation follows two basic steps, which are executed separately on each term of the radiative transfer equation. The first step is to substitute each angular dependent radiative transfer quantity, such as the radiance field $L$, by its (truncated) spherical harmonics projection (equation~\ref{eq:sh_projection}). The second step is to apply the spherical harmonics projection to each term of the radiative transfer equation. This results in terms, which will depend on the spherical harmonics indices $l,m$ and therefore form a system of equations. The size of this system is driven by the truncation order $N$. The higher the value $N$, the higher frequencies in angular domain are taken into account for an approximation of the radiative transport.
% --------------------------------------------------------------
\subsubsection*{Transport Term}
The transport term of the RTE is given as
\begin{align*}
(\vec{\omega}\cdot\nabla)L(\vec{x}, \vec{\omega})
\end{align*}
Replacing $L$ with its spherical harmonics expansion results in:
\begin{align*}
\left(\vec{\omega}\cdot\nabla\right)
\left(
\sum_{l,m}
{
L^{l,m}\left(\vec{x}\right )
\SHBC^{l,m}\left(\vec{\omega}\right)
}
\right)
\end{align*}
Next, the whole term is projected into spherical harmonics, which means a multiplication with $\overline{\SHBC^{l'm'}}$ and integration over solid angle:
\begin{align*}
\int_\Omega
{
\overline{Y^{l'm'}}(\vec{\omega}\cdot\nabla)
\sum_{l,m}
{
L^{l,m}\left(\vec{x}\right)
\SHBC^{l,m}\left(\vec{\omega}\right)
}
}
\ud\vec{\omega}
\end{align*}
The spatial derivative can be extracted to get:
\begin{align*}
\nabla\cdot\int_\Omega
{
\vec{\omega}\overline{\SHBC^{l'm'}}
\sum_{l,m}
{
L^{l,m}\left(\vec{x}\right)
\SHBC^{l,m}\left(\vec{\omega}\right)
}
\ud\vec{\omega}
}
\end{align*}
When applying the recursion relation (equation~\ref{eq:recursive_relation}) the following will be shown:
\begin{equation*}
\resizebox{1.0\hsize}{!}{$
\begin{pmatrix}
\frac{1}{2}\partial_x\\
\frac{i}{2}\partial_y\\
\partial_z
\end{pmatrix}
\cdot
\int_\Omega
\begin{pmatrix}
\ c^{l'-1, m'-1}\overline{\SHBC^{l'-1,m'-1}} - d^{l'+1, m'-1}\overline{\SHBC^{l'+1,m'-1}} - e^{l'-1, m'+1}\overline{\SHBC^{l'-1,m'+1}} + f^{l'+1, m'+1}\overline{\SHBC^{l'+1,m'+1}}\\
-c^{l'-1, m'-1}\overline{\SHBC^{l'-1,m'-1}} + d^{l'+1, m'-1}\overline{\SHBC^{l'+1,m'-1}} - e^{l'-1, m'+1}\overline{\SHBC^{l'-1,m'+1}} + f^{l'+1, m'+1}\overline{\SHBC^{l'+1,m'+1}} \\
a^{l'-1, m'}\overline{\SHBC^{l'-1,m'}}+b^{l'+1, m'}\overline{\SHBC^{l'+1,m'}}
\end{pmatrix}
\sum_{l,m}{
L^{l,m}\left(\vec{x}\right )\SHBC^{l,m}\left(\vec{\omega}\right)
}
\ud\vec{\omega}
$}
\end{equation*}
Integrating the vector term over solid angle, can be expressed as seperate solid angle integrals over each component. These integrals over a sum of terms are split into seperate integrals:
\begin{align*}
\begin{pmatrix}
\frac{1}{2}\partial_x\\
\frac{i}{2}\partial_y\\
\partial_z
\end{pmatrix}
\cdot
\begin{pmatrix}
\ c^{l'-1, m'-1}\sum_{l,m}{L^{l,m}\left(\vec{x}\right )\int_\Omega{\overline{\SHBC^{l'-1,m'-1}}\left(\vec{\omega}\right)\SHBC^{l,m}\left(\vec{\omega}\right)\ud\vec{\omega}}} \quad - \quad ...\\
-c^{l'-1, m'-1}\sum_{l,m}{L^{l,m}\left(\vec{x}\right )\int_\Omega{\overline{\SHBC^{l'-1,m'-1}}\left(\vec{\omega}\right)\SHBC^{l,m}\left(\vec{\omega}\right)\ud\vec{\omega}}} \quad + \quad ... \\
a^{l'-1, m'}\sum_{l,m}{L^{l,m}\left(\vec{x}\right )\int_\Omega{\overline{\SHBC^{l'-1,m'}}\left(\vec{\omega}\right)\SHBC^{l,m}\left(\vec{\omega}\right)\ud\vec{\omega}}} \quad + \quad ...
\end{pmatrix}
\end{align*}
Applying the orthogonality property to the solid angle integrals will select specific $l,m$ in each term:
\begin{equation*}
\resizebox{1.0\hsize}{!}{$
\begin{pmatrix}
\frac{1}{2}\partial_x\\
\frac{i}{2}\partial_y\\
\partial_z
\end{pmatrix}
\cdot
\begin{pmatrix}
\ c^{l-1, m-1}L^{l-1,m-1} - d^{l+1, m-1}L^{l+1,m-1} - e^{l-1, m+1}L^{l-1,m+1} + f^{l+1, m+1}L^{l+1,m+1}\\
-c^{l-1, m-1}L^{l-1,m-1} + d^{l+1, m-1}L^{l+1,m-1} - e^{l-1, m+1}L^{l-1,m+1} + f^{l+1, m+1}L^{l+1,m+1} \\
a^{l-1, m}L^{l-1,m}+b^{l+1, m}L^{l+1,m}
\end{pmatrix}
$}
\end{equation*}
Which gives the final moment equation for the transport term:
\begin{align}
&
\frac{1}{2}c^{l-1, m-1}\partial_x L^{l-1,m-1}
-\frac{1}{2}d^{l+1, m-1}\partial_x L^{l+1,m-1}
-\frac{1}{2}e^{l-1, m+1}\partial_x L^{l-1,m+1}
\nonumber
\\
&
+\frac{1}{2}f^{l+1, m+1}\partial_x L^{l+1,m+1}
-\frac{i}{2}c^{l-1, m-1}\partial_y L^{l-1,m-1}
+\frac{i}{2}d^{l+1, m-1}\partial_y L^{l+1,m-1}
\nonumber
\\
&
-\frac{i}{2}e^{l-1, m+1}\partial_y L^{l-1,m+1}
+\frac{i}{2}f^{l+1, m+1}\partial_y L^{l+1,m+1}
+
a^{l-1, m}\partial_z L^{l-1,m}+b^{l+1, m}\partial_z L^{l+1,m}
\label{eq:sh_complex_transport}
\end{align}

% --------------------------------------------------------------
\subsubsection*{Scattering Term}

The scattering term in the RTE is given as:
\begin{align*}
\sigma_s(\vec{x})\int_{\Omega}\phase(\vec{x}, \vec{\omega}'\cdot\vec{\omega})L(\vec{x}, \vec{\omega}')\ud\vec{\omega}'
\end{align*}
The phase function, used in isotropic scattering medium, only depends on the angle between incident and outgoing direction, and therefore is rotationally symmetric around the pole defining axis. This property allows to define a rotation $R(\vec{\omega})$, which rotates the phase function, such that the pole axis aligns with the outgoing direction vector $\vec{\omega}$. This rotation is expressed as:
\begin{align*}
\rho_{R(\vec{\omega})}(\phase)
\ ,
\end{align*}
which can be implemented by applying the inverse rotation $R(\vec{\omega})^{-1}$ to the arguments of $\phase$. With this rotated phase function, the integral of the scattering operator can be expressed as a convolution:
\begin{align}
\int_{\Omega'}\phase(\vec{x}, \vec{\omega}'\cdot\vec{\omega})L(\vec{x}, \vec{\omega}')\ud\vec{\omega}'
&=
\int_{\Omega'}{\rho_{R(\vec{\omega})}(\phase)(\operatorname{cos}\theta')L(\vec{x}, \vec{\omega}')\ud\vec{\omega}'}
\nonumber\\
&= \langle L,  \rho_{R(\vec{\omega})}(\phase)\rangle
\label{eq:complex_scatt_conv}
\end{align}
As the inner product integral of the convolution is evaluated, the phase function rotates along with the argument $\vec{\omega}$.

Now, the spherical harmonics expansions of $L$ and $\phase$ (equation~\ref{eq:sh_exp_phase}) in the definition for the inner product of the convolution (equation~\ref{eq:complex_scatt_conv}) is used:
\begin{align*}
\langle L,  \rho_{R(\vec{\omega})}(\phase)\rangle = \left < \sum_{l,m}{L^{l,m}(\vec{x}) \SHBC^{l,m}}, \rho_{R(\vec{\omega})}\left ( \sum_{l}{\phase^{l0}\SHBC^{l0}} \right )\right>
\end{align*}
Due to linearity of the inner product operator, the non-angular dependent parts of the expansions can be pulled out:
\begin{align*}
\langle L,  \rho_{R(\vec{\omega})}(\phase)\rangle
&=
\sum_{l,m}
{
L^{l,m}(\vec{x})
\left<
\SHBC^{l,m},
\rho_{R(\vec{\omega})}
\left(\sum_l{\phase^{l0} \SHBC^{l0}}\right)
\right>
}
\end{align*}
and further:
\begin{align*}
\langle L,  \rho_{R(\vec{\omega})}(\phase)\rangle
&=
\sum_{l'}
{
\sum_{l,m}
{
\phase^{l'0}L^{l,m}(\vec{x})
\left<\SHBC^{l,m}, \rho_{R(\vec{\omega})}\left( \SHBC^{l'0} \right)\right>
}
}
\end{align*}
The rotation $\rho_{R(\vec{\omega})}$ of a function with frequency $l$ results in a function of frequency $l$ (equation~\ref{eq:sh_rotation}). In addition the spherical harmonics basis functions $\SHBC^{l,m}$ are orthogonal. Therefore following applies:
\begin{align*}
\left<
\SHBC^{l,m}, \rho_{R(\vec{\omega})}\left(\SHBC^{l'm'}\right)
\right> = 0       \qquad    \text{for all}\ \ l\ne l' 
\end{align*}
which further simplifies the inner product integral to:
\begin{align*}
\langle L,  \rho_{R(\vec{\omega})}(\phase)\rangle
&=
\sum_{l,m}
{
\phase^{l0}L^{l,m}(\vec{x})
\left<
\SHBC^{l,m}, \rho_{R(\vec{\omega})}\left(\SHBC^{l0} \right )
\right>
}
\end{align*}
What remains to be resolved is the inner product. The spherical harmonics basis functions $ \SHBC^{l,m}$ are eigenfunctions of the inner product integral operator in the equation above (Dai~\cite{Dai13}):
\begin{align*}
\left<
\SHBC^{l,m}, \rho_{R(\vec{\omega})}\left ( \SHBC^{l0} \right )\right> = \lambda_l \SHBC^{l,m}
\end{align*}
with
\begin{align*}
\lambda_l=\sqrt{\frac{4\pi}{2l+1}}
\end{align*}
Replacing the inner product leads to:
\begin{align*}
\langle L,  \rho_{R(\vec{\omega})}(p)\rangle
&=
\sum_{l,m}
{
\lambda_l
\phase^{l0}L^{l,m}(\vec{x})
\SHBC^{l,m}
}
\end{align*}
This allows to express the scattering term using SH expansions of phase function $p$ and radiance field $L$:
\begin{align*}
\sigma_s(\vec{x})\int_{\Omega}\phase(\vec{x}, \vec{\omega}'\cdot\vec{\omega})L(\vec{x}, \vec{\omega}')\ud\vec{\omega}'
&=
\sigma_s(\vec{x})\langle L,  \rho_{R(\vec{\omega})}(p)\rangle
\\
&=
\sigma_s(\vec{x})
\sum_{l,m}
{
\lambda_l
\phase^{l0}L^{l,m}(\vec{x})
\SHBC^{l,m}
}
\end{align*}
However, a spherical harmonics expansion of the term itsself has not been done yet. It is still a scalar function that depends on direction $\vec{\omega}$. The scattering term is projected into spherical harmonics, by multiplying with $\overline {\SHBC^{l'm'}}$ and integrating over solid angle $\vec{\omega}$. Further all factors will be pulled out, which do not depend on $\vec{\omega}$ and apply the SH orthogonality property to arrive at the scattering term of the complex-valued $P_N$-equations:
\begin{align}
&
\int_{\Omega}
{
\overline{\SHBC^{l'm'}}(\vec{\omega})
\sigma_s(\vec{x})
\sum_{l,m}
{
\lambda_l
\phase^{l0}L^{l,m}(\vec{x})
\SHBC^{l,m}\left(\vec{\omega}\right)
}
\ud\vec{\omega}
}
\nonumber\\
=&
\lambda_l
\sigma_s(\vec{x})
\phase^{l0}L^{l,m}(\vec{x})
\sum_{l,m}
{
\int_{\Omega}
{
\overline{\SHBC^{l'm'}}(\vec{\omega})
\SHBC^{l,m}\left(\vec{\omega}\right)
\ud\vec{\omega}
}
}
\nonumber\\
=&
\lambda_l
\sigma_s(\vec{x})
\phase^{l0}L^{l,m}(\vec{x})
\sum_{l,m}
{
\delta_{ll'}\delta_{mm'}
}
\nonumber\\
=&
\lambda_l
\sigma_s(\vec{x})
\phase^{l0}L^{l,m}(\vec{x})
\label{eq:sh_complex_scattering}
\end{align}



% ------------------------------------------------------------
\subsubsection*{Collision and Emission Term}

The collision term of the RTE is given as:
\begin{align*}
-\sigma_t\left(\vec{x}\right)L\left(\vec{x}, \vec{\omega}\right)
\end{align*}
The radiance field $L$ is replaced with its spherical harmonics expansion:
\begin{align*}
-\sigma_t\left(\vec{x}\right)
\sum_{l,m}
{
L^{l,m}\left(\vec{x}\right )\SHBC^{l,m}\left(\vec{\omega}\right)
}
\end{align*}
Multiplying with $\overline{\SHBC^{l'm'}}$ and integrating over solid angle, after pulling some factors out of the integral and applying the orthogonality identity (equation~\ref{eq:sh_orthogonality}) results in:
\begin{align}
&-\sigma_t\left(\vec{x}\right)\sum_{l,m}{L^{l,m}\left(\vec{x}\right )\int_\Omega\overline{\SHBC^{l'm'}}\left(\vec{\omega}\right)\SHBC^{l,m}\left(\vec{\omega}\right)\ud\vec{\omega}}
\nonumber
\\
&= -\sigma_t\left(\vec{x}\right)\sum_{l,m}{L^{l,m}\left(\vec{x}\right )\delta_{ll'}\delta_{mm'}}
\nonumber
\\
&= -\sigma_t\left(\vec{x}\right)L^{l,m}\left(\vec{x}\right )
\label{eq:sh_complex_collision}
\end{align}
The derivation of the SH projected term is equal to the derivation of the projected collision term. After replacing the emission field with its SH projection and multiplying the term with the conjugate complex of $\SHBC$ results after application of the orthogonality property in:
\begin{align}
Q^{l,m}\left(\vec{x}, \vec{\omega}\right)
\label{eq:sh_complex_emission}
\end{align}


% --------------------------------------------------------------
\subsubsection*{Final Equation}

The results are the complex-valued $P_N$-equations, after putting all the projected terms together (equation~\ref{eq:sh_complex_transport},~\ref{eq:sh_complex_collision},~\ref{eq:sh_complex_scattering} and~\ref{eq:sh_complex_emission}):
\begin{align}
&
\frac{1}{2}c^{l-1, m-1}\partial_x L^{l-1,m-1}
-\frac{1}{2}d^{l+1, m-1}\partial_x L^{l+1,m-1}
-\frac{1}{2}e^{l-1, m+1}\partial_x L^{l-1,m+1}
\nonumber
\\
&
+\frac{1}{2}f^{l+1, m+1}\partial_x L^{l+1,m+1}
-\frac{i}{2}c^{l-1, m-1}\partial_y L^{l-1,m-1}
+\frac{i}{2}d^{l+1, m-1}\partial_y L^{l+1,m-1}
\nonumber
\\
&
-\frac{i}{2}e^{l-1, m+1}\partial_y L^{l-1,m+1}
+\frac{i}{2}f^{l+1, m+1}\partial_y L^{l+1,m+1}
+a^{l-1, m}\partial_z L^{l-1,m}
+b^{l+1, m}\partial_z L^{l+1,m}
\nonumber
\\
&
=
-\sigma_t\left(\vec{x}\right)L^{l,m}\left(\vec{x}\right )
+
\lambda_l
\sigma_s(\vec{x})
\phase^{l0}L^{l,m}(\vec{x}) + Q^{l,m}\left(\vec{x}, \vec{\omega}\right)
\label{eq:sh_pne_complex}
\end{align}
\section{Real-valued $P_N$-equations}
\label{sec:pn_rvalued}

Depending on the truncation value $N$, the complex-valued $P_N$-equations will induce a certain number of spherical harmonics coefficients for every discretized position within the domain. These will be complex and therefore produce two unknowns per coefficient. Moreover, the whole pipeline from solver to rendering integration will have to deal with complex numbers. This is an unnecessary burden, as we actually only have real-valued radiance fields and volumetric quantities in rendering.

If the spherical harmonics projected function or operator is real-valued, then the resulting projection will have a redundant structure which allows cutting the number of coefficients in half. The real-valued spherical harmonics basis function (denoted $\SHBR$) can be defined by projecting the remaining coefficients onto their real and imaginary parts:
% real valued SH ------------------
\begin{align}
\label{eq:real_sh_basis}
\SHBR^{l,m}=
\left\{
\begin{array}{lr}
\frac{\iu}{\sqrt{2}}\left(\SHBC^{l,m}-\left(-1\right)^m\SHBC^{l,-m}\right), & \text{for } m < 0\\
\SHBC^{l,m}, & \text{for } m = 0\\
\frac{1}{\sqrt{2}}\left(\SHBC^{l,-m}+\left(-1\right)^m\SHBC^{l,m}\right), & \text{for } m > 0
\end{array}
\right.
\end{align}
Therefore the number of spherical harmonics coefficients remains the same, but since those are now real only, the number of unknowns is cut in half when compared to complex-valued spherical harmonics.

In this section, the real-valued $P_N$-equations are derived by using $\SHBR$ instead of $\SHBC$ for the derivation. This allows to get rid of complex numbers everywhere in the pipeline and reduces the amount of unknowns. The real-valued $P_N$-equations as such are not new. However, performing the derivation steps manually produces very large equations which are hard to work with and make it very difficult to see structures and opportunities for simplification. Therefore in the literature, the real-valued $P_N$-equations are usually specified in form of the complex-valued $P_N$-equations in matrix form with an additional transformation matrix, representing the projection to real-valued spherical harmonics (see for example Seibold et al.~\cite{Seibold14}). An explicit form of the real-valued $P_N$-equation cannot be found in the literature. An attempt to derive such a form has been made by Frank et al.~\cite{Frank14} (appendix A), but their result still requires matrix notation with unwieldy coefficient matrices. 

In this thesis a new explicit form of the real-valued $P_N$-equations is derived, which is very concise and compact. This derivation was made possible by using the computer algebra representation framework which has been developed as part of the solver (see section~\ref{sec:pn_car}). The derivation steps for the $P_N$-equations were carried out using this representation and simplification opportunities could be discovered much more easily, leading to the compact form presented at the end of this section.

Being real-valued implies that no complex-conjugate is needed for the inner product. The projection into real-valued spherical harmonics therefore is
\begin{align}
\label{eq:sh_projection}
f^{l, m} = \left<f,\SHBR^{l, m}\right> = 
\int_\Omega f\left(\omega\right)\SHBR^{l,m}\left(\omega\right)\ud\omega
\ .
\end{align}
Similarly the reconstruction is done by using the real-valued spherical harmonics basis function
\begin{align}
f\left(\omega\right) = 
\sum_{l=0}^{N}
{
\sum_{m=-l}^{l}
{
f^{l,m}\left(\vec{x}\right)\SHBR^{l,m}\left(\omega\right)
}
}
\ .
\label{eq:sh_real_reconstruction}
\end{align}
All properties of the complex-valued spherical harmonics basis functions, except the recursion relation in equation~\ref{eq:recursive_relation}, carry through to the real-valued basis functions. In particular, the orthogonality property still holds:
\begin{align}
\label{eq:sh_orthogonality_real}
\left<\SHBR^{l_1, m_1},\SHBR^{l_2, m_2}\right>=
\int_{\Omega} \SHBR^{l_1m_1}\left(\omega\right) \SHBR^{l_2m_2}\left(\omega\right) \mathbf{d}\omega = \delta_{l_1m_1}\delta_{l_2m_2}
\ .
\end{align}

% --------------------------------------------------------------
\subsubsection*{Transport Term}

Carrying out the derivation for the transport term of the radiative transfer equation produces very different terms compared to the complex-valued counterpart. The derivation is extensive and given in appendix~\ref{app:rpn}. As the spherical harmonics basis functions are different depending on the sign of $m$ (see equation~\ref{eq:real_sh_basis}), the projected transport term is also different accordingly. After carrying out the derivation steps we arrive for $m<0$ at:
\begin{align*}
&-\frac{1}{2}c^{{l-1,m-1}}
\partial_y
L^{l-1,-m+1}
%\\
+\frac{1}{2}d^{{l+1,m-1}}
\partial_y
L^{l+1,-m+1}
%\\
-\frac{1}{2}\beta^{m}e^{{l-1,m+1}}
\partial_y
L^{l-1,-m-1}
\\&
+\frac{1}{2}\beta^{m}f^{{l+1,m+1}}
\partial_y
L^{l+1,-m-1}
%\\
+\frac{1}{2}c^{{l-1,m-1}}
\partial_x
L^{l-1,m-1}
\\&
-\frac{1}{2}\delta_{\scaleto{m\neq -1}{4pt}}e^{{l-1,m+1}}
\partial_x
L^{l-1,m+1}
%\\
+\frac{1}{2}\delta_{\scaleto{m\neq -1}{4pt}}f^{{l+1,m+1}}
\partial_x
L^{l+1,m+1}
%\\
-\frac{1}{2}d^{{l+1,m-1}}
\partial_x
L^{l+1,m-1}
\\&
+a^{{l-1,m}}
\partial_z
L^{l-1,m}
%\\
+b^{{l+1,m}}
\partial_z
L^{l+1,m}
\end{align*}
with
\begin{align}
\label{eq:real_sh_basis}
\beta^{x}=
\left\{
\begin{array}{lr}
\frac{2}{\sqrt{2}}, & \text{for } \vert x\vert = 1\\
1, & \text{for } \vert x\vert \neq 1
\end{array}
\right.
\end{align}
Following the derivation through for $m>0$ gives:
\begin{align}
&
\frac{1}{2}c^{l-1,-m-1}\partial_x L^{l-1,m+1}
%\\&
-\frac{1}{2}d^{l+1,-m-1}\partial_x L^{l+1,m+1}
%\\&
-\frac{1}{2}\beta^{m}e^{l-1,m-1}\partial_x L^{l-1,m-1}
\nonumber
\\&
\frac{1}{2}\beta^{m}f^{l+1,-m+1}\partial_x L^{l+1,m-1}
%\\&
\frac{1}{2}c^{l-1,-m-1}\partial_y L^{l-1,-m-1}
%\\&
-\frac{1}{2}d^{l+1,-m-1}\partial_y L^{l+1,-m-1}
\nonumber
\\&
\delta_{\scaleto{m\neq 1}{4pt}}\frac{1}{2}e^{l-1,-m+1}\partial_y L^{l-1,-m+1}
%\\&
-\delta_{\scaleto{m\neq 1}{4pt}}\frac{1}{2}f^{l+1,-m+1}\partial_y L^{l+1,-m+1}
%\\&
a^{l-1,-m}\partial_z L^{l-1,m}
\nonumber
\\&
b^{l+1,-m}\partial_z L^{l+1,m}
\nonumber
\end{align}
Similar simplifications apply to the remaining terms of the spherical harmonics expansion of the transport term for $m=0$, resulting in the expression:
\begin{align}
\label{eq:pn_rvalued_transport_m0}
&
\frac{1}{\sqrt{2}}c^{{l-1,-1}}\partial_x L^{{l-1,1}}
-\frac{1}{\sqrt{2}}d^{{l+1,-1}}\partial_x L^{{l+1,1}}
\\&
\frac{1}{\sqrt{2}}c^{{l-1,-1}}\partial_y L^{{l-1,-1}}
-\frac{1}{\sqrt{2}}d^{{l+1,-1}}\partial_y L^{{l+1,-1}}
\\&
a^{{l-1,0}}\partial_z L^{{l-1,0}}
+b^{{l+1,0}}\partial_z L^{{l+1,0}}
\end{align}

% --------------------------------------------------------------
\subsubsection*{Collision, Scattering and Emission}

The remaining terms of the radiative transfer equation are derived in the same way, as their complex-valued counterpart. The collision term for $m < 0$, $m=0$ and $m > 0$ is
\begin{align}
\label{eq:pn_rvalued_transport_collision}
-\sigma_t L^{l,m}
\ .
\end{align}
We have for the scattering term
\begin{align}
\label{eq:pn_rvalued_transport_scattering}
\sigma_s\lambda_{l}p^{l,0}\left(\vec{x}\right)L^{l,m}
\end{align}
and finally for the emission term
\begin{align}
\label{eq:pn_rvalued_transport_emission}
Q^{l,m}
\ .
\end{align}

% --------------------------------------------------------------
\subsubsection*{Final Equation}

By putting equation~\ref{eq:pn_rvalued_transport_m0}, equation~\ref{eq:pn_rvalued_transport_collision}, equation~\ref{eq:pn_rvalued_transport_scattering} and equation~\ref{eq:pn_rvalued_transport_emission} together we get for $m=0$:
\begin{align}
&
\frac{1}{\sqrt{2}}c^{\scaleto{l-1,-1}{4pt}}\partial_x L^{\scaleto{l-1,1}{4pt}}
-\frac{1}{\sqrt{2}}d^{\scaleto{l+1,-1}{4pt}}\partial_x L^{\scaleto{l+1,1}{4pt}}
%\\&
\frac{1}{\sqrt{2}}c^{\scaleto{l-1,-1}{4pt}}\partial_y L^{\scaleto{l-1,-1}{4pt}}
\nonumber
\\&
-\frac{1}{\sqrt{2}}d^{\scaleto{l+1,-1}{4pt}}\partial_y L^{\scaleto{l+1,-1}{4pt}}
%\\&
+a^{\scaleto{l-1,0}{4pt}}\partial_z L^{\scaleto{l-1,0}{4pt}}
+b^{\scaleto{l+1,0}{4pt}}\partial_z L^{\scaleto{l+1,0}{4pt}}
\nonumber
\\&
=
-\sigma_t L^{\scaleto{l,m}{4pt}}
+\sigma_s\lambda_{\scaleto{l}{4pt}}p^{\scaleto{l,0}{4pt}}L^{\scaleto{l,m}{4pt}}
+ Q^{\scaleto{l,m}{4pt}}
%\label{eq:rpn_m_=_z}
\end{align}
We can compact the equations for $m<0$ and $m>0$ into a single expression by differentiating the signs using $\pm$. The  top sign is associated with $m<0$ and for $m>0$ we use the bottom sign. This results in the real-valued $P_N$-equation for $m<0$ and $m>0$:
\begin{align}
&
\frac{1}{2}c^{\scaleto{l-1,\pm m-1}{4pt}}\partial_x L^{\scaleto{l-1,m\mp 1}{4pt}}
%\\&
-\frac{1}{2}d^{\scaleto{l+1,\pm m-1}{4pt}}\partial_x L^{\scaleto{l+1,m\mp 1}{4pt}}
%\\&
-\frac{1}{2}\beta_x^{\scaleto{m}{4pt}}e^{\scaleto{l-1,m\pm 1}{4pt}}\partial_x L^{\scaleto{l-1,m\pm 1}{4pt}}
\nonumber
\\&
+\frac{1}{2}\beta_x^{\scaleto{m}{4pt}}f^{l+1,\pm m+1}\partial_x L^{\scaleto{l+1,m\pm 1}{4pt}}
%\\&
\mp \frac{1}{2}c^{\scaleto{l-1,\pm m-1}{4pt}}\partial_y L^{\scaleto{l-1,-m\pm 1}{4pt}}
\nonumber
\\&
\pm \frac{1}{2}d^{\scaleto{l+1,\pm m-1}{4pt}}\partial_y L^{\scaleto{l+1,-m \pm 1}{4pt}}
%\nonumber
%\\&
\mp \beta_y^{\scaleto{m}{4pt}}\frac{1}{2}e^{\scaleto{l-1,\pm m+1}{4pt}}\partial_y L^{\scaleto{l-1,-m\mp 1}{4pt}}
\nonumber
\\&
\pm \beta_y^{\scaleto{m}{4pt}}\frac{1}{2}f^{\scaleto{l+1,\pm m+1}{4pt}}\partial_y L^{\scaleto{l+1,-m\mp 1}{4pt}}
%\\&
+a^{\scaleto{l-1,\pm m}{4pt}}\partial_z L^{\scaleto{l-1,\mp m}{4pt}}
%\nonumber
%\\&
+b^{\scaleto{l+1,\pm m}{4pt}}\partial_z L^{\scaleto{l+1,\mp m}{4pt}}
\nonumber
\\&
=
-\sigma_t L^{\scaleto{l,m}{4pt}}
+\sigma_s\lambda_{\scaleto{l}{4pt}}p^{\scaleto{l,0}{4pt}}L^{\scaleto{l,m}{4pt}}
+Q^{\scaleto{l,m}{4pt}}
%\nonumber
%\label{eq:rpn_m_>_z}
\end{align}
with
\begin{align*}
%\label{eq:real_sh_basis}
\beta_x^{m}=
\left\{
\begin{array}{ll}
0, & \text{for } m = -1\\
\frac{2}{\sqrt{2}}, & \text{for } m \neq 1\\
1, & \text{otherwise }
\end{array}
\right.
,\quad
\beta_y^{m}=
\left\{
\begin{array}{ll}
\frac{2}{\sqrt{2}}, & \text{for } m = -1\\
0, & \text{for } m \neq 1\\
1, & \text{otherwise }
\end{array}
\right.
\end{align*}
and
\begin{align*}
&
a^{\scaleto{l,m}{4pt}}= \sqrt{\frac{\left(l-m+1\right)\left(l+m+1\right)}{\left(2l+1\right)\left(2l-1\right)}} \qquad
b^{\scaleto{l,m}{4pt}}= \sqrt{\frac{\left(l-m\right)\left(l+m\right)}{\left(2l+1\right)\left(2l-1\right)}}
\\&
c^{\scaleto{l,m}{4pt}}= \sqrt{\frac{\left(l+m+1\right)\left(l+m+2\right)}{\left(2l+3\right)\left(2l+1\right)}} \qquad
d^{\scaleto{l,m}{4pt}}= \sqrt{\frac{\left(l-m\right)\left(l-m-1\right)}{\left(2l+1\right)\left(2l-1\right)}}
\\&
e^{\scaleto{l,m}{4pt}}= \sqrt{\frac{\left(l-m+1\right)\left(l-m+2\right)}{\left(2l+3\right)\left(2l+1\right)}} \qquad
f^{\scaleto{l,m}{4pt}}= \sqrt{\frac{\left(l+m\right)\left(l+m-1\right)}{\left(2l+1\right)\left(2l-1\right)}}
\end{align*}
\begin{align*}
\lambda_l=\sqrt{\frac{4\pi}{2l+1}}
\end{align*}
Throughout this thesis and in this chapter in particular only these real-valued $P_N$-equations are used. The next section will cover two-dimensional problems, followed by the introduction of a numerical method for solving these coupled partial differential equations.











\section{Two-dimensional problems}
\label{sec:pn_2d}

Working with three-dimensional problems is computationally demanding and makes debugging and visualization difficult. In other fields it is common practice to study and validate ideas on two-dimensional problems. The derivation of the two-dimensional $P_N$-equation is straightforward and will be presented in this section.

A domain in the two-dimensional space $\mathbb{R}^2$ is considered, which is extruded infinitely along the $z-axis$ in both directions into $\mathbb{R}^3$. It shall be further assumed that input fields, such as $\sigma_t$, and boundary conditions, also do not change along the $z$-axis.

Since the domain has infinite extent in $z$-direction and radiative quantities do not vary along that axis, the radiance field $L$ will also not change along the $z$-axis at any position within the domain. Further, the upper hemisphere of the spherical radiance function will be a mirrored version of the lower hemisphere, with the equator being the mirror plane. In mathematical terms this is expressed as $L$, being an even function in the polar angle $\theta$.

From the definition of the Associated Legendre Polynomials (see equation~\ref{eq:sh_Plm}) it can be concluded that they are either even or odd according to
\begin{align}
P^{l,m}\left(-\operatorname{cos}\theta\right) = 
\left(-1\right)^{l+m}
P^{l,m}\left(\operatorname{cos}\theta\right)
\ .
\end{align}
If $l+m$ is odd, then $P^{l,m}$ is odd. This means that it integrates to zero over a symmetric interval around the origin, such as the polar angle range. Since the spherical harmonics basis functions scales $P^{lm}$ only uniformly and along the azimuthal angle, it can be inferred that the spherical harmonics basis functions are odd in the polar angle $\theta$ if $l+m$ is odd.

The spherical harmonics projection of the radiance field multiplies the radiance field with the spherical harmonics basis function and integrates it over solid angle. This is done for each spherical harmonics coefficient ($l,m$-pair) that is required. It has been established that if $l+m$ is odd, the basis function will be odd in $\theta$. Since the radiance field is even in $\theta$, this means that their product will be odd. This implies that the integral of their product over the polar angle range will be zero.

This in turn leads to the spherical harmonics coffeficients of $L$ to be zero for all coefficients where $l+m$ is odd. Further, all occurances of $\partial_z$ in the $P_N$-equations can be set to zero.


\section{$P_N$-Solver with Automated Stencil Generation}
\label{sec:pn_solver}

In this section, a new method for solving the $P_N$-equations is being given. This method is based on the idea to automatically generate the linear system of equations directly from a computer algeba representation of the $P_N$-equations. First this section goes through the individual steps manually required to build such a system in general from partial differential equations, followed by the motivation for the automated approach taken in this thesis. In subsections~\ref{sec:pn_car}-~\ref{sec:pn_staggered}, the solver and the automation of the individual system-building steps are explained in detail, including the developed software framework and important numerical aspects, such as handling boundary conditions and staggered grids.

There are two reasons why the solver has been designed with the idea of automating the system building as much as possible. The first reason simply is that the number of equations for the system depends on the user parameter $N$ (the spherical harmonics truncation order) and is therefore not known in advance during solver implementation. In addition, the number of equations increases exponentially with truncation order $N$, which makes building the system increasingly laborious, time consuming and prone to errors, if done manually. The second reason is that there exists a rich variety of variations of the $P_N$-method in other fields, dealing with different aspects, such as reduction of ringing artefacts and improvement of convergence. The idea behind our system is to be able to generate the system of linear equations directly from the specification of a general coupled set of partial differential equations. This would allow easy implementation and cross-comparisons of these alternate theories from the literature.
\TD{give references for these variations}

The $P_N$-equations have been derived in previous sections by discretizing the continuous angular variable in the radiative transfer equation. This led to a discrete set of coupled partial differential equations which still depend on the continuous spatial variable~$\vec{x}$. Therefore, the next step is to discretize this spatial variable on a particular domain with input fields for the various radiative quantities, such as extinction or the phase function. Finally, using this discretization, a system of linear equations can be build by assembling a coefficient matrix $A$ and a right hand side vector~$\vec{b}$.

As mentioned in section~\ref{sec:discretization}, a discrete representation of the geometry that is involved in the problem is required in order to discretize the spatial variable. Throughout this thesis, a regular cartesian grid is used, which in this thesis will be referred to as finite difference grid. This type of grid is very simple and commonly used in graphics. In the future it might be worth exploring other more complex meshes, such as tetrahedral irregular grids, which are often used in finite element analysis. 

The finite difference grid defines the locations at which radiative transfer quantities and the solution will be specified. It is parameterized by its resolution $res=(I,J,K)$ or gridspacing $h=(h_x, h_y, h_z)$ in each dimension. The local space of that grid is referred to as voxelspace.

Using finite difference grids, the discretization consists of replacing the quantities which depend on the continuous variable~$\vec{x}$ with their counterpart which depend on~$x_{ijk}$, where $ijk$ is a coordinate in voxelspace. Note that this coordinate is still continuous. For example $x_{i+\frac{1}{2}jk}$ is a valid coordinate. Quantities around grid point locations are found by interpolation. Troughout this thesis we use trilinear interpolation for three-dimensional grids and bilinear interpolation for two-dimensional grids. Higher order interpolation would be possible and an avenue for future exploration.
\TD{image of finite difference grid showing domain, and gridpoints and indices and voxelsize h, and interpolation}

\subsubsection*{Spatial Discretization}
The first step with discretization is to do the replacement $\vec{x}\rightarrow x_{ijk}$. Using the two-dimensional laplace equation $\nabla^2\phi(\vec{x})=u(\vec{x})$ as example, this would give:
\begin{align*}
\nabla^2\phi\left(\vec{x}\right)=u\left(\vec{x}\right)
\xrightarrow{\makebox[1cm]{}}
\partial_x^2\phi_{ij}+
\partial_y^2\phi_{ij}
=
u_{ij}
\end{align*}
The spatial derivatives are approximated using central differences:
\begin{align}
\partial_x\phi_{ijk} = \frac{\phi_{i+\frac{1}{2}jk} - \phi_{i-\frac{1}{2}jk}}{h}
\label{eq:pn_solver_central_difference}
\end{align}
Continuing with the Laplace example, these substitutions are carried out next. Note how nested derivatives affect the indices accordingly:
\begin{align*}
&
\xrightarrow{\makebox[1cm]{}}
\left(
\frac
{
\partial_x
\phi_{i+\frac{1}{2}j}
-
\partial_x
\phi_{i-\frac{1}{2}j}
}
{h_x}
\right)
-
\left(
\frac
{
\partial_y
\phi_{ij+\frac{1}{2}}
-
\partial_y
\phi_{ij-\frac{1}{2}}
}
{h_y}
\right)
=
u_{ij}
\\
&
\xrightarrow{\makebox[1cm]{}}
\frac
{
\frac{\phi_{i+1j} - \phi_{ij}}{h_x}
-
\frac{\phi_{ij} - \phi_{i-1j}}{h_x}
}
{h_x}
-
\frac
{
\frac{\phi_{ij+1} - \phi_{ij}}{h_y}
-
\frac{\phi_{ij} - \phi_{ij-1}}{h_y}
}
{h_y}
=
u_{ij}
\end{align*}

\subsubsection*{Canonical Form}
After application of the discretization steps, the resulting equation is brought into canonical form, which means that the expression is factorized according to the unknowns $\phi_{ij}$:
\begin{align*}
\xrightarrow{\makebox[1cm]{}}
\frac{1}{h_x^2}\phi_{i+1j}
+\frac{1}{h_x^2}\phi_{i-1j}
+\frac{1}{h_y^2}\phi_{ij+1}
+\frac{1}{h_y^2}\phi_{ij-1}
-2\left(\frac{1}{h_x^2}+\frac{1}{h_y^2}\right)\phi_{ij}
=
u_{ij}
\end{align*}
\subsubsection*{Stencil Code Crafting}
In the next step, the canonical form is used to program the stencil code. This is done by interpreting the equation as a product between a row in the coefficient matrix $A$ and the vector of unknowns (the solution vector), resulting in the right hand side $u_{ij}$. The regular structure of the discretization reveals a pattern which is the same for every row and just changes the indicees per row. This is why it is called stencil code.
\TD{insert source code figure for simple stencil code}
The purpose of the stencil code is to be executed repeatedly in order to populate the system matrix $A$ and right hand side $\vec{b}$. Note that this requires a mapping of two-dimensional indices $ij$ to linear indices into columns of $A$. Also functions for accessing discretized input fields ($u$) and parameters, such as the voxelsize $h$, will be needed.

\subsubsection*{System Matrix Population and Solve}
The stencil code is compiled and executed for each row in $A$ to populate the system. Depending on the matrix-properties of $A$, different methods for solving the linear system $A\vec{x}=\vec{b}$ can be applied to find the solution $\vec{x}$, the discretized radiance field. This can then be used in a rendering application.
\TD{image overview of the numerical pipeline}

In the following subsections, the new solver will be introduced as as a general framwork for automization of the steps given above. The next section discusses how a computer algebra representation is used to represent the equation to be solved. Section~\ref{sec:pn_stencil_gen} shows how the spatial discretization and stencil code generation are executed using that representation. Section~\ref{sec:pn_bc} explains the aspects of the system which are related to boundary conditions, followed by section~\ref{sec:pn_system_matrix} which discusses how the generated system of linear equations is solved. Finally, section~\ref{sec:pn_staggered} is dedicated to staggered grids, which are necessary for the solver to produce useful results.

\subsection{Computer Algebra Representation}
\label{sec:pn_car}

At the core of our solver is a computer algebra representation of the equations which the system is trying to solve on a given domain. This representation is an expression tree. Each node within that tree can have any number of children and represents a constant, mathematical symbol or a mathematical operation, such as integration, derivation, power, addition or multiplication. Representations of complex equations are constructed by nesting nodes into a larger tree, where child nodes are considered arguments of the operation which is represented by the parent node.

There exists a rich number of computer algebra systems, such as SymPy~\cite{Meurer17}, which allow to generate and manipulate computer algebra representations like this. However, for this thesis a small lightweight computer algebra representation containing the required subset had been developed. Figure~\ref{fig:pn_math_expression_tree_generation} demonstrates how this system is used to build the expression trees for the examples above.
\begin{figure}[h]
\centering
\missingfigure{code listing showing how the expression tree is build}
\caption{Some caption}
\label{fig:pn_math_expression_tree_generation}
\end{figure}

Figure~\ref{fig:pn_math_expression_tree} shows the expression tree for a squared binom and the Laplace equation example used throughout this section.
\begin{figure}[h]
\centering
\missingfigure{image of a computer algebra representation of the laplace equation and binomial}
\caption{Some caption}
\label{fig:pn_math_expression_tree}
\end{figure}

Manipulations of mathematical expressions, such as expansions, the application of identities, substitution, factorization or rearranging of terms can be expressed as operations on the expression tree. Section~\ref{sec:pn_stencil_gen} will discuss how the spatial discretization is carried out as a manipulation of the expression tree (see figure~\ref{fig:pn_math_expression_tree_manipulation}).
\begin{figure}[h]
\centering
\missingfigure{image of representing binomial expansion as operation on expression tree}
\caption{Some caption}
\label{fig:pn_math_expression_tree_manipulation}
\end{figure}

Finally, after creation and manipulation, the final expression tree can be turned into another representation, such as latex expressions or source code. This is done by different rendering frontends which parse the expression tree and generate the appropriate output.
\begin{figure}[h]
\centering
\missingfigure{image showing rendering frontends for source code and latex}
\caption{Some caption}
\label{fig:pn_math_expression_tree_rendering}
\end{figure}

\subsection{Discretization and Stencil Generation}
\label{sec:pn_stencil_gen}

After expressing the $P_N$-equations in the computer algebra representation, the spatial discretization is carried out as a manipulation step on the mathematical expression tree. This is done by by parsing the tree from the root. The equation is supposed to be discretized at position $\vec{x}$, which initially coincides with the location $x_{ijk}$ in the discretized domain. This location is kept on a stack by the parser.

When a differential operator is encountered during tree parsing, the subtree representing the expression to be derived is instantiated twice according to the finite difference expression in equation~\ref{eq:pn_solver_central_difference} with appropriate weighting factors. The current position on the stack is adjusted and pushed on the stack again when decending down in the tree on each side of the central difference expression. This will produce higher order stencils as expected for nested differential operators.
\begin{figure}[h]
\centering
\missingfigure{image showing math expression tree for how the differential operator is being replaced with central difference. also mark the position on the stack.}
\caption{Some caption}
\label{fig:pn_discretization_differential}
\end{figure}

Once the parser arrives at the symbol $\vec{x}$ in the tree, it is replaced by its discrete counterpart. The top of the discrete position stack indicates where the unknown is expected to be evaluated relative to position $ijk$, at which the whole equation is being evaluated. If the position on top of the stack is the same, then just the unknown at $ijk$ is used to replace $\vec{x}$ in the expression. If the unknown is expected to be evaluated at a different location than $ijk$ (due to central differences), then $\vec{x}$ is being substituted by an expression which expresses the interpolation of the unknowns from the gridpoints (at which they are defined) to the position on top of the stack (see figure~\ref{fig:pn_discretization_interpolation}). This also makes sure, that our final expression only contains unknowns for which an element in the solution vector exists.
\begin{figure}[h]
\centering
\missingfigure{image showing interpolation.}
\caption{Some caption}
\label{fig:pn_discretization_interpolation}
\end{figure}

After applying the discretization to the expression tree, the equation is being factorized according to the unknowns ($\phi$ in the Laplace example). The factorization is applied as a sequence of manipulation operations to the expression tree, including application of the distributive law. A seperate step iterates over all resulting terms and merges coefficients from multiple instances of the same unknown, so that each unknown appears only once. 

After the discretized equation has been brought into canonical form, a seperate pass renders the expression tree into the stencil code. Then each term is analyzed and turned into source code.

If a term contains an unknown, then its factor expression subtree is rendered into a source code string. The symbols for voxelsize and other radiative quantities are replaced by function calls into an API which is being provided by the framework. Finally, the voxelspace coordinate associated with the unknown is used to find a discrete index into the finite difference grid, relative to the original coordinate $ijk$ at which the equation is being evaluated. This discrete index is being used to compute the column in $A$. With that, the assignment expression in the stencil code can be put together.
\begin{figure}[h]
\centering
\missingfigure{image showing code generation from expression tree in canonical form.}
\caption{Some caption}
\label{fig:pn_discretization_codegen}
\end{figure}
If the term does not contain an unknown, then the whole term is rendered into a single expression which is being assigned to the current row of the right hand side vector $\vec{b}$.

The stencil code is generated with the assumption of certain function calls being available. These are integrated into a single API which is being provided by the solver framework. This API allows the stencil to query the current row in the system for which it is being executed, voxelsize as well as functions for accessing discrete radiative transfer properties.
\begin{figure}[h]
\centering
\missingfigure{image showing the stencil API}
\caption{Some caption}
\label{fig:pn_discretization_codegen_stencilAPI}
\end{figure}


\subsection{Boundary Conditions}
\label{sec:pn_bc}
\TD{explain dirichlet. neumann BC}
\TD{explain the idea of coefficient index bending and how it is implemented}

\subsection{System Matrix Generation and Solution}
\label{sec:pn_system_matrix}
\TD{solver framework. classes, framework, inputs etc.}
\TD{describe properties of the global system matrix}
\TD{touch on eigen solver etc.}
\TD{introduce idea of normal form and how it relates to least squares rte. mention effect on stability}


\subsection{Staggered Grids}
\label{sec:pn_staggered}
% the idea is to motivate staggered grids and go over the whole pipeline again and introduce additional points related to staggered grids
\TD{show results from using colocated grids}
\TD{motivate staggered grids. show image explaining the problem with oscillating artefacts}
\TD{how to determine staggered grid locations}
\TD{staggered grids and stencil generation. mention the location stack and interpolation expressions.}
\TD{mention staggered grid locations effects in the code and archetecture of the solver. staggering coefficients will only affect interpolation expressions. therefore the code does not need to know? where does it need to know? unstaggering? boundary conditions?}
\TD{explain effect on boundary conditions and that it is important to have active coefficients in boundary voxels}
\TD{show correct result after using staggered grids}







\section{Rendering Integration and Results}
\label{sec:pn_results}
