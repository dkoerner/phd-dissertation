% ============================================================
\section{Spherical Harmonics}

Spherical harmonics are the Fourier series for the sphere and therefore as equally popular for spherical problems. They are often derived as eigensolutions to the surface Laplacian, which would be like developing the Fourier series as eigensolutions of the operator $(d/dx)^2$ on a finite line with the boundary conditions that $y$ and $dy/dx$ match at the two ends (see Riley et al.~\cite{Riley2006}). The result of this derivation are complex-valued functions $\SHBC$ which are defined as
\begin{align}
\label{eq:sh_definition}
\SHBC^{l,m}(\theta, \phi)=
\begin{cases}
C^{l,m}e^{im\phi}P^{l,m}\left(\operatorname{cos}\theta\right), & \text{for $m\ge0$}\\
\left(-1\right)^m\overline{\SHBC^{l\left|m\right|}}(\theta, \phi), & \text{for $m<0$}
\end{cases}
\quad ,
\end{align}
where $P^{l,m}$ are the associated Legendre polynomials. While those can be defined in many different ways, the most numerically robust way to evaluate them is by using the following set of recurrence relations (see Press et al.~\cite{Press07}):
\begin{align}
P^{0,0}\left(\operatorname{cos}\theta\right) &=
1
\  ,
\nonumber
\\
P^{m,m}\left(\operatorname{cos}\theta\right) &=
\left(2m-1\right)!!\left(1-\operatorname{cos}^2\theta\right)^\frac{m}{2}
\  ,
\nonumber
\\
P^{m+1,m}\left(\operatorname{cos}\theta\right) &=
\operatorname{cos}\theta\left(2m+1\right)P^{m,m}\left(\operatorname{cos}\theta\right)
\ 
\nonumber
\\
P^{l,m}\left(\operatorname{cos}\theta\right) &=
\frac{\operatorname{cos}\theta\left(2l-1\right)}{l-m}
P^{l-1,m}\left(\operatorname{cos}\theta\right)
-
\frac{l+m-1}{l-m}
P^{l-2,m}\left(\operatorname{cos}\theta\right)
\  .
\label{eq:sh_Plm}
\end{align}
The factor $C^{l,m}$ in equation~\ref{eq:sh_definition} is defined as
\begin{align}
\label{eq:sh_definition_C}
C^{l,m}=(-1)^m\sqrt{\frac{2l+1}{4\pi}\frac{(l-m)!}{(l+m)!}}
\end{align}
Since spherical harmonics are used so widely, their definition can vary and one has to be careful when comparing them across the literature. This concerns the $(-1)^m$ factor, which is called the Condon-Shortley phase. Sometimes, this factor is part of the definition of the associated Legendre polynomial $P^{l,m}$ and therefore does not appear in $C^{l,m}$. More importantly, the definition of $C^{l,m}$ depends on how the spherical harmonics are expected to be normalized.

\TD{figure of sh visualization}

\subsubsection*{Normalization and Orthogonality}

The coefficient $C^{l,m}$ has been defined, such that the spherical harmonics function $\SHBC$ scales to the unit norm
\begin{align}
\label{eq:sh_normalization}
\norm{\SHBC^{l,m}} = \sqrt{\left<\SHBC^{l,m}, \SHBC^{l,m}\right>} = 1
\ .
\end{align}
Other definitions define this norm differently and therefore arrive at different $C^{l,m}$. Since the spherical harmonics functions $\SHBC$ are complex-valued functions on the unit sphere, they are elements of the Hilbert space, for which the inner product is defined as
\begin{align}
\label{eq:sh_inner_product}
\left<f,g\right> = \int_\Omega f\left(\omega\right)\overline{g}\left(\omega\right)\ud\omega
\end{align}
with $\overline{g}$ being defined as the complex conjugate of $g$.

The spherical harmonics functions form a orthogonal family and thus we have
\begin{align}
\left<\SHBC^{l_1, m_1},\SHBC^{l_2, m_2}\right>
=
0
\ ,\quad\text{ for } l_1 \ne l_2 \text{ and } m_1 \ne m_2
\end{align}
The properties of normalization (equation~\ref{eq:sh_normalization}) and orthogonality (equation~\ref{eq:sh_orthogonality}) give
\begin{align}
\label{eq:sh_orthogonality}
\left<\SHBC^{l_1, m_1},\SHBC^{l_2, m_2}\right>=
\int_{\Omega} \SHBC^{l_1m_1}\left(\omega\right) \overline{\SHBC^{l_2m_2}}\left(\omega\right) \mathbf{d}\omega = \delta_{l_1m_1}\delta_{l_2m_2}
\ .
\end{align}

\subsubsection*{Projection and Reconstruction}

Because the spherical harmonics are orthonormal, a spherical function $f\left(\omega\right)$ can be projected into spherical harmonics basis function coefficients $f^{l,m}$ using the inner product from equation~\ref{eq:sh_inner_product}. This gives
\begin{align}
\label{eq:sh_projection}
f^{l, m} = \left<f,\SHBC^{l, m}\right> = 
\int_\Omega f\left(\omega\right)\overline{\SHBC^{l,m}}\left(\omega\right)\ud\omega
\ .
\end{align}
Given these coefficients, $f$ can be fully reconstructed by summing up the weighted contributions from the spherical harmonics basis functions
\begin{align}
f\left(\omega\right) = 
\sum_{l=0}^{N}
{
\sum_{m=-l}^{l}
{
f^{l,m}\left(\vec{x}\right)\SHBC^{l,m}\left(\omega\right)
}
}
\ .
\end{align}
The function can be fully reconstructed if $N=\infty$. However, as with the Fourier-expansion, it is useful to truncate the expansion at a specific $N$ (giving $P_N$-method its name). This limits the number of coefficients at the expense of introducing an approximation error by cutting off higher frequencies. Also note the use of the complex conjugate $\overline{\SHBC}$ for computing the coefficients $f^{l,m}$. This comes from the definition of the inner product (equation~\ref{eq:sh_inner_product}), which is a result of Hilber space theory.

\TD{projection operator}
\begin{align}
\label{eq:sh_projection_operator}
TODO
\end{align}
\TD{number of coefficients for truncation order N}

\subsubsection*{Frequency-invariant Rotation and Rotational Symmetry}

Applying a rotation to the argument of a spherical harmonics basis function of band $l$ can be expressed as a linear combination of spherical harmonics basis functions of the same order:
\begin{align}
\label{eq:sh_rotation}
\SHBC^{l,m}\left(R\omega\right)=\sum_{j=-l}^{l}r^{l,j}\SHBC^{l,j}\left(\omega\right)
\ .
\end{align}
This is derived from the fact that the spherical harmonics basis functions of order $l$ form an irreducible basis for the group of 3D-rotations (see Corollary $17.17$ in \cite{Hall13}). It implies that a rotation of a function represented in spherical harmonics does not introduce or loose any frequencies from or to other spherical harmonic bands and therefore is not subject to aliasing. Note that in the graphics literature spherical harmonics are often referred to as being rotationally invariant (e.g.~\cite{Wojciech08} or~\cite{Green03}), which is not correct.

The spherical harmonics projection of a rotationally symmetric function $f$ simplifies to Zonal Spherical Harmonics, which are characterized by the fact that only coefficients with $f^{l,0}$ are needed for reconstruction. This will be needed for the derivation of the $P_N$-equations. In particular, the phase function depends only on the angle between incident direction $\omega_i$ and outgoing direction $\omega_o$, which means that the phase function will be rotationally symmetric around the outgoing direction if it is fixed. 

A rotation $R(\alpha)$ of angle $\alpha$ around the pole axis is expressed in spherical harmonics as
\begin{align*}
\rho_{R(\alpha)}(\SHBC^{l,m}) = e^{-i m\alpha}\SHBC^{l,m}
\ .
\end{align*}
If a function $f$ is rotationally symmetric around the pole axis, we have:
\begin{align*}
\rho_{R(\alpha)}(f) = f
\end{align*}
and in spherical harmonics this would be:
\begin{align*}
\sum_{l,m}
{
e^{-i m\alpha}
f^{l,m}
\SHBC^{l,m} }\left(\omega\right)
=
\sum_{l,m}
{
p^{l,m}
\SHBC^{l,m}\left(\omega\right)
}
\end{align*}
By equating coefficients we get:
\begin{align*}
f^{l,m} = f^{l,m}e^{-i m\alpha}
\end{align*}
Since $e^{-i m\alpha}=1$ for all $\alpha$ only when $m=0$, we can conclude that $f^{l,m} = 0$ for all $m\ne0$. This means that for a function, which is rotationally symmetric around the pole axis, only the $m=0$ coefficients will be valid.

As mentioned earlier, this will be useful during the derivation of the $P_N$-equation and its scattering term in particular. If we fix the outgoing direction $\omega_o$ of our phase function at the north pole ($\omega_o=\vec{e}_3$), then the reconstruction requires only the spherical harmonics coefficients with $m=0$:
\begin{align}
\label{eq:sh_exp_phase}
p(\omega_i) =
\sum_l
{
p^{l0}
\SHBC^{l0}(\omega_i)
}
\end{align}

\subsubsection*{Recursive Relation}

Another important property which will be important for the derivation of the $P_N$-equations and its derivative term in particular is the following recursive relation
\begin{equation}
\label{eq:recursive_relation}
\resizebox{1.0\hsize}{!}{$\omega\overline{\SHBC^{l,m}} = \frac{1}{2}
\begin{pmatrix}
\ c^{l-1, m-1}\overline{\SHBC^{l-1,m-1}} - d^{l+1, m-1}\overline{\SHBC^{l+1,m-1}} - e^{l-1, m+1}\overline{\SHBC^{l-1,m+1}} + f^{l+1, m+1}\overline{\SHBC^{l+1,m+1}}\\
i\left(-c^{l-1, m-1}\overline{\SHBC^{l-1,m-1}} + d^{l+1, m-1}\overline{\SHBC^{l+1,m-1}} - e^{l-1, m+1}\overline{\SHBC^{l-1,m+1}} + f^{l+1, m+1}\overline{\SHBC^{l+1,m+1}}\right) \\
2\left(a^{l-1, m}\overline{\SHBC^{l-1,m}}+b^{l+1, m}\overline{\SHBC^{l+1,m}}\right)
\end{pmatrix}$}
\end{equation}
with
\begin{equation*}
\resizebox{1.0\hsize}{!}{$
a^{l,m}= \sqrt{\frac{\left(l-m+1\right)\left(l+m+1\right)}{\left(2l+1\right)\left(2l-1\right)}} \qquad
b^{l,m}= \sqrt{\frac{\left(l-m\right)\left(l+m\right)}{\left(2l+1\right)\left(2l-1\right)}} \qquad
c^{l,m}= \sqrt{\frac{\left(l+m+1\right)\left(l+m+2\right)}{\left(2l+3\right)\left(2l+1\right)}}
$}
\end{equation*}
\begin{equation*}
\resizebox{1.0\hsize}{!}{$
d^{l,m}= \sqrt{\frac{\left(l-m\right)\left(l-m-1\right)}{\left(2l+1\right)\left(2l-1\right)}} \qquad
e^{l,m}= \sqrt{\frac{\left(l-m+1\right)\left(l-m+2\right)}{\left(2l+3\right)\left(2l+1\right)}} \qquad
f^{l,m}= \sqrt{\frac{\left(l+m\right)\left(l+m-1\right)}{\left(2l+1\right)\left(2l-1\right)}}
$}
\end{equation*}
Such recursion relations seem to be hard to find in the standard literature and are preserved through citations in relevant articles\footnote{according to email exchange with experts from nuclear sciences}, such as Seibold et al.~\cite{Seibold14} or Brunner et al.~\cite{Brunner05}. Note that the signs for the $x$- and $y$- component depend on the handedness of the coordinate system in which the spherical harmonics basis functions are defined.

%\subsubsection*{Expansion of a Delta Function}
\TD{expanding delta function, used for light sources (Q)}
\TD{eigenfunction property}
\TD{product integrals etc. as needed}
