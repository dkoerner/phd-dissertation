\chapter{Foundations of Light Transport Simulation}
\label{sec:foundations}

% ============================================================
\section{Quantities in Radiative Transfer}

% ============================================================
\section{The Radiative Transfer Equation}


\todo[inline]{Derive the radiative transfer equation terms}
\todo[inline]{Advection}
\todo[inline]{absorption}
\todo[inline]{scattering (outscattering)}
\todo[inline]{scattering (inscattering)}
\todo[inline]{phase function}
\todo[inline]{isotropic phase function}
\todo[inline]{anisotropic phase function}

\missingfigure{visualization of inscattering, outscattering and absorption.}

\todo[inline]{Discuss wavelength dependence}

\todo[inline]{Discuss time dependence and steady state}

\todo[inline]{discuss isotropic medium, anisotropic medium}

\todo[inline]{operator notation}

% ============================================================
\section{Non-deterministic Methods}

\subsection{Ray-marching and Distributed Raytracing}
\subsection{Path-integral Formulation}
\subsection{Monte-Carlo Method}
\todo[inline]{explain incremental path construction}
\todo[inline]{explain the concept of variance}
\todo[inline]{explain the concept of bias}
\subsection{Variance Reduction Techniques}
\todo[inline]{Explain, that a great variety of Monte Carlo methods exist because there are many ways information and assumptions can be combined with the basic variance reduction techniques available for monte carlo integration}
\todo[inline]{Local variance reduction technique}
\todo[inline]{semi-global reduction technique}
\todo[inline]{global reduction technique}
\subsubsection{Importance Sampling}
\todo[inline]{explain importance sampling basics}
\todo[inline]{multiple importance sampling}
\todo[inline]{explain JIS as a semi-global technique}
\todo[inline]{explain SNEE as a semi-global technique}
\subsubsection{Control Variates}
\todo[inline]{explain control variates}
\todo[inline]{multi-level monte carlo}
\todo[inline]{residual ratio tracking (precomputing the maximum volume height for volume bricks as semi-global))}
\subsubsection{Russian Roulette and Splitting}
\todo[inline]{Point out, }
\todo[inline]{Point out, that the different variance reduction techniques all }

% ============================================================
\section{Deterministic Methods}

\subsection{Ad-hoc and Heuristical Methods}
\todo[inline]{What constitutes a heuristic method? Not directly connected to theory.}
\todo[inline]{Light Propagation Volumes}
\todo[inline]{Anton Bouthours stuff}
\todo[inline]{Wavefront stuff from Dreamworks}

\subsection{Discretized Radiative Transfer}
\todo[inline]{Point out difference to ad-hoc methods (that it is grounded to theory)}
\todo[inline]{mention isotropic medium, anisotropic medium}
\todo[inline]{Spatial Discretization}
\todo[inline]{Finite differences}
\todo[inline]{Finite elements}
\todo[inline]{Angular Discretization}
\todo[inline]{Lattice Boltzmann, Discrete Ordinates, Spherical Harmonics}
\missingfigure{overview of all angular discretization methods.}

\todo[inline]{reiterate, that in this thesis we are concerned with methods using SH based discretization of angular variable.}


% ============================================================
\subsection{Spherical Harmonics Discretization in Angular Domain}
\subsubsection{Spherical Harmonics Basis and Properties}
\subsubsection{Complex-valued $P_N$-equations}